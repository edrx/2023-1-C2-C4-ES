% (find-LATEX "2023-1-C2-edovs.tex")
% (defun c () (interactive) (find-LATEXsh "lualatex -record 2023-1-C2-edovs.tex" :end))
% (defun C () (interactive) (find-LATEXsh "lualatex 2023-1-C2-edovs.tex" "Success!!!"))
% (defun D () (interactive) (find-pdf-page      "~/LATEX/2023-1-C2-edovs.pdf"))
% (defun d () (interactive) (find-pdftools-page "~/LATEX/2023-1-C2-edovs.pdf"))
% (defun e () (interactive) (find-LATEX "2023-1-C2-edovs.tex"))
% (defun o () (interactive) (find-LATEX "2023-1-C2-edovs.tex"))
% (defun u () (interactive) (find-latex-upload-links "2023-1-C2-edovs"))
% (defun v () (interactive) (find-2a '(e) '(d)))
% (defun d0 () (interactive) (find-ebuffer "2023-1-C2-edovs.pdf"))
% (defun cv () (interactive) (C) (ee-kill-this-buffer) (v) (g))
%          (code-eec-LATEX "2023-1-C2-edovs")
% (find-pdf-page   "~/LATEX/2023-1-C2-edovs.pdf")
% (find-sh0 "cp -v  ~/LATEX/2023-1-C2-edovs.pdf /tmp/")
% (find-sh0 "cp -v  ~/LATEX/2023-1-C2-edovs.pdf /tmp/pen/")
%     (find-xournalpp "/tmp/2023-1-C2-edovs.pdf")
%   file:///home/edrx/LATEX/2023-1-C2-edovs.pdf
%               file:///tmp/2023-1-C2-edovs.pdf
%           file:///tmp/pen/2023-1-C2-edovs.pdf
%  http://anggtwu.net/LATEX/2023-1-C2-edovs.pdf
% (find-LATEX "2019.mk")
% (find-Deps1-links "Caepro5 Piecewise1")
% (find-Deps1-cps   "Caepro5 Piecewise1")
% (find-Deps1-anggs "Caepro5 Piecewise1")
% (find-MM-aula-links "2023-1-C2-edovs" "C2" "c2m231edovs" "c2ev")

% «.defs»		(to "defs")
% «.defs-caepro»	(to "defs-caepro")
% «.defs-pict2e»	(to "defs-pict2e")
% «.title»		(to "title")
% «.links»		(to "links")
%
% «.djvuize»		(to "djvuize")



% <videos>
% Video (not yet):
% (find-ssr-links     "c2m231edovs" "2023-1-C2-edovs")
% (code-eevvideo      "c2m231edovs" "2023-1-C2-edovs")
% (code-eevlinksvideo "c2m231edovs" "2023-1-C2-edovs")
% (find-c2m231edovsvideo "0:00")

\documentclass[oneside,12pt]{article}
\usepackage[colorlinks,citecolor=DarkRed,urlcolor=DarkRed]{hyperref} % (find-es "tex" "hyperref")
\usepackage{amsmath}
\usepackage{amsfonts}
\usepackage{amssymb}
\usepackage{pict2e}
\usepackage[x11names,svgnames]{xcolor} % (find-es "tex" "xcolor")
\usepackage{colorweb}                  % (find-es "tex" "colorweb")
%\usepackage{tikz}
%
% (find-dn6 "preamble6.lua" "preamble0")
%\usepackage{proof}   % For derivation trees ("%:" lines)
%\input diagxy        % For 2D diagrams ("%D" lines)
%\xyoption{curve}     % For the ".curve=" feature in 2D diagrams
%
\usepackage{edrx21}               % (find-LATEX "edrx21.sty")
\input edrxaccents.tex            % (find-LATEX "edrxaccents.tex")
\input edrx21chars.tex            % (find-LATEX "edrx21chars.tex")
\input edrxheadfoot.tex           % (find-LATEX "edrxheadfoot.tex")
\input edrxgac2.tex               % (find-LATEX "edrxgac2.tex")
%\usepackage{emaxima}              % (find-LATEX "emaxima.sty")
%
% (find-es "tex" "geometry")
\usepackage[a6paper, landscape,
            top=1.5cm, bottom=.25cm, left=1cm, right=1cm, includefoot
           ]{geometry}
%
\begin{document}

% «defs»  (to ".defs")
% (find-LATEX "edrx21defs.tex" "colors")
% (find-LATEX "edrx21.sty")

\def\drafturl{http://anggtwu.net/LATEX/2023-1-C2.pdf}
\def\drafturl{http://anggtwu.net/2023.1-C2.html}
\def\draftfooter{\tiny \href{\drafturl}{\jobname{}} \ColorBrown{\shorttoday{} \hours}}

% (find-LATEX "2023-1-C2-carro.tex" "defs-caepro")
% (find-LATEX "2023-1-C2-carro.tex" "defs-pict2e")

\catcode`\^^J=10
\directlua{dofile "dednat6load.lua"}  % (find-LATEX "dednat6load.lua")

% «defs-caepro»  (to ".defs-caepro")
%L dofile "Caepro5.lua"              -- (find-angg "LUA/Caepro5.lua" "LaTeX")
\def\Caurl   #1{\expr{Caurl("#1")}}
\def\Cahref#1#2{\href{\Caurl{#1}}{#2}}
\def\Ca      #1{\Cahref{#1}{#1}}

% «defs-pict2e»  (to ".defs-pict2e")
%L V = nil                           -- (find-angg "LUA/Pict2e1.lua" "MiniV")
%L dofile "Piecewise1.lua"           -- (find-LATEX "Piecewise1.lua")
%L Pict2e.__index.suffix = "%"
\def\pictgridstyle{\color{GrayPale}\linethickness{0.3pt}}
\def\pictaxesstyle{\linethickness{0.5pt}}
\def\pictnaxesstyle{\color{GrayPale}\linethickness{0.5pt}}
\celllower=2.5pt

\pu



%  _____ _ _   _                               
% |_   _(_) |_| | ___   _ __   __ _  __ _  ___ 
%   | | | | __| |/ _ \ | '_ \ / _` |/ _` |/ _ \
%   | | | | |_| |  __/ | |_) | (_| | (_| |  __/
%   |_| |_|\__|_|\___| | .__/ \__,_|\__, |\___|
%                      |_|          |___/      
%
% «title»  (to ".title")
% (c2m231edovsp 1 "title")
% (c2m231edovsa   "title")

\thispagestyle{empty}

\begin{center}

\vspace*{1.2cm}

{\bf \Large Cálculo 2 - 2023.1}

\bsk

Aulas 21 e 22: EDOs com variáveis separáveis.

\bsk

Eduardo Ochs - RCN/PURO/UFF

\url{http://anggtwu.net/2023.1-C2.html}

\end{center}

\newpage

% «links»  (to ".links")
% (c2m222edovsp 1 "title")
% (c2m222edovsa   "title")


{\bf Links}

\scalebox{0.8}{\def\colwidth{12cm}\firstcol{

Algumas figuras de campos de direções:

\Ca{DiffyQsP27},
\Ca{Stew9p9},
\Ca{2eT214}

\ssk

\par EDOs por chutar e testar:
\par \Ca{2gT40}, \Ca{2dT13}

\ssk

\par \Ca{2dT293} Material sobre EDOVSs de 2021.2
\par \Ca{2dT306} Slides sobre inversas de 2021.2
\par \Ca{Leit7} Funções inversas, logarítmicas e exponenciais

\ssk

\par Questões sobre EDOVSs nas provas de 2022.2:
\par \Ca{2fT123}, \Ca{2fT126} P2, gabarito
\par \Ca{2fT135}, \Ca{2fT137} VS, anexo

\ssk

\par \Ca{ZillCullenInicioP13} (p.6) Soluções implícitas e explícitas
\par \Ca{ZillCullenInicioP16} (p.9) parâmetros, solução particular
\par \Ca{ZillCullenInicioP51} (p.44) 2.2: Variáveis separáveis
\par \Ca{Stew9p18} (p.618) 9.3: Separable equations

\bsk

\par \Ca{2gQ41} Quadros da aula 21 (13/jun/2023)
\par \Ca{2gQ43} Quadros da aula 22 (16/jun/2023)

}\anothercol{
}}


% (c2m211edovsp 1 "title")
% (c2m211edovsa   "title")


% (find-books "__analysis/__analysis.el" "boyce-diprima")
% (find-books "__analysis/__analysis.el" "boyce-diprima" "2.2")
% (find-boycediprima11page (+ 14 33) "2.2 Separable Differential Equations")
% (find-books "__analysis/__analysis.el" "lebl")
% (find-books "__analysis/__analysis.el" "lebl" "1.3")
% (find-diffyqspage 33 "1.3 Separable equations")
% (find-books "__analysis/__analysis.el" "stewart")
% (find-books "__analysis/__analysis.el" "stewart" "9.3")
% (find-stewart7page (+ 32 618) "9.3 Separable Equations")
% (find-books "__analysis/__analysis.el" "thomas")
% (find-books "__analysis/__analysis.el" "thomas" "9.1")
% (find-thomas11-1page (+ 108 642) "9.1 Slope Fields and Separable Differential Equations")
% (find-books "__analysis/__analysis.el" "trench")
% (find-books "__analysis/__analysis.el" "trench" "2.2")
% (find-trenchpage (+ 10 45) "2.2 Separable Equations")
% Stew9p18

\newpage

{\bf Inversas: introdução}

\scalebox{0.7}{\def\colwidth{16cm}\firstcol{

Dê uma olhada nestes links:

\Ca{ZillCullenInicioP13} (p.6) Soluções implícitas e explícitas

\Ca{ZillCullenInicioP16} (p.9) parâmetros, solução particular

\Ca{ZillCullenInicioP51} (p.44) 2.2: Variáveis separáveis

\ssk

O método pra resolver EDOs com variáveis separáveis nos dá primeiro
``soluções implícitas'', como $x^2+y^2=C$ or $x^2+y^2=42$, e aí depois
disso a gente tem que transformar essas soluções implícitas em
``soluções explícitas'', em que $y$ é uma função de $x$... por
exemplo:
%
$$\begin{array}{rcl}
  x =   \sqrt{C-x^2} &⇒& f_1(x)=\sqrt{C-x^2} \\
  x =  -\sqrt{C-x^2} &⇒& f_2(x)=-\sqrt{C-x^2} \\
  x =  \sqrt{42-x^2} &⇒& f_3(x)=\sqrt{42-x^2} \\
  x = -\sqrt{42-x^2} &⇒& f_4(x)=-\sqrt{42-x^2} \\
  \end{array}
$$

Praticamente todo mundo se enrola na hora de passar das ``soluções
implícitas'' pras ``soluções implícitas'', principalmente nos casos em
que a gente tem ``várias inversas''...

\ssk

Eu vou usar uma terminologia que é meio errada, e vou dizer que
$g_1(y)=\sqrt{y}$ e $g_2(y)=-\sqrt{y}$ são duas inversas diferentes
para $f(x)=x^2$. Um bom lugar pra aprender a terminologia correta --
que precisa que a gente especifique os domínios! -- é o capítulo 7 do
Leithold: \Ca{Leit7}.

}\anothercol{
}}


\newpage

{\bf Inversas: um exemplo complicado}

\scalebox{0.48}{\def\colwidth{11.5cm}\firstcol{

Digamos que queremos inverter esta função:
%
$$f(x) = (x+3)^4+5
$$

O método é este aqui, mas repare que ele tem uma bifurcação...

\def\MA{
             y  &=& (x+3)^4+5 \\
           y-5  &=& (x+3)^4 \\
  \sqrt[4]{y-5} &=& \sqrt[4]{(x+3)^4} \\
}
\def\MB{
  \sqrt[4]{y-5} &=& x+3 \\
  -3 + \sqrt[4]{y-5} &=& x \\
}
\def\MC{
      \sqrt[4]{y-5}  &=& -(x+3) \\
      \sqrt[4]{y-5}  &=& -x-3 \\
      \sqrt[4]{y-5}  &=& -x-3 \\
    3+\sqrt[4]{y-5}  &=& -x \\
  -(3+\sqrt[4]{y-5}) &=& x \\
}

\msk

$\begin{array}{c}
    \begin{array}{rcl}\MA\end{array} \\ \\
    \begin{array}[t]{rcl}\MB\end{array}
    \begin{array}[t]{rcl}\MC\end{array} \\
  \end{array}
$

\bsk

Se a gente segue o caminho da esquerda a gente obtém
$$f^{-1}(y) = -3 + \sqrt[4]{y-5},$$
e se a gente segue o caminho da direita a gente obtém
$$f^-1(y)=-(3+\sqrt[4]{y-5}).$$

}\anothercol{

%\vspace*{0.25cm}

Sabemos que $\sqrt[4]{α^4} = |α|$, e portanto:
%
$$\begin{array}{rcl}
  α≥0 &⇒& \sqrt[4]{α^4} = α \\
  α≤0 &⇒& \sqrt[4]{α^4} = -α \\
  x+3≥0 &⇒& \sqrt[4]{(x+3)^4} = x+3 \\
  x+3≤0 &⇒& \sqrt[4]{(x+3)^4} = -(x+3) \\
  \end{array}
$$

Ou seja, nas contas à esquerda se $x+3≥0$ nós temos que seguir o
caminho da esquerda, e se $x+3≤0$ nós temos que seguir o caminho da
direita.

\ssk

O melhor modo da gente entender essas duas inversas é esse aqui.
Considere estes três conjuntos de $\R^2$:
%
$$\begin{array}{rcl}
  A_1 &=& \setofxyst{y=(x+3)^4+5}\\
  A_2 &=& \setofxyst{y=(x+3)^4+5, \; x+3≥0}\\
  A_3 &=& \setofxyst{y=(x+3)^4+5, \; x+3≤0}\\
  \end{array}
$$

Os conjuntos $A_2$ e $A_3$ são gráficos de funções inversíveis e $A_1$
é o gráfico de uma função não-inversível. Os domínios dessas funções
são relativamente fáceis de calcular -- eles são $\R$,
$\setofst{x∈\R}{x+3≥0}$ e $\setofst{x∈\R}{x+3≤0}$ respectivamente --
mas as imagens são um pouco mais complicadas...

\msk

...mas lembre que em C2 a gente costuma fazer as contas em duas
etapas: na primeira etapa a gente finge que as hipóteses vão ser todas
obedecidas e a gente nem escreve quais são essas hipóteses, e só na
segunda etapa a gente escreve explicitamente quais são essas hipóteses
e a gente vê se tudo realmente dá certo quando elas são obedecidas.
{\sl E neste curso a gente raramente vai ter tempo pra segunda etapa.}

}}







\GenericWarning{Success:}{Success!!!}  % Used by `M-x cv'

\end{document}


% Local Variables:
% coding: utf-8-unix
% ee-tla: "c2ev"
% ee-tla: "c2m231edovs"
% End:
