% (find-LATEX "2023-1-C2-carro.tex")
% (defun c () (interactive) (find-LATEXsh "lualatex -record 2023-1-C2-carro.tex" :end))
% (defun C () (interactive) (find-LATEXsh "lualatex 2023-1-C2-carro.tex" "Success!!!"))
% (defun D () (interactive) (find-pdf-page      "~/LATEX/2023-1-C2-carro.pdf"))
% (defun d () (interactive) (find-pdftools-page "~/LATEX/2023-1-C2-carro.pdf"))
% (defun e () (interactive) (find-LATEX "2023-1-C2-carro.tex"))
% (defun o () (interactive) (find-LATEX "2023-1-C2-carro.tex"))
% (defun u () (interactive) (find-latex-upload-links "2023-1-C2-carro"))
% (defun v () (interactive) (find-2a '(e) '(d)))
% (defun d0 () (interactive) (find-ebuffer "2023-1-C2-carro.pdf"))
% (defun cv () (interactive) (C) (ee-kill-this-buffer) (v) (g))
%          (code-eec-LATEX "2023-1-C2-carro")
% (find-pdf-page   "~/LATEX/2023-1-C2-carro.pdf")
% (find-sh0 "cp -v  ~/LATEX/2023-1-C2-carro.pdf /tmp/")
% (find-sh0 "cp -v  ~/LATEX/2023-1-C2-carro.pdf /tmp/pen/")
%     (find-xournalpp "/tmp/2023-1-C2-carro.pdf")
%   file:///home/edrx/LATEX/2023-1-C2-carro.pdf
%               file:///tmp/2023-1-C2-carro.pdf
%           file:///tmp/pen/2023-1-C2-carro.pdf
%  http://anggtwu.net/LATEX/2023-1-C2-carro.pdf
% (find-LATEX "2019.mk")
% (find-sh0 "cd ~/LUA/; cp -v Pict2e1.lua Pict2e1-1.lua Piecewise1.lua ~/LATEX/")
% (find-sh0 "cd ~/LUA/; cp -v Pict2e1.lua Pict2e1-1.lua Pict3D1.lua ~/LATEX/")
% (find-sh0 "cd ~/LUA/; cp -v Gram2.lua Tree1.lua Caepro5.lua ~/LATEX/")
% (find-Deps1-links "Caepro5 Piecewise1")
% (find-Deps1-cps   "Caepro5 Piecewise1")
% (find-Deps1-anggs "Caepro5 Piecewise1")
% (find-MM-aula-links "2023-1-C2-carro" "C2" "c2m231carro" "c2ca")

% «.defs»		(to "defs")
% «.defs-caepro»	(to "defs-caepro")
% «.defs-pict2e»	(to "defs-pict2e")
% «.title»		(to "title")
% «.introducao»		(to "introducao")
% «.links»		(to "links")
% «.exercicio-1»	(to "exercicio-1")
% «.exercicio-1-dicas»	(to "exercicio-1-dicas")
% «.exercicio-2»	(to "exercicio-2")
% «.exercicio-5»	(to "exercicio-5")



% <videos>
% Video (not yet):
% (find-ssr-links     "c2m231carro" "2023-1-C2-carro")
% (code-eevvideo      "c2m231carro" "2023-1-C2-carro")
% (code-eevlinksvideo "c2m231carro" "2023-1-C2-carro")
% (find-c2m231carrovideo "0:00")

\documentclass[oneside,12pt]{article}
\usepackage[colorlinks,citecolor=DarkRed,urlcolor=DarkRed]{hyperref} % (find-es "tex" "hyperref")
\usepackage{amsmath}
\usepackage{amsfonts}
\usepackage{amssymb}
\usepackage{pict2e}
\usepackage[x11names,svgnames]{xcolor} % (find-es "tex" "xcolor")
\usepackage{colorweb}                  % (find-es "tex" "colorweb")
%\usepackage{tikz}
%
% (find-dn6 "preamble6.lua" "preamble0")
%\usepackage{proof}   % For derivation trees ("%:" lines)
%\input diagxy        % For 2D diagrams ("%D" lines)
%\xyoption{curve}     % For the ".curve=" feature in 2D diagrams
%
\usepackage{edrx21}               % (find-LATEX "edrx21.sty")
\input edrxaccents.tex            % (find-LATEX "edrxaccents.tex")
\input edrx21chars.tex            % (find-LATEX "edrx21chars.tex")
\input edrxheadfoot.tex           % (find-LATEX "edrxheadfoot.tex")
\input edrxgac2.tex               % (find-LATEX "edrxgac2.tex")
%\usepackage{emaxima}              % (find-LATEX "emaxima.sty")
%
% (find-es "tex" "geometry")
\usepackage[a6paper, landscape,
            top=1.5cm, bottom=.25cm, left=1cm, right=1cm, includefoot
           ]{geometry}
%
\begin{document}


% «defs»  (to ".defs")
% (find-LATEX "edrx21defs.tex" "colors")
% (find-LATEX "edrx21.sty")
\def\u#1{\par{\footnotesize \url{#1}}}
\def\drafturl{http://anggtwu.net/LATEX/2023-1-C2.pdf}
\def\drafturl{http://anggtwu.net/2023.1-C2.html}
\def\draftfooter{\tiny \href{\drafturl}{\jobname{}} \ColorBrown{\shorttoday{} \hours}}


\catcode`\^^J=10
\directlua{dofile "dednat6load.lua"}  % (find-LATEX "dednat6load.lua")

% «defs-caepro»  (to ".defs-caepro")
%L dofile "Caepro5.lua"              -- (find-angg "LUA/Caepro5.lua" "LaTeX")
\def\Caurl   #1{\expr{Caurl("#1")}}
\def\Cahref#1#2{\href{\Caurl{#1}}{#2}}
\def\Ca      #1{\Cahref{#1}{#1}}

% «defs-pict2e»  (to ".defs-pict2e")
%L V = nil                           -- (find-angg "LUA/Pict2e1.lua" "MiniV")
%L dofile "Piecewise1.lua"           -- (find-LATEX "Piecewise1.lua")
%L Pict2e.__index.suffix = "%"
\def\pictgridstyle{\color{GrayPale}\linethickness{0.3pt}}
\def\pictaxesstyle{\linethickness{0.5pt}}
\def\pictnaxesstyle{\color{GrayPale}\linethickness{0.5pt}}
\celllower=2.5pt

\pu




%  _____ _ _   _                               
% |_   _(_) |_| | ___   _ __   __ _  __ _  ___ 
%   | | | | __| |/ _ \ | '_ \ / _` |/ _` |/ _ \
%   | | | | |_| |  __/ | |_) | (_| | (_| |  __/
%   |_| |_|\__|_|\___| | .__/ \__,_|\__, |\___|
%                      |_|          |___/      
%
% «title»  (to ".title")
% (c2m231carrop 1 "title")
% (c2m231carroa   "title")

\thispagestyle{empty}

\begin{center}

\vspace*{1.2cm}

{\bf \Large Cálculo 2 - 2023.1}

\bsk

Aulas 1 e 3: integração e derivação

com o mathologermóvel

\bsk

Eduardo Ochs - RCN/PURO/UFF

\url{http://anggtwu.net/2023.1-C2.html}

\end{center}

\newpage

% «introducao»  (to ".introducao")
% (c2m231carrop 2 "introducao")
% (c2m231carroa   "introducao")

{\bf Introdução}

\scalebox{0.7}{\def\colwidth{12cm}\firstcol{

Nesta parte do curso nós vamos tentar entender

este trecho do vídeo do Mathologer,

\ssk

\Ca{CalcEasy03:19} até 12:47

\bsk

e vamos fazer alguns exercícios --

que podem ser feitos em vários níveis de detalhe.

Leia estes trechos das legendas de uns vídeos meus:

\ssk

\par \Ca{Slogans01:10} até 08:51: sobre chutar e testar
\par \Ca{Slogans07:17} até 07:48: ...do tamanho de um apartamento
\par \Ca{Visaud45:14} até 52:24: ajustar o nível de detalhe
\par \Ca{Slogans1:11:02} até 1:17:42: seja o seu prório Geogebra
\par \Ca{Slogans1:39:46} até 1:45:02: ...com quem vale a pena estudar

\bsk

Leia também estes slides:

\par \Ca{2gT4}  (intro, p.3) ``Releia a Dica 7''
\par \Ca{2gT13} (intro, p.12) Sobre Português
\par \Ca{2gT14} (intro, p.13) Sobre Português (2)
\par \Ca{2gT16} (intro, p.15) Unexpected end of input
\par \Ca{2gT19} (intro, p.18) Retas reversas

}\anothercol{
}}



\newpage


% «links»  (to ".links")
% (c2m231carrop 99 "links")
% (c2m231carroa    "links")

{\bf Links}

\scalebox{0.9}{\def\colwidth{12cm}\firstcol{

Os slides das próximas páginas são versões ligeiramente

reescritas destes slides de outros semestres:

\ssk

\par \Ca{2fT17} (mathologermovel, p.3) Item 3
\par \Ca{2fT18} (mathologermovel, p.4) Item 4
\par \Ca{2eT62} (TFC1, p.3) Algumas propriedades da integral
\par \Ca{2eT66} (TFC1, p.7) Exercício 1
\par \Ca{2eT69} (TFC1, p.10) A função G(x) é esta aqui
\par \Ca{2dT225} (MT3, p.4) Uma espécie de gabarito
\par \Ca{2eT199} (P1, p.7) eu defini as funções f e g desta forma
\par \Ca{2eT200} (P1, p.8) gabarito

}\anothercol{
}}


\newpage

% «exercicio-1»  (to ".exercicio-1")
% (c2m231carrop 4 "exercicio-1")
% (c2m231carroa   "exercicio-1")
% (c2m221tfc1p 10 "exercicio-4")
% (c2m221tfc1a    "exercicio-4")

{\bf Exercício 1.}

\scalebox{1.0}{\def\colwidth{10.5cm}\firstcol{

Seja $G(x)$ esta função:
%
%L putcellat = function (xy, str) return pformat("\\put%s{\\cell{%s}}", xy, str) end
%L Pict2e.bounds = PictBounds.new(v(0,-2), v(15,5))
%L spec = 
%L   "(0,-1)--(1,-1)--(2,0)--(3,-1)--(3.5,0)--" ..
%L   "(4,1)--(5,0)--(6,1)--(8,1)--(9,5)--(10,2)--(11,4)--(12,3)--(13,4)--(15,2)"
%L pws = PwSpec.from(spec)
%L p = PictList {
%L   pws:topict():prethickness("1.5pt"),
%L   putcellat(v(5, -0.7), "5"),
%L   putcellat(v(10,-0.7), "10")
%L }
%L p:pgat("pgatc"):sa("Ex 4"):output()
\pu
%
$$G(x) \;\;=\;\;\,
    \unitlength=15pt
    \scalebox{0.7}{$\ga{Ex 4}$}
$$

Relembre como calcular coeficientes angulares e derivadas

no olhômetro e faça um gráfico da função $G'(x)$.

\ssk

Dica 1: $G'(3.5)=2$.

Dica 2: $G'(4)$ não existe --- use uma bolinha

vazia pra representar isso no seu gráfico.

}\anothercol{
}}

\newpage

% «exercicio-1-dicas»  (to ".exercicio-1-dicas")
% (c2m231carrop 5 "exercicio-1-dicas")
% (c2m231carroa   "exercicio-1-dicas")
% (c2m221tfc1p 11 "exercicio-4-dicas")
% (c2m221tfc1a    "exercicio-4-dicas")

{\bf Exercício 1: mais dicas}

\scalebox{1.0}{\def\colwidth{9cm}\firstcol{

\vspace*{0.0cm}

\par \Ca{Leit1p18} (p.17: inclinação)
\par \Ca{Miranda66} (Capítulo 3: Derivadas)
\par \Ca{Miranda22} (Seção 1.4: Limites laterais)
\par \Ca{Miranda74} (Seção 3.2.3: Derivadas laterais)

\msk

\par \Ca{2eT70} (TFC1, p.11) Dicas pro exercício 4

}\anothercol{
}}

\newpage

% «exercicio-2»  (to ".exercicio-2")
% (c2m231carrop 6 "exercicio-2")
% (c2m231carroa   "exercicio-2")
% (c2m221tfc1p 7 "exercicio-1")
% (c2m221tfc1a   "exercicio-1")

{\bf Exercício 2.}

%L Pict2e.bounds = PictBounds.new(v(0,-2), v(7,4))
%L spec = "(0,0)--(1,0)o (1,2)c--(2,2)o (2,3)c--(4,3)c (4,-1)o--(6,-1)o (6,0)c--(7,0)"
%L pws = PwSpec.from(spec)
%L pws:topict():prethickness("1.5pt"):pgat("pgatc"):sa("Ex 1"):output()
\pu

\unitlength=7.5pt

\ssk

Seja $f(x) \; = \;\; \ga{Ex 1}$ \; .

Note que:

$\Intx{1}{2}{f(x)} = 2·(2-1)$,

$\Intx{3}{4}{f(x)} = 3·(4-3)$,

$\Intx{4}{6}{f(x)} = -1·(6-4)$,

\msk

Calcule:

a) $\Intx{1.5}{2}{f(x)}$

b) $\Intx{2}{4}{f(x)}$

c) $\Intx{1.5}{4}{f(x)}$

d) $\Intx{1.5}{6}{f(x)}$




\newpage

{\bf Exercício 3.}

\msk

Sejam $f(x) \; = \;\; \ga{Ex 1}$

e $F(β) = \Intx{2}{β}{f(x)}$.

\msk

a) Calcule $F(2), F(2.5), F(3), \ldots, F(6)$.

b) Calcule $F(1.5), F(1), F(0.5), F(0)$.


\newpage

% (c2m221tfc1p 9 "exercicio-3")
% (c2m221tfc1a   "exercicio-3")

{\bf Exercício 4.}

No exercício 3 você obteve alguns valores da função $F(β)$,

mas não todos... por exemplo, você {\sl ainda} não calculou $F(2.1)$.

\msk

a) Desenhe num gráfico só todos os pontos $(x,F(x))$

que você calculou nos itens (a) e (b) do exercício 3.

Dica: o conjunto que você quer desenhar é este aqui:

$\{(0,F(0)), \, (0.5,F(0.5)), \ldots, (6,F(6))\}$.

\msk

b) Tente descobrir --- lendo os próximos slides, assitindo

o vídeo, e discutindo com os seus colegas --- qual é o jeito

certo de ligar os pontos do item (a).



\newpage

% «exercicio-5»  (to ".exercicio-5")

{\bf Exercício 5.}

% 2fT18 (c2m222mmp 4 "item-4")
%       (c2m222mma   "item-4")
% (c2m221p1p 7 "escadas")
% (c2m221p1a   "escadas")

%L hx = function (x, y) return format(" (%s,%s)c--(%s,%s)o", x-1,y, x,y) end
%L hxs = function (ys)
%L     local str = ""
%L     for x,y in ipairs(ys) do str = str .. hx(x, y) end
%L     return str
%L   end
%L mtintegralspec = function (specf, xmax, y0)
%L     local pws = PwSpec.from(specf)
%L     local f = pws:fun()
%L     local ys = {[0] = y0}
%L     for x=1,xmax do
%L       PP("FOO", x, f(x-0.5), ys)
%L       ys[x] = ys[x - 1] + f(x - 0.5)
%L     end
%L     local strx = function (x) return tostring(v(x, ys[x])) end
%L     local specF = mapconcat(strx, seq(0, xmax), "--")
%L     return specF
%L   end
%L
%L ysf   = {1, 2, 1, 0, -1, -2, -1, 0, 1, 2, 1, 0}
%L specf = hxs(ysf)
%L ysg   = {0, 1, 2, 3, -2, -1, 0, -1, -2, 3, 2, 1, 0}
%L specg = hxs(ysg)
%L specF = mtintegralspec(specf, #ysf,  0)
%L specG = mtintegralspec(specf, #ysf, -3)
%L specI = mtintegralspec(specg, #ysg,  0)
%L pwsf  = PwSpec.from(specf)
%L pwsg  = PwSpec.from(specg)
%L pwsF  = PwSpec.from(specF)
%L pwsG  = PwSpec.from(specG)
%L pwsI  = PwSpec.from(specI)
%L pf    = pwsf:topict():setbounds(v(0,-2), v(#ysf,2)):pgat("pgatc")
%L pg    = pwsg:topict():setbounds(v(0,-2), v(#ysg,3)):pgat("pgatc")
%L pF    = pwsF:topict():setbounds(v(0,-0), v(#ysf,4)):pgat("pgatc")
%L pG    = pwsG:topict():setbounds(v(0,-3), v(#ysf,1)):pgat("pgatc")
%L pI    = pwsI:topict():setbounds(v(0,0),  v(#ysg,6)):pgat("pgatc")
%L pf:sa("Fig f"):output()
%L pg:sa("Fig g"):output()
%L pF:sa("Fig F"):output()
%L pG:sa("Fig G"):output()
%L pI:sa("Fig I"):output()
%L
%L PictList{}:setbounds(v(0,-4),v(13,4)):pgat("pgatc"):sa("respgrid"):output()
%L
%L mtintegralspec2 = function (x0, y0, Dys, dot0, dot1)
%L     local mkxy = function (x,y) return format("(%d,%d)", x, y) end
%L     local xys = { mkxy(x0,y0) .. (dot0 or "") }
%L     local x,y = x0,y0
%L     for i,Dy in ipairs(Dys) do
%L       x = x + 1
%L       y = y + Dy
%L       table.insert(xys, mkxy(x,y))
%L     end
%L     xys[#xys] = xys[#xys] .. (dot1 or "")
%L     return table.concat(xys, "--")
%L   end
%L
%L -- = mtintegralspec2(10, 20, {1, 2, -3, -3}, "a", "b")
%L ysf   = {1, 2, 1, 0, -1, -2, -1, 0, 1, 2, 1, 0}
%L ysf_  = {1, 2, 1, 0, -1, -2, -1}
%L ysg   = {0, 1, 2, 3, -2, -1,  0, -1, -2, 3, 2, 1, 0}
%L ysg_  =                         {-1, -2, 3, 2, 1}
%L specH = mtintegralspec2(0, -4, ysf_, "", "o\n") ..
%L         mtintegralspec2(7,  1, ysg_, "o", "")
%L specM = mtintegralspec2(0, -4, ysf_, "", "o\n") ..
%L         mtintegralspec2(7,  2, ysg_, "o", "")
%L -- = specH
%L -- = specM
%L pwsH  = PwSpec.from(specH)
%L pwsM  = PwSpec.from(specM)
%L pH    = pwsH:topict():setbounds(v(0,-4), v(12,4)):pgat("pgatc")
%L pM    = pwsM:topict():setbounds(v(0,-4),  v(12,5)):pgat("pgatc")
%L pH:sa("Fig H"):output()
%L pM:sa("Fig M"):output()
\pu

\unitlength=9pt

Sejam:

\msk

$\begin{array}{lcc}
  f(x) = \ga{Fig f} \; , \\
  g(x) = \ga{Fig g} \; .\\
  \end{array}
$

\msk

Faça os gráficos destas funções:

\ssk

a) $\D F(x) = \Intt{0}{x}{f(t)}$

\ssk

b) $\D G(x) = \Intt{3}{x}{g(t)}$



\GenericWarning{Success:}{Success!!!}  % Used by `M-x cv'

\end{document}



% Local Variables:
% coding: utf-8-unix
% ee-tla: "c2ca"
% ee-tla: "c2m231carro"
% End:
