% (find-LATEX "2023-1-C2-P1.tex")
% (defun c () (interactive) (find-LATEXsh "lualatex -record 2023-1-C2-P1.tex" :end))
% (defun C () (interactive) (find-LATEXsh "lualatex 2023-1-C2-P1.tex" "Success!!!"))
% (defun D () (interactive) (find-pdf-page      "~/LATEX/2023-1-C2-P1.pdf"))
% (defun d () (interactive) (find-pdftools-page "~/LATEX/2023-1-C2-P1.pdf"))
% (defun e () (interactive) (find-LATEX "2023-1-C2-P1.tex"))
% (defun o () (interactive) (find-LATEX "2022-2-C2-P1.tex"))
% (defun u () (interactive) (find-latex-upload-links "2023-1-C2-P1"))
% (defun v () (interactive) (find-2a '(e) '(d)))
% (defun d0 () (interactive) (find-ebuffer "2023-1-C2-P1.pdf"))
% (defun cv () (interactive) (C) (ee-kill-this-buffer) (v) (g))
%          (code-eec-LATEX "2023-1-C2-P1")
% (find-pdf-page   "~/LATEX/2023-1-C2-P1.pdf")
% (find-sh0 "cp -v  ~/LATEX/2023-1-C2-P1.pdf /tmp/")
% (find-sh0 "cp -v  ~/LATEX/2023-1-C2-P1.pdf /tmp/pen/")
%     (find-xournalpp "/tmp/2023-1-C2-P1.pdf")
%   file:///home/edrx/LATEX/2023-1-C2-P1.pdf
%               file:///tmp/2023-1-C2-P1.pdf
%           file:///tmp/pen/2023-1-C2-P1.pdf
%  http://anggtwu.net/LATEX/2023-1-C2-P1.pdf
% (find-LATEX "2019.mk")
% (find-sh0 "cd ~/LUA/; cp -v Pict2e1.lua Pict2e1-1.lua Piecewise1.lua ~/LATEX/")
% (find-sh0 "cd ~/LUA/; cp -v Pict2e1.lua Pict2e1-1.lua Pict3D1.lua ~/LATEX/")
% (find-sh0 "cd ~/LUA/; cp -v C2Subst1.lua C2Formulas1.lua ~/LATEX/")
% (find-sh0 "cd ~/LUA/; cp -v Gram2.lua Tree1.lua Caepro5.lua ~/LATEX/")
% (find-MM-aula-links "2023-1-C2-P1" "C2" "c2m231p1" "c2p1")

% «.defs»		(to "defs")
% «.defs-caepro»	(to "defs-caepro")
% «.defs-pict2e»	(to "defs-pict2e")
% «.defs-T-and-B»	(to "defs-T-and-B")
% «.title»		(to "title")
% «.questoes-123»	(to "questoes-123")
% «.questoes-123-dicas»	(to "questoes-123-dicas")
% «.questoes-45»	(to "questoes-45")
% «.questao-5-grids»	(to "questao-5-grids")
% «.questao-1-gab»	(to "questao-1-gab")
% «.questao-2-gab»	(to "questao-2-gab")
% «.questao-3-gab»	(to "questao-3-gab")
% «.questao-4-gab»	(to "questao-4-gab")
% «.questao-5-gab»	(to "questao-5-gab")
% «.links»		(to "links")
%
% «.djvuize»		(to "djvuize")



% <videos>
% Video (not yet):
% (find-ssr-links     "c2m231p1" "2023-1-C2-P1")
% (code-eevvideo      "c2m231p1" "2023-1-C2-P1")
% (code-eevlinksvideo "c2m231p1" "2023-1-C2-P1")
% (find-c2m231p1video "0:00")

\documentclass[oneside,12pt]{article}
\usepackage[colorlinks,citecolor=DarkRed,urlcolor=DarkRed]{hyperref} % (find-es "tex" "hyperref")
\usepackage{amsmath}
\usepackage{amsfonts}
\usepackage{amssymb}
\usepackage{pict2e}
\usepackage[x11names,svgnames]{xcolor} % (find-es "tex" "xcolor")
\usepackage{colorweb}                  % (find-es "tex" "colorweb")
%\usepackage{tikz}
%
% (find-dn6 "preamble6.lua" "preamble0")
%\usepackage{proof}   % For derivation trees ("%:" lines)
%\input diagxy        % For 2D diagrams ("%D" lines)
%\xyoption{curve}     % For the ".curve=" feature in 2D diagrams
%
\usepackage{edrx21}               % (find-LATEX "edrx21.sty")
\input edrxaccents.tex            % (find-LATEX "edrxaccents.tex")
\input edrx21chars.tex            % (find-LATEX "edrx21chars.tex")
\input edrxheadfoot.tex           % (find-LATEX "edrxheadfoot.tex")
\input edrxgac2.tex               % (find-LATEX "edrxgac2.tex")
%\usepackage{emaxima}              % (find-LATEX "emaxima.sty")
%
% (find-es "tex" "geometry")
\usepackage[a6paper, landscape,
            top=1.5cm, bottom=.25cm, left=1cm, right=1cm, includefoot
           ]{geometry}
%
\begin{document}

% «defs»  (to ".defs")
% (find-LATEX "edrx21defs.tex" "colors")
% (find-LATEX "edrx21.sty")

\def\drafturl{http://anggtwu.net/LATEX/2023-1-C2.pdf}
\def\drafturl{http://anggtwu.net/2023.1-C2.html}
\def\draftfooter{\tiny \href{\drafturl}{\jobname{}} \ColorBrown{\shorttoday{} \hours}}

% (find-LATEX "2023-1-C2-carro.tex" "defs-caepro")
% (find-LATEX "2023-1-C2-carro.tex" "defs-pict2e")

\catcode`\^^J=10
\directlua{dofile "dednat6load.lua"}  % (find-LATEX "dednat6load.lua")

% «defs-caepro»  (to ".defs-caepro")
%L dofile "Caepro5.lua"              -- (find-angg "LUA/Caepro5.lua" "LaTeX")
\def\Caurl   #1{\expr{Caurl("#1")}}
\def\Cahref#1#2{\href{\Caurl{#1}}{#2}}
\def\Ca      #1{\Cahref{#1}{#1}}

% «defs-pict2e»  (to ".defs-pict2e")
%L V = nil                           -- (find-angg "LUA/Pict2e1.lua" "MiniV")
%L dofile "Piecewise1.lua"           -- (find-LATEX "Piecewise1.lua")
%L Pict2e.__index.suffix = "%"
\def\pictgridstyle{\color{GrayPale}\linethickness{0.3pt}}
\def\pictaxesstyle{\linethickness{0.5pt}}
\def\pictnaxesstyle{\color{GrayPale}\linethickness{0.5pt}}
\celllower=2.5pt

\pu

\sa{[IP]}{\CFname{IP}{}}
\sa{[TFC2]}{\CFname{TFC2}{}}

% «defs-T-and-B»  (to ".defs-T-and-B")
\long\def\ColorOrange#1{{\color{orange!90!black}#1}}
\def\T(Total: #1 pts){{\bf(Total: #1)}}
\def\T(Total: #1 pts){{\bf(Total: #1 pts)}}
\def\T(Total: #1 pts){\ColorRed{\bf(Total: #1 pts)}}
\def\B       (#1 pts){\ColorOrange{\bf(#1 pts)}}




%  _____ _ _   _                               
% |_   _(_) |_| | ___   _ __   __ _  __ _  ___ 
%   | | | | __| |/ _ \ | '_ \ / _` |/ _` |/ _ \
%   | | | | |_| |  __/ | |_) | (_| | (_| |  __/
%   |_| |_|\__|_|\___| | .__/ \__,_|\__, |\___|
%                      |_|          |___/      
%
% «title»  (to ".title")
% (c2m231p1p 1 "title")
% (c2m231p1a   "title")

\thispagestyle{empty}

\begin{center}

\vspace*{1.2cm}

{\bf \Large Cálculo 2 - 2023.1}

\bsk

P1 (Primeira prova)

\bsk

Eduardo Ochs - RCN/PURO/UFF

\url{http://anggtwu.net/2023.1-C2.html}

\end{center}

\newpage

% «questoes-123»  (to ".questoes-123")
% (c2m231p1p 2 "questoes-123")
% (c2m231p1a   "questoes-123")

% (c2m222p1p 1 "questao-1")
% (c2m222p1a   "questao-1")
% (c2m222p1p 2 "subst-trig")
% (c2m222p1a   "subst-trig")
% (c2m222mva "title")
% (c2m222mva "title" "Aula 10: Mudança de variáveis")
% (c2m222tudop 49)
% (c2m222striga "title")
% (c2m222striga "title" "Aulas 11 e 12: substituição trigonométrica")
% (find-es "maxima" "subst-trig-questions")
% (find-es "maxima" "subst-trig-questions" "F(2,1)")

%\vspace*{-0.4cm}


\scalebox{0.6}{\def\colwidth{9cm}\firstcol{

{\bf Questão 1}

\T(Total: 2.5 pts)

\msk

Calcule:

$$\ints{s^3 \sqrt{1-s^2}}\;.$$

\bsk


{\bf Questão 2}

\T(Total: 2.5 pts)

\msk

Calcule a integral abaixo fazendo pelo menos duas mudanças de variável
e teste o seu resultado:

$$\intx{\frac{\cos(2+\sqrt x)}{2 \sqrt x}}.$$

\bsk

{\bf Questão 3}

\T(Total: 2.5 pts)

\msk

Calcule e teste o seu resultado:

$$\intx{\frac{2x+3}{(x-4)(x+5)}}\;.$$

\bsk

% (setq eepitch-preprocess-regexp "^")
% (setq eepitch-preprocess-regexp "^%T ?")
% (find-es "maxima" "subst-trig-questions")
%
%T  (eepitch-maxima)
%T  (eepitch-kill)
%T  (eepitch-maxima)
%T
%T f(a,b) := x^a * sqrt(1 - x^2)^b;
%T F(a,b) := integrate(f(a,b), x);
%T f(3,1);
%T F(3,1);
%T
%T F : sin(2+sqrt(x));
%T diff(F, x);
%T
%T f : (2*x + 3) / ((x-4) * (x+5));
%T F : integrate(f, x);




}\anothercol{

% «questoes-123-dicas»  (to ".questoes-123-dicas")
% (c2m231p1p 2 "questoes-123-dicas")
% (c2m231p1a   "questoes-123-dicas")
{}

{\bf Dicas:}

\ssk

1) Nestas questões o que vai contar mais pontos é você organizar as
contas de modo que cada passo seja fácil de entender, de verificar, e
de justificar -- ``chegar no resultado certo'' vai valer relativamente
pouco.

\ssk

2) Recomendo que vocês usem o método das ``caixinhas de anotações''
nas mudanças de variável... numa caixinha de anotações a primeira
linha diz a relação entre a variável nova e a antiga, todas as outras
linhas são consequências da primeira, e dentro da caixinha de
anotações você pode usar as gambiarras com diferenciais, como isto
aqui: $dx = 42\,du$...

\ssk

3) ...por exemplo:
%
$$\bmat{
  s = \sen θ \\
  \sqrt{1-s^2} = \cos θ \\
  \frac{ds}{dθ} = \cos θ \\
  ds = \cos θ \, dθ \\
  θ = \arcsen s \\
  }
$$

}}



\newpage

%  _  _            ____  
% | || |     ___  | ___| 
% | || |_   / _ \ |___ \ 
% |__   _| |  __/  ___) |
%    |_|    \___| |____/ 
%                        
% «questoes-45»  (to ".questoes-45")
% (c2m231p1p 3 "questoes-45")
% (c2m231p1a   "questoes-45")

\scalebox{0.6}{\def\colwidth{9cm}\firstcol{

{\bf Questão 4}

\T(Total: 1.5 pts)

\msk

No curso nós definimos que {\sl pra nós} a ``fórmula'' do TFC2 seria
esta aqui:
%
$$\ga{[TFC2]}
  \;=\;
  \left( \Intx{a}{b}{F'(x)} \;=\; \difx{a}{b}{F(x)} \right)
$$

Mostre que quando $a=1$, $b=3$ e
%
$$F(x) =
  \begin{cases}
     x & \text{quando $x<2$}, \\
     -x & \text{quando $x≥2$} \\
  \end{cases}
$$

a fórmula $\ga{[TFC2]}$ é falsa.

\msk

Dicas: o melhor modo de fazer isto é representando graficamente $F(x)$
e $F'(x)$ e calculando certas coisas a partir dos gráficos. Considere
que o leitor sabe calcular áreas de retângulos, triângulos e trapézios
no olhômetro quando as coordenadas deles são números simples, mas
complemente os seus gráficos com um pouquinho de português quando nem
tudo for óbvio só a partir dos gráficos.



}\anothercol{

{\bf Questão 5}

\T(Total: 1.0 pts)

\msk

Seja $f(t)$ a função no topo da página seguinte.

Seja
%
$$F(x) \;=\; \Intt{5}{x}{f(t)}.$$

Desenhe o gráfico de $F(x)$ em algum dos grids vazios da próxima
página. Indique claramente qual é a versão final e quais desenhos são
rascunhos.

}}

\newpage

% «questao-5-grids»  (to ".questao-5-grids")
% (c2m231p1p 4 "questao-5-grids")
% (c2m231p1a   "questao-5-grids")
% (c2m222p1p 4 "questao-5-grids")
% (c2m222p1a   "questao-5-grids")

%L -- (find-angg "LUA/Pict2e1-1.lua" "FromYs")
%L fryF = FromYs.fromys({0,-1,1,-2,2,-3,3,-3,2,-2,1,-1,0}):getYs(1)
%L fryF = FromYs.fromys({0,-1,-3,3,1,0,1,2,1,0,-1,-2,-1,0}):getYs(0)
%L fryF:getypict():pgat("pgatc"):sa("fig f"):output()
%L fryF:getYpict():pgat("pgatc"):sa("fig F"):output()
%L fryF:getYgrid(-4,4):
%L                pgat("pgatc"):sa("grid F"):output()
\pu

\unitlength=8pt

$\begin{array}{ll}
 \ga{fig f}  \phantom{mm} & \ga{fig f}  \\ \\
 \ga{grid F} & \ga{grid F} \\ \\
 \ga{grid F} & \ga{grid F} \\
 \end{array}
$



\newpage

% «questao-1-gab»  (to ".questao-1-gab")
% (c2m231p1p 5 "questao-1-gab")
% (c2m231p1a   "questao-1-gab")
% (c2m222p1p 5 "questao-1-gab")
% (c2m222p1a   "questao-1-gab")

% (setq eepitch-preprocess-regexp "^")
% (setq eepitch-preprocess-regexp "^%T ")
%
%T  (eepitch-maxima)
%T  (eepitch-kill)
%T  (eepitch-maxima)
%T s : sqrt(1-4*x^2);
%T f : x^3 * s;
%T F : integrate(f, x);
%T G : (1/16) * (s^5/5 - s^3/3);
%T g : diff(G, x);
%T expand(rat(g*s));
%T expand(rat(f*s));

{\bf Questão 1: gabarito}

$$\scalebox{0.6}{$
  \begin{array}{rcl}
  \ints{s^3 \sqrt{1-s^2}}
   %&=& \intu{\frac18 u^3 \sqrt{1-u^2}·\frac12} \\
   %&=& \frac1{16}\intu{u^3 \sqrt{1-u^2}} \\
    &=& \intth{(\senθ)^3 (\cosθ)(\cosθ)} \\
    &=& \intth{(\cosθ)^2 (\senθ)^3} \\
    &=& \intth{(\cosθ)^2 (\senθ)^2 (\senθ)} \\
    &=& \intc{c^2 (1-c^2)(-1)} \\
    &=& \intc{c^2 (c^2-1)} \\
    &=& \intc{c^4 - c^2} \\
    &=& \frac{c^5}{5} - \frac{c^3}{3} \\
    &=& \frac{(\cosθ)^5}{5} - \frac{(\cosθ)^3}{3} \\
    &=& \frac{\sqrt{1-s^2}^5}{5} - \frac{\sqrt{1-s^2}^3}{3} \\
    \\
    \frac{d}{ds}(\frac{\sqrt{1-s^2}^5}{5} - \frac{\sqrt{1-s^2}^3}{3})
    &=& \frac15 \frac{d}{ds} \sqrt{1-s^2}^5 - \frac13 \frac{d}{ds} \sqrt{1-s^2}^3 \\
    &=& \frac15 \frac{d}{ds} (1-s^2)^{5/2} - \frac13 \frac{d}{ds} (1-s^2)^{3/2} \\
    &=& \frac15 \frac52 (1-s^2)^{3/2} \frac{d}{ds}(1-s^2)
      - \frac13 \frac32 (1-s^2)^{1/2} \frac{d}{ds}(1-s^2) \\
    &=& \frac15 \frac52 (1-s^2)^{3/2} (-2s)
      - \frac13 \frac32 (1-s^2)^{1/2} (-2s) \\
    &=& \frac15 \frac52 (-2)s (1-s^2)^{3/2}
      - \frac13 \frac32 (-2)s (1-s^2)^{1/2} \\
    &=& - s (1-s^2)^{3/2}
        + s (1-s^2)^{1/2} \\
    &=& - s (1-s^2)^{2/2} (1-s^2)^{1/2}
        + s (1-s^2)^{1/2} \\
    &=& - s (1-s^2) (1-s^2)^{1/2}
        + s (1-s^2)^{1/2} \\
    &=& (- s (1-s^2) + s) (1-s^2)^{1/2} \\
    &=& (- s + s^3 + s) (1-s^2)^{1/2} \\
    &=& s^3  \sqrt{1-s^2} \\
  \end{array}
  \hspace*{-3cm}
  \begin{array}{l}
    % \bsm{u = 2x \\
    %      u^2 = 4x^2 \\
    %      x = u/2 \\
    %      x^3 = u^3/8 \\
    %      du = 2dx \\
    %      dx = \frac12 du \\
    %     } \\
    % \\[-7pt]
    \bsm{s = \senθ \\
         s^2 = (\senθ)^2 \\
         1-s^2 = (\cosθ)^2 \\
         \sqrt{1-s^2} = \cosθ \\
         \frac{ds}{dθ} = \cosθ \\
         ds = \cosθ\,dθ \\
        } \\
    \\[-7pt]
    \bsm{c = \cosθ \\
         \frac{dc}{dθ} = -\senθ \\
         dc = -\senθ\,dθ \\
         (-1)dc = \senθ\,dθ \\
         (\senθ)^2 = 1-c^2 \\
        } \\
    \\
    \vspace*{6cm}
  \end{array}
  $}
$$



\newpage

% «questao-2-gab»  (to ".questao-2-gab")
% (c2m231p1p 6 "questao-2-gab")
% (c2m231p1a   "questao-2-gab")
% (c2m222p1p 6 "questao-2-gab")
% (c2m222p1a   "questao-2-gab")

{\bf Questão 2: gabarito}

$$\begin{array}{rcl}
  \intx{\frac{\cos(2+\sqrt x)}{2 \sqrt x}}
    &=& \intu{\cos(2+u)} \\
    &=& \intv{\cos v} \\
    &=& \sen v \\
    &=& \sen(2+u) \\
    &=& \sen(2+\sqrt{x}) \\
  \\[-5pt]
  \ddx \sen(2+\sqrt{x})
    &=& \cos(2+\sqrt{x}) \, \ddx(2+\sqrt{x}) \\
    &=& \cos(2+\sqrt{x}) \, \ddx x^{1/2} \\
    &=& \cos(2+\sqrt{x}) \, \frac{1}{2} x^{-1/2} \\
    &=& \cos(2+\sqrt{x}) \, \frac{1}{2\sqrt{x}} \\
    &=& \frac{\cos(2+\sqrt{x})}{2\sqrt{x}} \\
  \end{array}
  %\qquad
  \begin{array}{c}
    \subst{u \;=\; \sqrt{x} \;=\; x^{1/2} \\
           \frac{du}{dx} \;=\; \frac12 x^{-1/2} \;=\; \frac{1}{2\sqrt{x}} \\
           du \;=\; \frac{1}{2\sqrt{x}} dx \\
           u^2 \;=\; x \\
           x \;=\; u^2 \\
          } \\
    \\[-5pt]
    \subst{v \;=\; 2+u \\
           dv \;=\; du \\
           v-2 \;=\; u \\
           u \;=\; v-2 \\
          } \\
    \\
    \vspace*{1.5cm}
  \end{array}
$$

\newpage

% «questao-3-gab»  (to ".questao-3-gab")
% (c2m231p1p 7 "questao-3-gab")
% (c2m231p1a   "questao-3-gab")
% (c2m222p1p 8 "questao-4-gab")
% (c2m222p1a   "questao-4-gab")

{\bf Questão 3: gabarito}

$$\scalebox{0.6}{$
  \begin{array}{rcl}
  \frac{2x+3}{(x-4)(x+5)} &=& \frac{A}{x-4} + \frac{B}{x+5} \\
                          &=& \frac{A(x+5)}{(x-4)(x+5)} + \frac{B(x-4)}{(x-4)(x+5)} \\
                          &=& \frac{A(x+5)+B(x-4)}{(x-4)(x+5)} \\
                          &=& \frac{Ax+5A+Bx-4B}{(x-4)(x+5)} \\
                          &=& \frac{(A+B)x+(5A-4B)}{(x-4)(x+5)} \\
                          \\[-5pt]
                     2x+3 &=& (A+B)x+(5A-4B) \\
                      A+B &=& 2 \\
                    5A-4B &=& 3 \\
                        A &=& 11/9 \\
                        B &=& 7/9 \\
                          \\[-5pt]
        \frac{2x+3}{(x-4)(x+5)}  &=& \frac{11/9}{x-4} + \frac{7/9}{x+5} \\
  \intx{\frac{2x+3}{(x-4)(x+5)}} &=& \intx{\frac{11/9}{x-4} + \frac{7/9}{x+5}} \\
                                 &=& \frac{11}{9}\intx{\frac{1}{x-4}}
                                   + \frac{7}{9}\intx{\frac{1}{x+5}} \\
                                 &=& \frac{11}{9} \ln |x-4|
                                   + \frac{7}{9}  \ln |x+5| \\
                          \\[-5pt]
  \ddx(\frac{11}{9} \ln |x-4| + \frac{7}{9} \ln |x+5|)
    &=& \frac{11}{9} \frac{1}{x-4} + \frac{7}{9} \frac{1}{x+5} \\
    &=& \frac{\frac{11}{9}(x+5) + \frac{7}{9}(x-4)}{(x-4)(x+5)} \\
    &=& \frac{(\frac{11}{9}+ \frac{7}{9})x + (\frac{55}{9} - \frac{28}{9})}{(x-4)(x+5)} \\
    &=& \frac{2x+3}{(x-4)(x+5)} \\
  \end{array}
  $}
$$


% (setq eepitch-preprocess-regexp "^")
% (setq eepitch-preprocess-regexp "^%T ?")
%
%T  (eepitch-maxima)
%T  (eepitch-kill)
%T  (eepitch-maxima)
%T linsolve ([A+B=4, 3*A-2*B=5], [A, B]);
%T linsolve ([A+B=2, 5*A-4*B=3], [A, B]);
%T
%T f : (4*x + 5) / ((x-2)*(x+3));
%T partfrac(f, x);
%T F : integrate(f, x);

\newpage

% «questao-4-gab»  (to ".questao-4-gab")
% (c2m231p1p 8 "questao-4-gab")
% (c2m231p1a   "questao-4-gab")
% (c2m222p1p 7 "questao-3-gab")
% (c2m222p1a   "questao-3-gab")
% (c2m221atisp 21 "1-then-2")
% (c2m221atisa    "1-then-2")

{\bf Questão 4: gabarito}

%L Pict2e.bounds = PictBounds.new(v(0,-4), v(4,4))
%L spec = "(0,0)--(2,2)c (2,-2)o--(4,-4)"
%L pws = PwSpec.from(spec)
%L pws:topict():prethickness("1pt"):pgat("pgatc"):sa("F(x)"):output()
%L
%L Pict2e.bounds = PictBounds.new(v(0,-4), v(4,4))
%L spec = "(0,1)--(2,1)o (2,-1)o--(4,-1)"
%L pws = PwSpec.from(spec)
%L pws:topict():prethickness("1pt"):pgat("pgatc"):sa("F'(x)"):output()
%L
%L spec = "(0,1)--(2,1)o (2,-1)c--(4,-1)"
%L pwsa = PwSpec.from(spec)
%L pf = PictList{
%L   pwsa:topwfunction():areaify(1, 3):Color("Orange"),
%L   pws:topict()
%L }
%L pf:pgat("pgatc"):sa("int F'(x)"):output()

\pu

\msk

\unitlength=5pt

$$F(x) = \ga{F(x)}
 \quad
 F'(x) = \ga{F'(x)}
 \quad
 \textstyle \Intx{1}{3}{F'(x)} = \ga{int F'(x)} = 0
$$

\def\und#1#2{\underbrace{#1}_{#2}}

$$\und{
  \und{\Intx{1}{3}{F'(x)}}{0} \;=\;
  \und{\und{\und{\difx{1}{3}{F(x)}}{F(3)-F(1)}}{(-3)-1}}{-4}
  }{\False}
$$

% (c2m221vsbp 6 "questao-1-gab")
% (c2m221vsba   "questao-1-gab")

\newpage

% «questao-5-gab»  (to ".questao-5-gab")
% (c2m231p1p 9 "questao-5-gab")
% (c2m231p1a   "questao-5-gab")
% (c2m222p1p 9 "questao-5-gab")
% (c2m222p1a   "questao-5-gab")

{\bf Questão 5: gabarito}

\unitlength=10pt

$$\begin{array}{r}
 f(x) \;=\; \ga{fig f}  \\ \\
 F(x) \;=\; \Intt{5}{x}{f(t)}
      \;=\; \ga{fig F}  \\
 \end{array}
$$



% «links»  (to ".links")

\GenericWarning{Success:}{Success!!!}  % Used by `M-x cv'

\end{document}

% Local Variables:
% coding: utf-8-unix
% ee-tla: "c2p1"
% ee-tla: "c2m231p1"
% End:
