% (find-LATEX "2023-1-C2-P2.tex")
% (defun c () (interactive) (find-LATEXsh "lualatex -record 2023-1-C2-P2.tex" :end))
% (defun C () (interactive) (find-LATEXsh "lualatex 2023-1-C2-P2.tex" "Success!!!"))
% (defun D () (interactive) (find-pdf-page      "~/LATEX/2023-1-C2-P2.pdf"))
% (defun d () (interactive) (find-pdftools-page "~/LATEX/2023-1-C2-P2.pdf"))
% (defun e () (interactive) (find-LATEX "2023-1-C2-P2.tex"))
% (defun o () (interactive) (find-LATEX "2022-2-C2-P2.tex"))
% (defun u () (interactive) (find-latex-upload-links "2023-1-C2-P2"))
% (defun v () (interactive) (find-2a '(e) '(d)))
% (defun d0 () (interactive) (find-ebuffer "2023-1-C2-P2.pdf"))
% (defun cv () (interactive) (C) (ee-kill-this-buffer) (v) (g))
%          (code-eec-LATEX "2023-1-C2-P2")
% (find-pdf-page   "~/LATEX/2023-1-C2-P2.pdf")
% (find-sh0 "cp -v  ~/LATEX/2023-1-C2-P2.pdf /tmp/")
% (find-sh0 "cp -v  ~/LATEX/2023-1-C2-P2.pdf /tmp/pen/")
%     (find-xournalpp "/tmp/2023-1-C2-P2.pdf")
%   file:///home/edrx/LATEX/2023-1-C2-P2.pdf
%               file:///tmp/2023-1-C2-P2.pdf
%           file:///tmp/pen/2023-1-C2-P2.pdf
%  http://anggtwu.net/LATEX/2023-1-C2-P2.pdf
% (find-LATEX "2019.mk")
% (find-Deps1-links "Caepro5 Piecewise1")
% (find-Deps1-cps   "Caepro5 Piecewise1")
% (find-Deps1-anggs "Caepro5 Piecewise1")
% (find-MM-aula-links "2023-1-C2-P2" "C2" "c2m231p2" "c2p2")

% «.defs»		(to "defs")
% «.defs-T-and-B»	(to "defs-T-and-B")
% «.defs-caepro»	(to "defs-caepro")
% «.defs-pict2e»	(to "defs-pict2e")
% «.title»		(to "title")
% «.questao-1»		(to "questao-1")
% «.questao-2»		(to "questao-2")
% «.questao-3»		(to "questao-3")
% «.questao-4»		(to "questao-4")
% «.anexo-L»		(to "anexo-L")
% «.anexo-R»		(to "anexo-R")
% «.anexo»		(to "anexo")
% «.questao-1-gab»	(to "questao-1-gab")
% «.questao-2-gab»	(to "questao-2-gab")
% «.links»		(to "links")
%
% «.djvuize»		(to "djvuize")



% <videos>
% Video (not yet):
% (find-ssr-links     "c2m231p2" "2023-1-C2-P2")
% (code-eevvideo      "c2m231p2" "2023-1-C2-P2")
% (code-eevlinksvideo "c2m231p2" "2023-1-C2-P2")
% (find-c2m231p2video "0:00")

\documentclass[oneside,12pt]{article}
\usepackage[colorlinks,citecolor=DarkRed,urlcolor=DarkRed]{hyperref} % (find-es "tex" "hyperref")
\usepackage{amsmath}
\usepackage{amsfonts}
\usepackage{amssymb}
\usepackage{pict2e}
\usepackage[x11names,svgnames]{xcolor} % (find-es "tex" "xcolor")
\usepackage{colorweb}                  % (find-es "tex" "colorweb")
%\usepackage{tikz}
%
% (find-dn6 "preamble6.lua" "preamble0")
%\usepackage{proof}   % For derivation trees ("%:" lines)
%\input diagxy        % For 2D diagrams ("%D" lines)
%\xyoption{curve}     % For the ".curve=" feature in 2D diagrams
%
\usepackage{edrx21}               % (find-LATEX "edrx21.sty")
\input edrxaccents.tex            % (find-LATEX "edrxaccents.tex")
\input edrx21chars.tex            % (find-LATEX "edrx21chars.tex")
\input edrxheadfoot.tex           % (find-LATEX "edrxheadfoot.tex")
\input edrxgac2.tex               % (find-LATEX "edrxgac2.tex")
%\usepackage{emaxima}              % (find-LATEX "emaxima.sty")
%
% (find-es "tex" "geometry")
\usepackage[a6paper, landscape,
            top=1.5cm, bottom=.25cm, left=1cm, right=1cm, includefoot
           ]{geometry}
%
\begin{document}

% «defs»  (to ".defs")
% (find-LATEX "edrx21defs.tex" "colors")
% (find-LATEX "edrx21.sty")

\def\drafturl{http://anggtwu.net/LATEX/2023-1-C2.pdf}
\def\drafturl{http://anggtwu.net/2023.1-C2.html}
\def\draftfooter{\tiny \href{\drafturl}{\jobname{}} \ColorBrown{\shorttoday{} \hours}}

\sa{[M]}{\CFname{M}{}}
\sa{[F]}{\CFname{F}{}}
\sa{[S]}{\CFname{S}{}}

% (find-LATEXgrep "grep --color=auto -nH --null -e mname 202{1,2}*.tex")
\def\sumiN#1{\sum_{i=1}^N #1 (b_i-a_i)}
\def\mname#1{\text{[#1]}}

% (find-LATEX "2023-1-C2-carro.tex" "defs-caepro")
% (find-LATEX "2023-1-C2-carro.tex" "defs-pict2e")

\catcode`\^^J=10
\directlua{dofile "dednat6load.lua"}  % (find-LATEX "dednat6load.lua")
%L dofile "Piecewise1.lua"           -- (find-LATEX "Piecewise1.lua")
%L -- dofile "QVis1.lua"             -- (find-LATEX "QVis1.lua")
%L -- dofile "Pict3D1.lua"           -- (find-LATEX "Pict3D1.lua")
%L -- dofile "C2Formulas1.lua"          -- (find-LATEX "C2Formulas1.lua")
%L -- dofile "Lazy5.lua"                -- (find-LATEX "Lazy5.lua")
%L -- dofile "2022-1-C2-P2.lua"         -- (find-LATEX "2022-1-C2-P2.lua")
%L Pict2e.__index.suffix = "%"
\pu
\def\pictgridstyle{\color{GrayPale}\linethickness{0.3pt}}
\def\pictaxesstyle{\linethickness{0.5pt}}
\def\pictnaxesstyle{\color{GrayPale}\linethickness{0.5pt}}
\celllower=2.5pt

% «defs-T-and-B»  (to ".defs-T-and-B")
\long\def\ColorOrange#1{{\color{orange!90!black}#1}}
\def\T(Total: #1 pts){{\bf(Total: #1)}}
\def\T(Total: #1 pts){{\bf(Total: #1 pts)}}
\def\T(Total: #1 pts){\ColorRed{\bf(Total: #1 pts)}}
\def\B       (#1 pts){\ColorOrange{\bf(#1 pts)}}

% «defs-caepro»  (to ".defs-caepro")
%L dofile "Caepro5.lua"              -- (find-angg "LUA/Caepro5.lua" "LaTeX")
\def\Caurl   #1{\expr{Caurl("#1")}}
\def\Cahref#1#2{\href{\Caurl{#1}}{#2}}
\def\Ca      #1{\Cahref{#1}{#1}}

% «defs-pict2e»  (to ".defs-pict2e")
%L -- V = nil                           -- (find-angg "LUA/Pict2e1.lua" "MiniV")
%L V = MiniV
%L v = V.fromab
%L dofile "Piecewise1.lua"           -- (find-LATEX "Piecewise1.lua")
%L Pict2e.__index.suffix = "%"
\def\pictgridstyle{\color{GrayPale}\linethickness{0.3pt}}
\def\pictaxesstyle{\linethickness{0.5pt}}
\def\pictnaxesstyle{\color{GrayPale}\linethickness{0.5pt}}
\celllower=2.5pt

\pu



%  _____ _ _   _                               
% |_   _(_) |_| | ___   _ __   __ _  __ _  ___ 
%   | | | | __| |/ _ \ | '_ \ / _` |/ _` |/ _ \
%   | | | | |_| |  __/ | |_) | (_| | (_| |  __/
%   |_| |_|\__|_|\___| | .__/ \__,_|\__, |\___|
%                      |_|          |___/      
%
% «title»  (to ".title")
% (c2m231p2p 1 "title")
% (c2m231p2a   "title")

\thispagestyle{empty}

\begin{center}

\vspace*{1.2cm}

{\bf \Large Cálculo 2 - 2023.1}

\bsk

P2 (Segunda prova)

\bsk

Eduardo Ochs - RCN/PURO/UFF

\url{http://anggtwu.net/2023.1-C2.html}

\end{center}

\newpage

%  _     _____ ____   _____     ______  
% / |   | ____|  _ \ / _ \ \   / / ___| 
% | |   |  _| | | | | | | \ \ / /\___ \ 
% | |_  | |___| |_| | |_| |\ V /  ___) |
% |_(_) |_____|____/ \___/  \_/  |____/ 
%                                       
% «questao-1»  (to ".questao-1")
% (c2m231p2p 2 "questao-1")
% (c2m231p2a   "questao-1")
% (c2m222p2p 2 "questao-1")
% (c2m222p2a   "questao-1")
% (find-es "maxima" "separable-2")
% (find-es "maxima" "2022-2-C2-P2-edovs")
% (find-es "maxima" "2023-1-C2-P2-edovs")

{\bf Questão 1}

\sa{(M)}{
  \left(\begin{array}{rcl}
             \D \dydx &=& \D \frac{g(x)}{h(y)} \\
             h(y)\,dy &=& g(x)\,dx \\
          \inty{h(y)} &=& \intx{g(x)} \\
           \mcc{\veq} & & \mcc{\veq} \\
        \mcc{H(y)+C1} & & \mcc{G(x)+C2} \\
             H(y)     &=& G(x)+C2-C1 \\
                      &=& G(x)+C3 \\
         H^{-1}(H(y)) &=& H^{-1}(G(x)+C3) \\
           \mcc{\veq} & & \\
              \mcc{y} & & \\
   \end{array}
   \right)
  }
\sa{(F)}{
  \left(\begin{array}{rcl}
             \D \dydx &=& \D \frac{g(x)}{h(y)} \\
         H^{-1}(H(y)) &=& H^{-1}(G(x)+C3) \\
           \mcc{\veq} & & \\
              \mcc{y} & & \\
   \end{array}
   \right)
  }


\scalebox{0.5}{\def\colwidth{11cm}\firstcol{

\vspace*{-0.4cm}

\T(Total: 4.0 pts)

Lembre que no curso eu mostrei que o meu modo preferido de escrever o
``método'' para resolver EDOs com variáveis separáveis --- ``EDOVSs''
--- é a ``demonstração'' \ga{[M]} abaixo... eu pus o termo
``demonstração'' entre aspas porque alguns dos passos da \ga{[M]} são
gambiarras nas quais a gente não pode confiar totalmente, e aí a gente
precisa sempre testar as nossas soluções. O \ga{[F]} abaixo --- a
``fórmula'' --- é uma versão resumida do \ga{[M]}.
%
$$\begin{array}{rcl}
  \ga{[M]} &=& \ga{(M)} \\\\[-5pt]
  \ga{[F]} &=& \ga{(F)} \\
  \end{array}
$$

\vspace*{-5cm}

}\anothercol{

{}

Seja $(*)$ esta EDOVS:
%
$$\frac{dy}{dx} \;=\; - \frac{1}{2y}
$$

a) \B (2.0 pts) Encontre as duas soluções gerais da EDO $(*)$ -- a
solução ``positiva'' e a ``negativa'' -- e teste-as.

\msk

b) \B (1.0 pts) Encontre a solução particular que passa pelo ponto
$(3,2)$ e teste-a.

\msk

c) \B (1.0 pts) Encontre a solução particular que passa pelo ponto
$(4,-3)$ e teste-a.


\bsk

\standout{Muito importante:} em todas as questões desta prova exceto a
questão sobre somas de Riemann eu vou corrigir as respostas de vocês
como se eu fosse o ``colega menos seu amigo e sem paciência pra
adivinhar nada'' da Dica 7 e do slide sobre contextos... por exemplo,
se você escrever só ``$a=42$'' eu vou interpretar isso como ``aqui
essa pessoa tá dizendo que é óbvio que `$a=42$' é sempre verdade -- e
isso é falso!!!'', e aí babau. Ou seja, a parte em português das
questões de vocês vai ser MUUUUITO importante!

\msk

A prova tem um anexo que é um gabarito de uma prova antiga, e que tem
exemplos de uso de várias partículas em português como ``seja'',
``isto é'', ``temos'' e ``então''. Esse anexo não tem exemplos de
todas as partículas mais comuns -- por exemplo, faltam o ``queremos
que'', o ``vamos testar se'' e o ``lembre que'' -- mas acho que ele
deve ajudar bastante.




% (find-es "maxima" "2022-2-C2-P2")



}}

\newpage

%  ____      _____ ____   ___  _     ____ ____     
% |___ \    | ____|  _ \ / _ \| |   / ___/ ___|___ 
%   __) |   |  _| | | | | | | | |  | |  | |   / __|
%  / __/ _  | |___| |_| | |_| | |__| |__| |___\__ \
% |_____(_) |_____|____/ \___/|_____\____\____|___/
%                                                  
% «questao-2»  (to ".questao-2")
% (c2m231p2p 3 "questao-2")
% (c2m231p2a   "questao-2")
% (c2m222p2p 3 "questao-2")
% (c2m222p2a   "questao-2")
% «edolccs»  (to ".edolccs")
% (c2m222p2p 3 "edolccs")
% (c2m222p2a   "edolccs")
% (find-es "maxima" "2022-2-C2-P2-edolccs")


{\bf Questão 2}

\scalebox{0.7}{\def\colwidth{8cm}\firstcol{

\vspace*{-0.4cm}

\T(Total: 4.0 pts)

Lembre que nós vimos dois tipos de EDOs lineares com coeficientes
constantes --- ``EDOLCCs'' --- no curso: o primeiro tipo tinha
soluções básicas da forma $e^{ax}$ e $e^{bx}$, onde $a$ e $b$ são
reais, e o segundo tipo tinha ``soluções básicas complexas'' da forma
$e^{(a+ib)x}$ e $e^{(a-ib)x}$ e ``soluções básicas reais'' da forma
$e^{αx}\cos βx$ e $e^{αx}\sen βx$; as soluções básicas reais eram
combinações lineares das soluções básicas complexas e vice-versa.

\msk

Sejam $(**)$ e $({*}{*}{*})$ as EDOs abaixo:
%
$$\begin{array}{rcll}
  y'' +  y' - 20y &=& 0 & \qquad (**) \\
  y'' + 4y' + 29y &=& 0 & \qquad ({*}{*}{*}) \\
  \end{array}
$$

A EDO $(**)$ é do primeiro tipo e a EDO $({*}{*}{*})$ é do segundo tipo.

}\anothercol{

{}

a) \B (0.5 pts) Encontre as soluções básicas e a solução geral da EDO
$(**)$. Dê um nome para cada uma delas.

\msk

b) \B (1.5 pts) Encontre uma solução da EDO $(**)$ -- vou chamá-la de
$g(x)$ -- que obedece $g(0) = 4$ e $g'(0)=5$, e teste-a. Dica: você
vai ter que resolver um sistema pra descobrir a quantidade certa de
cada ``vetor'' na combinação linear!

\bsk

c) \B (0.5 pts) Diga quais são as ``soluções básicas complexas'' e as
``soluções básicas reais'' para a EDO $({*}{*}{*})$.

\msk

d) \B (1.5 pts) Escolha uma das suas ``soluções básicas reais'' do
item anterior e verifique que ela realmente é uma solução da EDO
$({*}{*}{*})$.



% (find-es "maxima" "2022-2-C2-P2")



}}

\newpage


%   ___                  _                _____ 
%  / _ \ _   _  ___  ___| |_ __ _  ___   |___ / 
% | | | | | | |/ _ \/ __| __/ _` |/ _ \    |_ \ 
% | |_| | |_| |  __/\__ \ || (_| | (_) |  ___) |
%  \__\_\\__,_|\___||___/\__\__,_|\___/  |____/ 
%                                               
% «questao-3»  (to ".questao-3")
% (c2m231p2p 4 "questao-3")
% (c2m231p2a   "questao-3")
% (c2m222p2p 4 "questao-3")
% (c2m222p2a   "questao-3")

%L Pict2e.bounds = PictBounds.new(v(0,0), v(7,6))
%L spec = "(0,2)--(1,2)--(2,1)o (2,2)c (2,4)o--(3,3)--(4,3)o--(5,3)o--(7,3) (4,2)c (5,5)c"
%L pws = PwSpec.from(spec)
%L pws:topict():prethickness("1pt"):pgat("pgatc"):sa("F(x)"):output()
\pu

\unitlength=10pt

{\bf Questão 3}


\scalebox{0.45}{\def\colwidth{10.5cm}\firstcol{

\vspace*{-0.25cm}

\T(Total: 1.0 pts)

Lembre que nós vimos estes tipos de Somas de Riemann,
%
$$\scalebox{0.95}{$
  \begin{array}{ccl}
  \mname{L}    &=& \sumiN {f(a_i)}                    \\[2pt]
  \mname{R}    &=& \sumiN {f(b_i)}                    \\[2pt]
  \mname{Trap} &=& \sumiN {\frac{f(a_i) + f(b_i)}{2}} \\[2pt]
  \mname{M}    &=& \sumiN {f(\frac{a_i+b_i}{2})}      \\[2pt]
  \mname{min}  &=& \sumiN {\min(f(a_i), f(b_i))}      \\[2pt]
  \mname{max}  &=& \sumiN {\max(f(a_i), f(b_i))}      \\[2pt]
  \mname{inf}  &=& \sumiN {\inf(f([a_i,b_i]))}        \\[2pt]
  \mname{sup}  &=& \sumiN {\sup(f([a_i,b_i]))}        \\
  \end{array}
  $}
$$

% e vimos que o $\mname{Trap}$ pode ser interpretado tanto como uma soma
% de trapézios como como uma soma de retângulos.

\msk

Seja $f(x)$ a função dos gráficos à direita.

Represente graficamente cada um dos somatórios abaixo.

\def\Sitem#1#2#3{#1) $\mname{#2}_{\{#3\}}$}

\msk

\begin{tabular}{lll}
\Sitem a {sup} {1,6} &
\Sitem b {sup} {1,3,6} &
\Sitem c {sup} {1,3,5,6} \\
\Sitem d {inf} {1,6} &
\Sitem e {inf} {1,3,6} &
\Sitem f {inf} {1,3,5,6} \\
\Sitem g {max} {1,5} &
\Sitem h {max} {1,3,5} &
\Sitem i {max} {1,3,4,5} \\
\Sitem j {min} {1,5} &
\Sitem k {min} {1,3,5} &
\Sitem l {min} {1,3,4,5} \\
\end{tabular}


% d) $\mname{Trap}_{\{1,3,5\}}$ usando retângulos

% e) $\mname{Trap}_{\{1,3,5\}}$ usando trapézios

\msk

Indique claramente qual desenho é a resposta final de cada item e
quais desenhos são rascunhos.

}\anothercol{

\vspace*{-2cm}

\def\Fx{\scalebox{1.1}{$\ga{F(x)}$}}

$\begin{matrix}
 \Fx & \Fx & \Fx & \Fx & \Fx \\ \\[-5pt]
 \Fx & \Fx & \Fx & \Fx & \Fx \\ \\[-5pt]
 \Fx & \Fx & \Fx & \Fx & \Fx \\ \\[-5pt]
 \Fx & \Fx & \Fx & \Fx & \Fx \\ \\[-5pt]
 \Fx & \Fx & \Fx & \Fx & \Fx \\ \\[-5pt]
 \Fx & \Fx & \Fx & \Fx & \Fx \\
 \end{matrix}
$

\vspace*{-2cm}


}}


\newpage

%  _  _      ____        _ _     _           
% | || |    / ___|  ___ | (_) __| | ___  ___ 
% | || |_   \___ \ / _ \| | |/ _` |/ _ \/ __|
% |__   _|   ___) | (_) | | | (_| | (_) \__ \
%    |_|(_) |____/ \___/|_|_|\__,_|\___/|___/
%                                            
% «questao-4»  (to ".questao-4")
% (c2m231p2p 5 "questao-4")
% (c2m231p2a   "questao-4")

{\bf Questão 4}


\scalebox{0.8}{\def\colwidth{11cm}\firstcol{

\vspace*{-0.25cm}

\T(Total: 2.0 pts)

\msk

Seja
%
$$A \;=\; \setofxyst{0≤x≤\pi, \; 0≤y≤1+\cos x}.
$$

Seja $B$ o sólido que obtemos rodando a região $A$ em torno do eixo
$x$ e seja $C$ o sólido que obtemos rodando a região $A$ em torno do
eixo $y$.

\msk

a) \B (0.2 pts) Faça um esboço da região $A$.

\ssk

b) \B (0.8 pts) Calcule o volume de $B$.

\ssk

c) \B (1.0 pts) Calcule o volume de $C$.

\bsk

\standout{Importante:} nos itens (b) e (c) você provavelmente vai
chegar em integrais difíceis de resolver. Você não precisa resolver
elas, basta chegar em respostas que sejam integrais definidas.



}\anothercol{
}}



\newpage

% «anexo-L»  (to ".anexo-L")
\def\anexoL{

A substituição é:
%
$$\ga{[S]} \;=\;
  \bmat{
    G(x) := x^4 + 5 \\
    H(y) := y^2 + 3 \\
    g(x) := 4x^3 \\
    h(y) := 2y \\
    H^{-1}(x) := \sqrt{x-3} \\
  }
$$

a) Seja:
%
$$\frac{dy}{dx} = \frac{4x^3}{2y} \qquad (*)$$

b)
%
 $\begin{array}[t]{lrcl}
  \text{Seja:}  & H^{-1}(x) &=& \sqrt{x-3}. \\
  \text{Temos:} & H^{-1}(H(y)) &=& \sqrt{H(y)-3} \\
                &              &=& \sqrt{(y^2+3)-3} \\
                &              &=& y. \\
  \end{array}
 $

\msk

c) $\begin{array}[t]{lrcl}
       & y &=& H^{-1}(G(x)+C_3) \\
          &&=& \sqrt{(G(x)+C_3)-3} \\
          &&=& \sqrt{((x^4+5)+C_3)-3} \\
          &&=& \sqrt{x^4+2+C_3} \\
    \text{Seja:} &
      f(x) &=& \sqrt{x^4+2+C_3}. \\
    \end{array}
   $

}


% «anexo-R»  (to ".anexo-R")

\def\anexoR{

d) $\begin{array}[t]{l}
    \text{Será que $f(x)$ obedece $(*)$?} \\
    \text{Temos }
    f'(x) = \frac{2x^3}{\sqrt{x^4 + 2 + C_3}},
    \text{ e com isso:}
    \\
    \\[-5pt]
    \left(
      f'(x) = \frac{4x^3}{2f(x)}
    \right)
    \bmat{
      f(x) = \sqrt{x^4+2+C_3} \\
      f'(x) = \frac{2x^3}{\sqrt{x^4 + 2 + C_3}} \\
    }
    \\
    = \;\;
    \left(
      \frac{2x^3}{\sqrt{x^4 + 2 + C_3}}
      = \frac{4x^3}{2\sqrt{x^4+2+C_3}}
    \right)
    \qquad \smile \\
    \end{array}
   $

\bsk

e) $\begin{array}[t]{lrcl}
    \text{Se}    & f(x_1) &=& y_1, \\
    \text{i.e.,} & f(1)   &=& 2,   \\
    \text{então} & f(1)   &=& \sqrt{1^4+2+C_3} \\
                         &&=& \sqrt{3+C_3} \\
                         &&=& 2 \\
                 & 2^2    &=& \sqrt{3+C_3}^2 \\
                 & 4      &=&       3+C_3    \\
                 & C_3    &=& 1 \\
                 & f(x)   &=& \sqrt{x^4+2+C_3} \\
                 &        &=& \sqrt{x^4+3} \\
    \text{Seja:} & f_1(x) &=& \sqrt{x^4+3}. \\
    \end{array}
   $

\bsk

f) $\begin{array}[t]{lrcl}
    \text{Será que} & f_1(x_1) &=& y_1, \\
    \text{i.e.,}    & f_1(1)   &=& 2?   \\
                & \sqrt{1^4+3} &=& \sqrt{4} \\
                              &&=& 2 \qquad \smile \\
    \end{array}
   $

}

% «anexo»  (to ".anexo")

\scalebox{0.6}{\def\colwidth{9cm}\firstcol{

\vspace*{-0.5cm}

{\bf Anexo: gabarito de uma}

{\bf questão da P2 de 2022.2}

\ssk

\anexoL

}\anothercol{

\anexoR

}}



\newpage

{\bf Mini-gabarito}


\scalebox{0.6}{\def\colwidth{12cm}\firstcol{

% «questao-1-gab»  (to ".questao-1-gab")
% (c2m231p2p 7 "questao-1-gab")
% (c2m231p2a   "questao-1-gab")
% (find-es "maxima" "2023-1-C2-P2-edovs")

1a)      $f_1(x) = \ph{ii} \sqrt{- x - C_3}$,

\ph{aai} $f_2(x) = - \sqrt{- x - C_3}$

1b) $f_3(x) = \ph{ii} \sqrt{7 - x}$ \ph{iii} (passa por $(x,y)=(3,2)$)

1c) $f_4(x) = - \sqrt{13 - x}$ \ph{i} (passa por $(x,y)=(4,-3)$)

\bsk


% «questao-2-gab»  (to ".questao-2-gab")
% (c2m231p2p 7 "questao-2-gab")
% (c2m231p2a   "questao-2-gab")
% (find-es "maxima" "2023-1-C2-P2-edolccs")

2) $y''+y'-20y = (D+5)(D-4)y$,

\ph{aa} $y''+4y'+29y = (D-(-2+5i))(D-(-2-5i))y$,

2a) $f_1(x) = e^{4x}$, $f_2(x) = e^{-5x}$

2b) $g(x) = \frac{25}{9} e^{4x} + \frac{11}{9} e^{-5x}$

2c)  $f_1(x) = e^{(-2+5i)x}$, $f_2(x) = e^{(-2-5i)x}$,

\ph{aai} $f_3(x) = e^{-2x}\cos 5x$, $f_2(x) = e^{-2x}\sen 5x$

\bsk

\def\area{\textsf{área}}
\def\vol {\textsf{vol}}

4b) $\begin{array}[t]{rcl}
     r(x) &=& 1+\cos x \\
     \area(x) &=& π(1+\cos x)^2 \\
     \vol &=& \Intx{0}{π}{π(1+\cos x)^2} \\
     \end{array}$

4c) $\begin{array}[t]{rcl}
     y   &=& 1+\cos x \\
     y-1 &=& \cos x \\
     x   &=& \arccos(y-1) \\
     \area(y) &=& π(\arccos(y-1))^2 \\
     \vol     &=& \Inty{0}{2}{π(\arccos(y-1))^2} \\
     \end{array}$

}\anothercol{
}}




\newpage



%L Pict2e.bounds = PictBounds.new(v(0,0), v(7,6))
%L spec = "(0,1)--(1,1)--(2,4)--(3,5)--(4,4)o (4,3)c (4,1)o--(6,3)--(7,3)"
%L pws = PwSpec.from(spec)
%L pws:topict():prethickness("1pt"):pgat("pgatc"):sa("F(x)"):output()
\pu

% «links»  (to ".links")

\GenericWarning{Success:}{Success!!!}  % Used by `M-x cv'

\end{document}

%  ____  _             _         
% |  _ \(_)_   ___   _(_)_______ 
% | | | | \ \ / / | | | |_  / _ \
% | |_| | |\ V /| |_| | |/ /  __/
% |____// | \_/  \__,_|_/___\___|
%     |__/                       
%
% «djvuize»  (to ".djvuize")
% (find-LATEXgrep "grep --color -nH --null -e djvuize 2020-1*.tex")

 (eepitch-shell)
 (eepitch-kill)
 (eepitch-shell)
# (find-fline "~/2023.1-C2/")
# (find-fline "~/LATEX/2023-1-C2/")
# (find-fline "~/bin/djvuize")

cd /tmp/
for i in *.jpg; do echo f $(basename $i .jpg); done

f () { rm -v $1.pdf;  textcleaner -f 50 -o  5 $1.jpg $1.png; djvuize $1.pdf; xpdf $1.pdf }
f () { rm -v $1.pdf;  textcleaner -f 50 -o 10 $1.jpg $1.png; djvuize $1.pdf; xpdf $1.pdf }
f () { rm -v $1.pdf;  textcleaner -f 50 -o 20 $1.jpg $1.png; djvuize $1.pdf; xpdf $1.pdf }

f () { rm -fv $1.png $1.pdf; djvuize $1.pdf }
f () { rm -fv $1.png $1.pdf; djvuize WHITEBOARDOPTS="-m 1.0 -f 15" $1.pdf; xpdf $1.pdf }
f () { rm -fv $1.png $1.pdf; djvuize WHITEBOARDOPTS="-m 1.0 -f 30" $1.pdf; xpdf $1.pdf }
f () { rm -fv $1.png $1.pdf; djvuize WHITEBOARDOPTS="-m 1.0 -f 45" $1.pdf; xpdf $1.pdf }
f () { rm -fv $1.png $1.pdf; djvuize WHITEBOARDOPTS="-m 0.5" $1.pdf; xpdf $1.pdf }
f () { rm -fv $1.png $1.pdf; djvuize WHITEBOARDOPTS="-m 0.25" $1.pdf; xpdf $1.pdf }
f () { cp -fv $1.png $1.pdf       ~/2023.1-C2/
       cp -fv        $1.pdf ~/LATEX/2023-1-C2/
       cat <<%%%
% (find-latexscan-links "C2" "$1")
%%%
}

f 20201213_area_em_funcao_de_theta
f 20201213_area_em_funcao_de_x
f 20201213_area_fatias_pizza



%  __  __       _        
% |  \/  | __ _| | _____ 
% | |\/| |/ _` | |/ / _ \
% | |  | | (_| |   <  __/
% |_|  |_|\__,_|_|\_\___|
%                        
% <make>

 (eepitch-shell)
 (eepitch-kill)
 (eepitch-shell)
# (find-LATEXfile "2019planar-has-1.mk")
make -f 2019.mk STEM=2023-1-C2-P2 veryclean
make -f 2019.mk STEM=2023-1-C2-P2 pdf

% Local Variables:
% coding: utf-8-unix
% ee-tla: "c2p2"
% ee-tla: "c2m231p2"
% End:
