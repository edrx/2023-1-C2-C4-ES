% (find-LATEX "2023-1-C2-dicas-pra-P2.tex")
% (defun c () (interactive) (find-LATEXsh "lualatex -record 2023-1-C2-dicas-pra-P2.tex" :end))
% (defun C () (interactive) (find-LATEXsh "lualatex 2023-1-C2-dicas-pra-P2.tex" "Success!!!"))
% (defun D () (interactive) (find-pdf-page      "~/LATEX/2023-1-C2-dicas-pra-P2.pdf"))
% (defun d () (interactive) (find-pdftools-page "~/LATEX/2023-1-C2-dicas-pra-P2.pdf"))
% (defun e () (interactive) (find-LATEX "2023-1-C2-dicas-pra-P2.tex"))
% (defun o () (interactive) (find-LATEX "2023-1-C2-dicas-pra-P2.tex"))
% (defun u () (interactive) (find-latex-upload-links "2023-1-C2-dicas-pra-P2"))
% (defun v () (interactive) (find-2a '(e) '(d)))
% (defun d0 () (interactive) (find-ebuffer "2023-1-C2-dicas-pra-P2.pdf"))
% (defun cv () (interactive) (C) (ee-kill-this-buffer) (v) (g))
%          (code-eec-LATEX "2023-1-C2-dicas-pra-P2")
% (find-pdf-page   "~/LATEX/2023-1-C2-dicas-pra-P2.pdf")
% (find-sh0 "cp -v  ~/LATEX/2023-1-C2-dicas-pra-P2.pdf /tmp/")
% (find-sh0 "cp -v  ~/LATEX/2023-1-C2-dicas-pra-P2.pdf /tmp/pen/")
%     (find-xournalpp "/tmp/2023-1-C2-dicas-pra-P2.pdf")
%   file:///home/edrx/LATEX/2023-1-C2-dicas-pra-P2.pdf
%               file:///tmp/2023-1-C2-dicas-pra-P2.pdf
%           file:///tmp/pen/2023-1-C2-dicas-pra-P2.pdf
%  http://anggtwu.net/LATEX/2023-1-C2-dicas-pra-P2.pdf
% (find-LATEX "2019.mk")
% (find-Deps1-links "Caepro5 Piecewise1")
% (find-Deps1-cps   "Caepro5 Piecewise1")
% (find-Deps1-anggs "Caepro5 Piecewise1")
% (find-MM-aula-links "2023-1-C2-dicas-pra-P2" "C2" "c2m231dicasp2" "c2d2")

% «.defs»		(to "defs")
% «.defs-T-and-B»	(to "defs-T-and-B")
% «.defs-caepro»	(to "defs-caepro")
% «.defs-pict2e»	(to "defs-pict2e")
% «.title»		(to "title")
% «.links»		(to "links")
% «.as-questoes»	(to "as-questoes")
% «.lembre-que»		(to "lembre-que")
% «.dicas-edolccs»	(to "dicas-edolccs")
%
% «.djvuize»		(to "djvuize")



% <videos>
% Video (not yet):
% (find-ssr-links     "c2m231dicasp2" "2023-1-C2-dicas-pra-P2")
% (code-eevvideo      "c2m231dicasp2" "2023-1-C2-dicas-pra-P2")
% (code-eevlinksvideo "c2m231dicasp2" "2023-1-C2-dicas-pra-P2")
% (find-c2m231dicasp2video "0:00")

\documentclass[oneside,12pt]{article}
\usepackage[colorlinks,citecolor=DarkRed,urlcolor=DarkRed]{hyperref} % (find-es "tex" "hyperref")
\usepackage{amsmath}
\usepackage{amsfonts}
\usepackage{amssymb}
\usepackage{pict2e}
\usepackage[x11names,svgnames]{xcolor} % (find-es "tex" "xcolor")
\usepackage{colorweb}                  % (find-es "tex" "colorweb")
%\usepackage{tikz}
%
% (find-dn6 "preamble6.lua" "preamble0")
%\usepackage{proof}   % For derivation trees ("%:" lines)
%\input diagxy        % For 2D diagrams ("%D" lines)
%\xyoption{curve}     % For the ".curve=" feature in 2D diagrams
%
\usepackage{edrx21}               % (find-LATEX "edrx21.sty")
\input edrxaccents.tex            % (find-LATEX "edrxaccents.tex")
\input edrx21chars.tex            % (find-LATEX "edrx21chars.tex")
\input edrxheadfoot.tex           % (find-LATEX "edrxheadfoot.tex")
\input edrxgac2.tex               % (find-LATEX "edrxgac2.tex")
%\usepackage{emaxima}              % (find-LATEX "emaxima.sty")
%
% (find-es "tex" "geometry")
\usepackage[a6paper, landscape,
            top=1.5cm, bottom=.25cm, left=1cm, right=1cm, includefoot
           ]{geometry}
%
\begin{document}

% «defs»  (to ".defs")
% (find-LATEX "edrx21defs.tex" "colors")
% (find-LATEX "edrx21.sty")

\def\drafturl{http://anggtwu.net/LATEX/2023-1-C2.pdf}
\def\drafturl{http://anggtwu.net/2023.1-C2.html}
\def\draftfooter{\tiny \href{\drafturl}{\jobname{}} \ColorBrown{\shorttoday{} \hours}}

% (find-LATEX "2023-1-C2-carro.tex" "defs-caepro")
% (find-LATEX "2023-1-C2-carro.tex" "defs-pict2e")

\catcode`\^^J=10
\directlua{dofile "dednat6load.lua"}  % (find-LATEX "dednat6load.lua")

% «defs-T-and-B»  (to ".defs-T-and-B")
\long\def\ColorOrange#1{{\color{orange!90!black}#1}}
\def\T(Total: #1 pts){{\bf(Total: #1)}}
\def\T(Total: #1 pts){{\bf(Total: #1 pts)}}
\def\T(Total: #1 pts){\ColorRed{\bf(Total: #1 pts)}}
\def\B       (#1 pts){\ColorOrange{\bf(#1 pts)}}

% «defs-caepro»  (to ".defs-caepro")
%L dofile "Caepro5.lua"              -- (find-angg "LUA/Caepro5.lua" "LaTeX")
\def\Caurl   #1{\expr{Caurl("#1")}}
\def\Cahref#1#2{\href{\Caurl{#1}}{#2}}
\def\Ca      #1{\Cahref{#1}{#1}}

% «defs-pict2e»  (to ".defs-pict2e")
%L V = nil                           -- (find-angg "LUA/Pict2e1.lua" "MiniV")
%L dofile "Piecewise1.lua"           -- (find-LATEX "Piecewise1.lua")
%L Pict2e.__index.suffix = "%"
\def\pictgridstyle{\color{GrayPale}\linethickness{0.3pt}}
\def\pictaxesstyle{\linethickness{0.5pt}}
\def\pictnaxesstyle{\color{GrayPale}\linethickness{0.5pt}}
\celllower=2.5pt

\pu



%  _____ _ _   _                               
% |_   _(_) |_| | ___   _ __   __ _  __ _  ___ 
%   | | | | __| |/ _ \ | '_ \ / _` |/ _` |/ _ \
%   | | | | |_| |  __/ | |_) | (_| | (_| |  __/
%   |_| |_|\__|_|\___| | .__/ \__,_|\__, |\___|
%                      |_|          |___/      
%
% «title»  (to ".title")
% (c2m231dicasp2p 1 "title")
% (c2m231dicasp2a   "title")

\thispagestyle{empty}

\begin{center}

\vspace*{1.2cm}

{\bf \Large Cálculo 2 - 2023.1}

\bsk

Dicas pra P2

\bsk

Eduardo Ochs - RCN/PURO/UFF

\url{http://anggtwu.net/2023.1-C2.html}

\end{center}

\newpage

% «links»  (to ".links")


\newpage

% «as-questoes»  (to ".as-questoes")
% (c2m231dicasp2p 2 "as-questoes")
% (c2m231dicasp2a   "as-questoes")

{\bf As questões da prova}

\scalebox{0.55}{\def\colwidth{9.5cm}\firstcol{

A P2 vai ter:

\begin{itemize}

\item Uma questão de somas de Riemann, valendo 1 ponto;

\item Uma questão de volumes, valendo 1 ponto;

\item Uma questão de EDOs com variáveis separáveis (``EDOVSs''),

\item Uma questão de EDOs lineares com coeficientes constantes
  (``EDOLCCs'').

\end{itemize}

As questões de EDOs provavelmente vão valer 4 pontos cada uma.

\msk

Nas questões sobre EDOs o jeito de escrever vai importar {\bf MUITO}.
Se vocês não usarem as partículas em português do jeito certo as
respostas de vocês vai ficar ou erradas ou ambíguas, e:

\msk

\standout{
\begin{tabular}{l}
nessas questões toda vez que eu encontrar algo \\
ambíguo eu VOU interpretar do jeito errado!!!
\end{tabular}
}


}\anothercol{

{}

Lembre que PRA MIM um dos objetivos principais do curso de Cálculo 2 é
fazer as pessoas aprenderem a escrever contas muito complicadas de um
jeito que essas contas fiquem não só corretas como muito fáceis de
entender e de revisar!!!

\msk

Releia estas dicas várias vezes:

\ssk

\Ca{2gT4} (p.3) ``Releia a Dica 7''

\Ca{2gT23} (p.22) Formal vs.\ coloquial

\bsk

O melhor modo de treinar como escrever de forma clara e precisa é
tentar usar \standout{só} notação matemática e as ``partículas'' como
``seja'', ``então'', ``digamos que'', ``se'', ``i.e.'', ``queremos
que'', ``vamos testar se'', etc... isso vai te obrigar a fazer tudo na
ordem certa -- exatamente como em linguagens de programação, em que se
você trocar a ordem das linhas desse programa aqui ele vai fazer algo
errado:
%
% (find-LATEX "edrx21defs.tex" "co")
$$\begin{tabular}{l}
  \textsf{a = 42;}   \\
  \textsf{print(a);} \\
  \end{tabular}
$$

\msk

Você pode usar a VS do semestre passado como referência:

\Ca{2fT135}, \Ca{2fT137} Questão 1, anexo

% 2gQ53


}}

\newpage

% «lembre-que»  (to ".lembre-que")
% (c2m231dicasp2p 3 "lembre-que")
% (c2m231dicasp2a   "lembre-que")

{\bf ``Lembre que'' e ``queremos que''}

\scalebox{0.9}{\def\colwidth{12cm}\firstcol{

{}

A VS do semestre passado não tem exemplos de todas as partículas
importantes... por exemplo, ela não tem nem ``lembre que''s nem
``queremos que''s. Dê uma olhada aqui:

\msk

\Ca{2gQ53} Quadros da aula 25 (27/jun/2023)

\Ca{2gT20} (p.19) Contexto

\msk

% ...e lembre que 1) os livros básicos não usam o `[:=]', 2) eles 







% Dica 1: releia os capítulos que nós usamos no curso e encontre os
% lugares em que eles usam algo como o truque do ``lembre que'' pra usar
% alguma fórmula cuja demonstração é complicada sem precisar demonstrar
% ela de novo!
% 
% \ssk
% 
% Dica 2: Procure também os lugares em que eles fazem algo como o
% ``queremos que''!

}\anothercol{
}}




\newpage

% «dicas-edolccs»  (to ".dicas-edolccs")
% (c2m231dicasp2p 4 "dicas-edolccs")
% (c2m231dicasp2a   "dicas-edolccs")

{\bf Dicas pras questões de EDOLCCs}


\scalebox{0.55}{\def\colwidth{10cm}\firstcol{

Releia os quadros da aula de 23/jun/2023: \Ca{2gQ50}.

\ssk

As contas desses quadros têm alguns errinhos, você vai ter que
corrigi-los!

\ssk

Treine bastante as contas que mostram que $e^{ix}$ e $e^{-ix}$ são
combinações lineares de $\cos x$ e $\sen x$ e vice-versa e as contas
que mostram que $e^{(α+βi)x}$ e $e^{(α-βi)x}$ são combinações lineares
de $e^{γx}\cos δx$ e $e^{γx}\sen δx$ e vice-versa. Note que eu não
estou dizendo qual é a relação entre $α$ e $β$ e $γ$ e $δ$; você vai
ter que descobrir.

\ssk

Estude pelos livros!!!! Eles têm figuras que explicam porque algumas
EDOLCCs ``com raízes complexas'' descrevem oscilações.

\ssk

Em alguns itens do tipo ``teste o seu resultado'' algumas contas vão
ficar {\bf MUITO} mais curtas e fáceis de revisar se você souber
definir funções intermediárias e souber usar o ``seja'' e o ``então''.
Por exemplo, digamos que você precise calcular a quarta derivada de
$e^{42x}\sen 99x$; essa conta fica muito mais fácil de fazer se você
começar fazendo $f=e^{42x}$, $g=\sen 99x$ e $h=\cos 99x$. A maioria
das pessoas faz de tudo pra evitar aprender a usar funções
intermediárias, mas o Bob faz contas complicadas rápido porque ele
sabe usar funções intermediárias muito bem! \standout{Seja como o
  Bob!}

}\anothercol{

{}

Treine o modo de encontrar as raízes de polinômios de 2º grau
``simples'' por chutar-e-testar, sem usar Bhaskara -- ``simples'' aqui
quer dizer ``com raízes inteiras''. Por exemplo,
%
\def\und#1#2{\underbrace{#1}_{#2}}
%
$$\begin{array}{l}
  x^2-x-6 \\
  = \; (x-a)(x-b) \\
  = \; x^2-\und{(a+b)}{1}x+\und{ab}{-6} \\
  \end{array}
$$

$$\begin{array}{rrrrr}
  a & b & ab & a+b\!\!\!\!\! \\\hline
  -6 & 1 & -6 & -5 \\
  -3 & 2 & -6 & -1 \\
  -2 & 3 & -6 & 1  \\
  -1 & 6 & -6 & 5  \\
   1 & -6 & -6 & -5  \\
   2 & -3 & -6 & -1  \\
   3 & -2 & -6 & 1  \\
   6 & -1 & -6 & 5  \\
  \end{array}
$$

Dá pra fazer algo parecido com equações de 2º grau cujas raízes são
números complexos conjugados. Se $z=a+ib$ então $\ovl z=a-ib$; tente
expandir $(x-z)(x-\ovl z)$ e simplificar o resultado o máximo que você
puder, e depois faça alguns exemplos em que $a$ e $b$ são inteiros
pequenos e tente montar você mesmo um método como o da tabela ali de
acima.


}}


\GenericWarning{Success:}{Success!!!}  % Used by `M-x cv'

\end{document}

%  ____  _             _         
% |  _ \(_)_   ___   _(_)_______ 
% | | | | \ \ / / | | | |_  / _ \
% | |_| | |\ V /| |_| | |/ /  __/
% |____// | \_/  \__,_|_/___\___|
%     |__/                       
%
% «djvuize»  (to ".djvuize")
% (find-LATEXgrep "grep --color -nH --null -e djvuize 2020-1*.tex")

 (eepitch-shell)
 (eepitch-kill)
 (eepitch-shell)
# (find-fline "~/2023.1-C2/")
# (find-fline "~/LATEX/2023-1-C2/")
# (find-fline "~/bin/djvuize")

cd /tmp/
for i in *.jpg; do echo f $(basename $i .jpg); done

f () { rm -v $1.pdf;  textcleaner -f 50 -o  5 $1.jpg $1.png; djvuize $1.pdf; xpdf $1.pdf }
f () { rm -v $1.pdf;  textcleaner -f 50 -o 10 $1.jpg $1.png; djvuize $1.pdf; xpdf $1.pdf }
f () { rm -v $1.pdf;  textcleaner -f 50 -o 20 $1.jpg $1.png; djvuize $1.pdf; xpdf $1.pdf }

f () { rm -fv $1.png $1.pdf; djvuize $1.pdf }
f () { rm -fv $1.png $1.pdf; djvuize WHITEBOARDOPTS="-m 1.0 -f 15" $1.pdf; xpdf $1.pdf }
f () { rm -fv $1.png $1.pdf; djvuize WHITEBOARDOPTS="-m 1.0 -f 30" $1.pdf; xpdf $1.pdf }
f () { rm -fv $1.png $1.pdf; djvuize WHITEBOARDOPTS="-m 1.0 -f 45" $1.pdf; xpdf $1.pdf }
f () { rm -fv $1.png $1.pdf; djvuize WHITEBOARDOPTS="-m 0.5" $1.pdf; xpdf $1.pdf }
f () { rm -fv $1.png $1.pdf; djvuize WHITEBOARDOPTS="-m 0.25" $1.pdf; xpdf $1.pdf }
f () { cp -fv $1.png $1.pdf       ~/2023.1-C2/
       cp -fv        $1.pdf ~/LATEX/2023-1-C2/
       cat <<%%%
% (find-latexscan-links "C2" "$1")
%%%
}

f 20201213_area_em_funcao_de_theta
f 20201213_area_em_funcao_de_x
f 20201213_area_fatias_pizza



%  __  __       _        
% |  \/  | __ _| | _____ 
% | |\/| |/ _` | |/ / _ \
% | |  | | (_| |   <  __/
% |_|  |_|\__,_|_|\_\___|
%                        
% <make>

 (eepitch-shell)
 (eepitch-kill)
 (eepitch-shell)
# (find-LATEXfile "2019planar-has-1.mk")
make -f 2019.mk STEM=2023-1-C2-dicas-pra-P2 veryclean
make -f 2019.mk STEM=2023-1-C2-dicas-pra-P2 pdf

% Local Variables:
% coding: utf-8-unix
% ee-tla: "c2d2"
% ee-tla: "c2m231dicasp2"
% End:
