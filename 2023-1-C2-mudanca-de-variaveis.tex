% (find-LATEX "2023-1-C2-mudanca-de-variaveis.tex")
% (defun c () (interactive) (find-LATEXsh "lualatex -record 2023-1-C2-mudanca-de-variaveis.tex" :end))
% (defun C () (interactive) (find-LATEXsh "lualatex 2023-1-C2-mudanca-de-variaveis.tex" "Success!!!"))
% (defun D () (interactive) (find-pdf-page      "~/LATEX/2023-1-C2-mudanca-de-variaveis.pdf"))
% (defun d () (interactive) (find-pdftools-page "~/LATEX/2023-1-C2-mudanca-de-variaveis.pdf"))
% (defun e () (interactive) (find-LATEX "2023-1-C2-mudanca-de-variaveis.tex"))
% (defun o () (interactive) (find-LATEX "2022-2-C2-mudanca-de-variaveis.tex"))
% (defun u () (interactive) (find-latex-upload-links "2023-1-C2-mudanca-de-variaveis"))
% (defun v () (interactive) (find-2a '(e) '(d)))
% (defun d0 () (interactive) (find-ebuffer "2023-1-C2-mudanca-de-variaveis.pdf"))
% (defun cv () (interactive) (C) (ee-kill-this-buffer) (v) (g))
%          (code-eec-LATEX "2023-1-C2-mudanca-de-variaveis")
% (find-pdf-page   "~/LATEX/2023-1-C2-mudanca-de-variaveis.pdf")
% (find-sh0 "cp -v  ~/LATEX/2023-1-C2-mudanca-de-variaveis.pdf /tmp/")
% (find-sh0 "cp -v  ~/LATEX/2023-1-C2-mudanca-de-variaveis.pdf /tmp/pen/")
%     (find-xournalpp "/tmp/2023-1-C2-mudanca-de-variaveis.pdf")
%   file:///home/edrx/LATEX/2023-1-C2-mudanca-de-variaveis.pdf
%               file:///tmp/2023-1-C2-mudanca-de-variaveis.pdf
%           file:///tmp/pen/2023-1-C2-mudanca-de-variaveis.pdf
%  http://anggtwu.net/LATEX/2023-1-C2-mudanca-de-variaveis.pdf
% (find-LATEX "2019.mk")
% (find-sh0 "cd ~/LUA/; cp -v Pict2e1.lua Pict2e1-1.lua Piecewise1.lua ~/LATEX/")
% (find-sh0 "cd ~/LUA/; cp -v Pict2e1.lua Pict2e1-1.lua Pict3D1.lua ~/LATEX/")
% (find-sh0 "cd ~/LUA/; cp -v C2Subst1.lua C2Formulas1.lua ~/LATEX/")
% (find-sh0 "cd ~/LUA/; cp -v Gram2.lua Tree1.lua Caepro5.lua ~/LATEX/")
% (find-MM-aula-links "2023-1-C2-mudanca-de-variaveis" "C2" "c2m231mv" "c2mv")

% «.defs»		(to "defs")
% «.defs-caepro»	(to "defs-caepro")
% «.title»		(to "title")
% «.links»		(to "links")
% «.contas-1»		(to "contas-1")
% «.contas-2»		(to "contas-2")
% «.um-exemplo»		(to "um-exemplo")
% «.horriveis-1»	(to "horriveis-1")
% «.mais-anotacoes»	(to "mais-anotacoes")
% «.MVI»		(to "MVI")



% <videos>
% Video (not yet):
% (find-ssr-links     "c2m231mv" "2023-1-C2-mudanca-de-variaveis")
% (code-eevvideo      "c2m231mv" "2023-1-C2-mudanca-de-variaveis")
% (code-eevlinksvideo "c2m231mv" "2023-1-C2-mudanca-de-variaveis")
% (find-c2m231mvvideo "0:00")

%\documentclass[oneside,12pt]{article}
\documentclass[oneside,12pt]{article}
\usepackage[colorlinks,citecolor=DarkRed,urlcolor=DarkRed]{hyperref} % (find-es "tex" "hyperref")
\usepackage{amsmath}
\usepackage{amsfonts}
\usepackage{amssymb}
\usepackage{pict2e}
\usepackage[x11names,svgnames]{xcolor} % (find-es "tex" "xcolor")
\usepackage{colorweb}                  % (find-es "tex" "colorweb")
%\usepackage{tikz}
%
% (find-dn6 "preamble6.lua" "preamble0")
%\usepackage{proof}   % For derivation trees ("%:" lines)
%\input diagxy        % For 2D diagrams ("%D" lines)
%\xyoption{curve}     % For the ".curve=" feature in 2D diagrams
%
\usepackage{edrx21}               % (find-LATEX "edrx21.sty")
\input edrxaccents.tex            % (find-LATEX "edrxaccents.tex")
\input edrx21chars.tex            % (find-LATEX "edrx21chars.tex")
\input edrxheadfoot.tex           % (find-LATEX "edrxheadfoot.tex")
\input edrxgac2.tex               % (find-LATEX "edrxgac2.tex")
%\usepackage{emaxima}              % (find-LATEX "emaxima.sty")
%
% (find-es "tex" "geometry")
\usepackage[a6paper, landscape,
            top=1.5cm, bottom=.25cm, left=1cm, right=1cm, includefoot
           ]{geometry}
%
\begin{document}

% «defs»  (to ".defs")
% (find-LATEX "edrx21defs.tex" "colors")
% (find-LATEX "edrx21.sty")

\def\drafturl{http://anggtwu.net/LATEX/2023-1-C2.pdf}
\def\drafturl{http://anggtwu.net/2023.1-C2.html}
\def\draftfooter{\tiny \href{\drafturl}{\jobname{}} \ColorBrown{\shorttoday{} \hours}}

% (find-LATEX "2023-1-C2-carro.tex" "defs-caepro")
% (find-LATEX "2023-1-C2-carro.tex" "defs-pict2e")

\catcode`\^^J=10
\directlua{dofile "dednat6load.lua"}  % (find-LATEX "dednat6load.lua")

% «defs-caepro»  (to ".defs-caepro")
%L dofile "Caepro5.lua"              -- (find-angg "LUA/Caepro5.lua" "LaTeX")
\def\Caurl   #1{\expr{Caurl("#1")}}
\def\Cahref#1#2{\href{\Caurl{#1}}{#2}}
\def\Ca      #1{\Cahref{#1}{#1}}
\pu

\def\eqnpfull#1{\overset{\scriptscriptstyle(#1)}{=}}
\def\eqnpbare#1{=}
\def\eqnp      {\eqnpfull}

\def\redname#1{{\color{Red3}\text{#1}}}
\sa{II}{\redname{[II]}}
\sa{MVI}{\redname{[MVI]}}




%  _____ _ _   _                               
% |_   _(_) |_| | ___   _ __   __ _  __ _  ___ 
%   | | | | __| |/ _ \ | '_ \ / _` |/ _` |/ _ \
%   | | | | |_| |  __/ | |_) | (_| | (_| |  __/
%   |_| |_|\__|_|\___| | .__/ \__,_|\__, |\___|
%                      |_|          |___/      
%
% «title»  (to ".title")
% (c2m231mvp 1 "title")
% (c2m231mva   "title")

\thispagestyle{empty}

\begin{center}

\vspace*{1.2cm}

{\bf \Large Cálculo C2 - 2023.1}

\bsk

Aulas 10 até 13: mudança de variáveis

\bsk

Eduardo Ochs - RCN/PURO/UFF

\url{http://anggtwu.net/2023.1-C2.html}

\end{center}

\newpage

% «links»  (to ".links")
% (c2m231mvp 2 "links")
% (c2m231mva   "links")
% (c2m222mvp 2 "links")
% (c2m222mva   "links")

{\bf Links}

\scalebox{0.6}{\def\colwidth{12cm}\firstcol{

Mudança de variável na integral definida (MVD):

% (c2m221atisp 12 "substituicao-figura")
% (c2m221atisa    "substituicao-figura")
\Ca{2eT131} (t-ints, p.12) Uma figura pra mudança de variável

% (find-books "__analysis/__analysis.el" "thomas" "Substitution in definite integrals")
\Ca{Thomas55p11} (p.376) Theorem 5: Substitution in definite integrals

\Ca{2fT49} Meu PDF de 2022.2 sobre mudança de variáveis

\bsk

Mudança de variável na integral indefinida (MVI):

% (c2m221atisp 14 "exemplo-contas")
% (c2m221atisa    "exemplo-contas")
\Ca{2eT133} (t-ints, p.14) Um exemplo com contas

% (c2m221atisp 16 "exemplo-contas-2")
% (c2m221atisa    "exemplo-contas-2")
\Ca{2eT135} (t-ints, p.16) Outro exemplo com contas

% (find-books "__analysis/__analysis.el" "thomas")
% (find-books "__analysis/__analysis.el" "thomas" "5: The substitution rule")
% (find-books "__analysis/__analysis.el" "thomas" "5: The substitution rule" "Example 3")
\Ca{Thomas55p3} (p.370) Theorem 5: The substitution rule

% (find-books "__analysis/__analysis.el" "leithold")
% (find-books "__analysis/__analysis.el" "leithold" "5.2.1. Regra da cadeia")
% (find-books "__analysis/__analysis.el" "leithold" "9.2" "potências de seno e co-seno")
\Ca{Leit5p13} (p.296) A regra da cadeia para a antidiferenciação

\Ca{Leit9p10} (p.537) Integração de potências de sen e cos

% (find-dmirandacalcpage 189 "6.2 Integração por Substituição")
% (find-dmirandacalcpage 192   "Exemplo 6.6")
% (find-dmirandacalcpage 193   "não podemos")
% (find-dmirandacalcpage 196   "Exercícios")
% (find-dmirandacalcpage 255 "8.3 Integrais Trigonométricas")
\Ca{Miranda189} 6.2. Integração por substituição

\Ca{Miranda192} Exemplo 6.6

\Ca{Miranda193} Não podemos

\Ca{Miranda196} Exercícios

\Ca{Miranda255} 8.3 Integrais Trigonométricas

\msk

Vídeo do Reginaldo:

\url{https://www.youtube.com/watch?v=PTCUjrEBc4g}

\msk

\msk

\par \Ca{2gQ22} Quadros da aula 10 (05/maio/2023)
\par \Ca{2gQ24} Quadros da aula 11 (09/maio/2023)
\par \Ca{2gQ26} Quadros da aula 12 (12/maio/2023)
\par \Ca{2gQ28} Quadros da aula 13 (16/maio/2023)



% (c2m221vsbp 8 "questao-3-gab")
% (c2m221vsba   "questao-3-gab")
% (find-books "__analysis/__analysis.el" "miranda")
% (find-books "__analysis/__analysis.el" "miranda" "6.2 Integração por Substituição")
% (find-books "__analysis/__analysis.el" "miranda" "Exemplo 6.6")
% (find-books "__analysis/__analysis.el" "miranda" "8.3 Integrais Trigonométricas")

% (find-fline "/home/angg_slow_html/eev-videos/" "2020_int_subst_1.mp4")
% (find-LATEX "2020-1-C2-int-subst.tex" "videos" "2020_int_subst_1")

}\anothercol{
}}

\newpage

% «contas-1»  (to ".contas-1")
% (c2m231mvp 3 "contas-1")
% (c2m231mva   "contas-1")

{\bf Contas (1)}

\scalebox{0.9}{\def\colwidth{12cm}\firstcol{

$$\begin{array}{rcl}
  \Intx  {a}{b}{f'(x)}    & \eqnp {1} & \difx{a}{b}{f(x)} \\
  \Intu  {α}{β}{f'(u)}    & \eqnp {2} & \difu{α}{β}{f(u)} \\
  \Intx  {a}{b}{ \cos  x} & \eqnp {3} & \difx  {a}{b}{ \sen  x} \\
  \Intx  {a}{b}{2\cos 2x} & \eqnp {4} & \difx  {a}{b}{ \sen 2x} \\
                          & \eqnp {5} & \sen 2b - \sen 2a \\
                          & \eqnp {6} & \difu{2a}{2b}{ \sen  u} \\
  \Intu  {α}{β}{ \cos  u} & \eqnp {7} & \difu  {α}{β}{ \sen  u} \\
  \Intu{2a}{2b}{ \cos  u} & \eqnp {8} & \difu{2a}{2b}{ \sen  u} \\
                          & \eqnp {9} & \sen 2b - \sen 2a \\
                          & \eqnp{10} & \difx  {a}{b}{ \sen 2x} \\
                          & \eqnp{11} & \Intx  {a}{b}{2\cos 2x} \\
  \Intx  {a}{b}{2\cos 2x} & \eqnp{12} & \Intu{2a}{2b}{ \cos  u} \\
  \end{array}
$$

}\anothercol{
}}

\newpage

% «contas-2»  (to ".contas-2")
% (c2m231mvp 4 "contas-2")
% (c2m231mva   "contas-2")

{\bf Contas (2)}

\scalebox{0.7}{\def\colwidth{12cm}\firstcol{

$$\begin{array}{rcl}
  \Intx  {a}{b}{f'(x)}         & \eqnp {1} & \difx    {a}{b}{f(x)}     \\
  \Intu  {α}{β}{f'(u)}         & \eqnp {2} & \difu    {α}{β}{f(u)}     \\ \\[-5pt]
  %
  \Intx  {a}{b}{(\cos x^2)·2x} & \eqnp {3} & \difx    {a}{b}{\sen x^2} \\
                               & \eqnp {4} & \sen b^2 - \sen a^2       \\
                               & \eqnp {5} & \difu{a^2}{b^2}{\sen   u} \\
  \Intu{a^2}{b^2}{\cos u}      & \eqnp {6} & \difu{a^2}{b^2}{\sen   u} \\ \\[-5pt]
  \Intx  {a}{b}{(\cos x^2)·2x} & \eqnp {7} & \Intu{a^2}{b^2}{\cos   u} \\ \\[-5pt]
  %
  \Intx  {a}{b}{g'(h(x))h'(x)} & \eqnp {8} & \difx      {a}{b}{g(h(x))} \\
                               & \eqnp {9} & g(h(b)) - g(h(a))          \\
                               & \eqnp {10} & \difu{h(a)}{h(b)}{g(u)}   \\
  \Intu{h(a)}{h(b)}{g'(u)}     & \eqnp {11} & \difu{h(a)}{h(b)}{g(u)}   \\ \\[-5pt]
  \Intx  {a}{b}{g'(h(x))h'(x)} & \eqnp {12} & \Intu{h(a)}{h(b)}{g'(u)}  \\ \\[-5pt]
  \Intx  {a}{b}{g (h(x))h'(x)} & \eqnp {13} & \Intu{h(a)}{h(b)}{g (u)}  \\
  \end{array}
$$

}\anothercol{
}}


\newpage

%  _   _                                           _       
% | | | |_ __ ___     _____  _____ _ __ ___  _ __ | | ___  
% | | | | '_ ` _ \   / _ \ \/ / _ \ '_ ` _ \| '_ \| |/ _ \ 
% | |_| | | | | | | |  __/>  <  __/ | | | | | |_) | | (_) |
%  \___/|_| |_| |_|  \___/_/\_\___|_| |_| |_| .__/|_|\___/ 
%                                           |_|            
% «um-exemplo»  (to ".um-exemplo")
% (c2m231mvp 5 "um-exemplo")
% (c2m231mva   "um-exemplo")
% (c2m221atisp 14 "exemplo-contas")
% (c2m221atisa    "exemplo-contas")
% (c2m221atisp 16 "exemplo-contas-2")
% (c2m221atisa    "exemplo-contas-2")
% \Ca{2eT133} (t-ints, p.14) Um exemplo com contas
% \Ca{2eT135} (t-ints, p.16) Outro exemplo com contas

{\bf Um exemplo}

\sa{2 cos(3x+4) full}{
  \begin{array}{l}
  \D \Intx{a}{b}{2 \cos(3x+4)}                                 \\[8pt]
  = \;\; \D \Intu{3a+4}{3b+4} {2 (\cos u) · \frac13}           \\[8pt]
  = \;\; \D \frac23 \Intu{3a+4}{3b+4} {\cos u}                 \\[8pt]
  = \;\; \D \frac23 \left(\difu{3a+4}{3b+4} {(\sen u)} \right) \\[8pt]
  = \;\; \D \frac23 \left(\difx{a}{b} {(\sen (3x+4))} \right)  \\
  \end{array}
}
\sa{2 cos(3x+4) thin}{
  \begin{array}{l}
  \D \intx{2 \cos(3x+4)}                 \\[8pt]
  = \;\; \D \intu {2 (\cos u) · \frac13} \\[8pt]
  = \;\; \D \frac23 \intu{\cos u}        \\[8pt]
  = \;\; \D \frac23 \sen u               \\[8pt]
  = \;\; \D \frac23 \sen (3x+4)          \\
  \end{array}
}

\scalebox{0.65}{\def\colwidth{8.5cm}\firstcol{

Isto aqui é um exemplo de como contas com mudança
de variável costumam ser feitas na prática:
%
$$\scalebox{0.95}{$
  \ga{2 cos(3x+4) thin}
  $}
$$

É necessário indicar em algum lugar que a relação
entre a variável nova e a antiga é esta: $u=3x+4$.

\msk

Compare com:

\Ca{Miranda189} 6.2: Integração por substituição

\Ca{Leit5p13} (p.296) Teorema 5.2.1: a regra da

cadeia para a antidiferenciação

\Ca{Leit5p16} (p.299) Exemplo 5

}\anothercol{

  Compare as contas à esquerda, que não têm nem os limites de
  integração nem as barras de diferença, com estas:
  % 
  $$\scalebox{0.8}{$
    \ga{2 cos(3x+4) full}
    $}
  $$

\ssk

Nós vamos tratar a versão à esquerda como uma abreviação pra versão da
direita. Note que pra ir da versão ``completa'' pra ``abreviada'' é
super fácil, é só apagar os limites de integração e as barras de
diferença -- mas pra ir da versão ``abreviada'' pra ``completa'' a
gente precisa reconstruir os limites de integração e as barras de
diferença, o que é bem mais difícil.

}}

\newpage

{\bf Caixinhas de anotações}

\scalebox{0.725}{\def\colwidth{7.5cm}\firstcol{

  O meu truque preferido pra não me enrolar nas contas de uma mudança
  de variável é fazer uma caixinha de anotações como essa aqui,
  %
  $$\bmat{
    u = 3x+4 \\
    \frac{du}{dx} = \ddx(3x+4) = 3 \\
    \frac{du}{dx} = 3 \\
    \ColorRed{du = 3 \, dx} \\
    \ColorRed{dx = \frac13 \,du} \\
    }
  $$

  na qual: a) a primeira linha diz a relação entre a variável antiga e
  a variável nova -- que nesse exemplo é $u=3x+4$, b) todas as outras
  linhas da caixinha são consequências dessa primeira, e c) dentro da
  caixinha a gente permite gambiarras como:
  %
  $$dx = 42\,du$$

}\anothercol{

  Durante quase todo o curso de C2 a gente vai tratar esse tipo de
  coisa como uma igualdade entre expressões incompletas -- mais ou
  menos como se a gente estivesse dizendo isso aqui:
  %
  $$+20) = /99]$$

  Na caixinha à esquerda eu colori as linhas que são gambiarras em
  vermelho.

  \bsk
  \bsk
  \bsk

  Aqui tem um exemplo grande:

  \Ca{2fT112} (C2-P1, p.5) Questão 1: gabarito


}}


\newpage

% «horriveis-1»  (to ".horriveis-1")
% (c2m231mvp 5 "horriveis-1")
% (c2m231mva   "horriveis-1")

{\bf Os detalhes horríveis}

\scalebox{0.55}{\def\colwidth{10cm}\firstcol{

    Nesta página aqui -- \Ca{Miranda193} -- o Miranda diz ``Não
    podemos calcular uma integral que possui tanto um $x$ e um $u$
    nela'', mas ele não explica porquê... se em
    % 
    $$\Intx{a}{b}{2 \cos(u)}$$
    % 
    esse $u$ fosse uma abreviação para $3x+4$ essa integral acima
    seria equivalente à do início do slide anterior, né?... \frown

    \msk

    Neste slide eu vou tentar contar o que eu sei sobre como o método
    da substituição funciona -- {\sl pra convencer vocês de que não
      vale a pena vocês tentarem entender os detalhes agora}.

    \msk

    Toda mudança de variável numa integral definida é consequência da
    igualdade (13) do slide ``Contas (2)''. Por exemplo, compare:
    %
    $$\begin{array}{rcl}
      \D \Intx{a}{b}{g (h(x))h'(x)} &\eqnp{13}& \D \Intu{h(a)}{h(b)}{g (u)} \\
      \D \Intx{a}{b}{2 \cos(3x+4)}  & =       & \D \Intu{3a+4}{3b+4}{2(\cos u)·\frac13} \\
      \end{array}
    $$

}\anothercol{

  A gente pode tentar descobrir qual é a substituição certa passo a
  passo, começando pelas funções mais simples.... eu faria assim:
  olhando pra parte direita eu chuto que $g(u) = 2(\cos u)·\frac13$;
  olhando pra parte esquerda eu chuto que $h(x) = 3x+4$, e daí
  $h'(x) = 3$; aí eu testo esta substituição aqui,
  %
  $$(13) \bmat{g(u):=2(\cos u)·\frac13 \\
               h(x):=3x+4 \\
               h'(x):=3 \\
              }
  $$

  e vejo que o resultado dela é {\sl equivalente} (mas não igual!!!) à
  última igualdade da coluna da esquerda -- não preciso nem substituir
  o $a$ e o $b$.

}}

\newpage

{\bf Os detalhes horríveis (2)}

\scalebox{0.62}{\def\colwidth{14cm}\firstcol{

Estas contas aqui,
%
$$\begin{array}{rcl}
  u &=& x^4 \\
  \frac{du}{dx} &=& 4x^3 \\
  du &=& \frac{du}{dx} dx \\
     &=& 4x^3 \, dx \\
  \end{array}
$$

fazem sentido se a gente considerar que:

\msk

1. $x$ é uma variável independente,

2. $u$ é uma variável dependente, com $u=u(x)=x^4$,

3. $dx$ é uma variável independente,

4. $du$ é uma variável dependente, com $du=\frac{du}{dx}dx$,

5. estas regras sobre diferenciais valem: \Ca{Leit4p61} (p.275),

6. estas regras sobre variáveis dependentes valem: \Ca{Stew14p53} (p.951),

7. o $dx$ num $\intx{f(x)}$ funciona como uma diferencial.

\msk

Eu já perguntei pra vários matemáticos fodões que eu conheço --
incluindo os desenvolvedores do Maxima, na mailing list -- onde eu
posso encontrar alguma formalização das regras de como lidar com
variáveis dependentes, diferenciais e mudança de variável na integral
indefinida, e todos eles me responderam a mesma coisa: ``{\sl não faço
  a menor idéia! Eu sei algumas das regras mas não todas, e não sei
  onde você pode procurar...}'' \frown

\msk

Moral: \standout{é melhor a gente tratar o $du = 4x^3 \, dx$ como uma
  gambiarra...}

}\anothercol{
}}

\newpage

% «mais-anotacoes»  (to ".mais-anotacoes")
% (c2m231mvp 9 "mais-anotacoes")
% (c2m231mva   "mais-anotacoes")

\def\S{\senθ}
\def\C{\cosθ}

{\bf Caixinhas com mais anotações}

\scalebox{0.8}{\def\colwidth{8cm}\firstcol{

$$\begin{array}{rcl}
  \intth{(\S)^4(\C)^7} &=& \intth{(\S)^4(\C)^6\C}       \\
                       &=& \intth{(\S)^4((\C)^2)^3\C}   \\
                       &=& \intth{(\S)^4(1-(\S)^2)^3\C} \\
                       &=& \ints {   s^4(1-   s^2)^3  } \\
  \end{array}
$$

$$\begin{array}{rcl}
  \intth{(\S)^4(\C)^7} &=& \intth{(\S)^4(\C)^6\C}       \\
                       &=& \ints {   s^4(1-   s^2)^3  } \\
  \end{array}
$$

}\anothercol{

\vspace*{0.25cm}

$$\bmat{\senθ = s \\
        \frac{ds}{dθ} = \frac{d}{dθ}\senθ = \cosθ \\
        ds = \cosθ \,dθ \\
        \cosθ \,dθ = ds \\
       }
$$

\bsk

$$\bmat{\senθ = s \\
        \frac{ds}{dθ} = \frac{d}{dθ}\senθ = \cosθ \\
        ds = \cosθ \,dθ \\
        \cosθ \,dθ = ds \\
        (\C)^2 = 1-(\S^2) \\
        (\C)^2 = 1-s^2 \\
        (\C)^6 = (1-s^2)^3 \\
       }
$$

}}


\newpage

{\bf Caixinhas com mais anotações (2)}

\scalebox{0.6}{\def\colwidth{7.5cm}\firstcol{

$$\begin{array}{rcl}
  \D \ints{s \sqrt{1-s^2}} &=& \D \intth{(\S) \sqrt{1-(\S)^2} \C} \\
                           &=& \D \intth{(\S) \sqrt{(\C)^2} \C} \\
                           &=& \D \intth{(\S) (\C) \C} \\
                           &=& \D \intth{(\S) (\C)^2} \\
  \end{array}
$$

$$\begin{array}{rcl}
  \D \ints{s \sqrt{1-s^2}} &=& \D \intth{(\S) (\C) \C} \\
                           &=& \D \intth{(\S) (\C)^2} \\
  \end{array}
$$

$$\begin{array}{rcl}
  \D \ints{\frac{1}{\sqrt{1-s^2}}} &=& \D \intth{\frac{1}{\C} \C} \\
                                   &=& \D \intth{1} \\
                                   &=& θ \\
                                   &=& \arcsen s \\
  \end{array}
$$

}\anothercol{

\vspace*{0.25cm}

$$\bmat{s = \senθ \\
        \frac{ds}{dθ} = \frac{d}{dθ}\senθ = \cosθ \\
        ds = \cosθ \,dθ \\
       }
$$

\vspace*{3cm}

$$\bmat{s = \senθ \\
        \frac{ds}{dθ} = \frac{d}{dθ}\senθ = \cosθ \\
        ds = \cosθ \,dθ     \\
        s^2 = (\S)^2        \\
        1 - s^2 = 1-(\S)^2  \\
        1 - s^2 = (\C)^2    \\
        \sqrt{1 - s^2} = \C \\
        \arcsen s = \arcsen \sen θ \\
        \arcsen s = θ \\
        θ = \arcsen s \\
       }
$$

}}


\newpage

{\bf O macaco, de novo}

\scalebox{0.7}{\def\colwidth{8cm}\firstcol{

Estas duas igualdades são falsas
%
$$\begin{array}{rcl}
  \sqrt{1-(\S)^2} &=& \cos θ \\
  \arcsen \sen θ  &=& θ \\
  \end{array}
$$

quando $θ=π$... confira!

\msk

Mas elas são verdadeiras para $θ=0$, e para todo $θ$ num certo
intervalo em torno do 0 que eu não quero contar qual é.

\msk

Lembre quem em Cálculo 2 a gente vai primeiro fazer as contas como o
macaco que faz todas as contas como se tudo funcionasse, e a gente vai
deixar pra checar os detalhes, como se $θ$ está no intervalo certo, só
no final, depois de termos feito as contas todas.

}\anothercol{

  O Leithold é super cuidadoso nas contas e nesses detalhes como os
  domínios da funções e o intervalo onde mora o $θ$, mas a maioria dos
  outros livros de Cálculo 2 que eu conheço não são -- eles são meio
  porcalhões com esses detalhes... e a gente também vai ser, senão não
  vai dar tempo de cobrir o suficiente da matéria.

}}


\newpage

{\bf Desabreviando o $42=99$}

\scalebox{0.675}{\def\colwidth{8cm}\firstcol{

    Lembre que a nossa regra básica pra integral indefinida é esta
    aqui,
    %
    $$\ga{II} \;=\; \left( \D\intx{f'(x)} = f(x) \right)$$

    e eu usei ela pra demostrar isto aqui:
    %
    $$\begin{array}{rcll}
        \D \intx{0} &\eqnp{1}& 42 \\
        \D \intx{0} &\eqnp{2}& 99 \\
                 42 &\eqnp{3}& 99 \\
      \end{array}
    $$

    As justificativas são:

    (1): por $\ga{II}$, com $f(x)=42$

    (2): por $\ga{II}$, com $f(x)=99$

    (3): por (1) e (2)

}\anothercol{

  Se a gente desabreviar as contas da esquerda -- como num dos
  primeiros slides -- a gente obtém isto aqui:
  % 
  $$\begin{array}{rcll}
      \D \Intx{a}{b}{0} &\eqnp{4}& \difx{a}{b}{42} \\
      \D \Intx{a}{b}{0} &\eqnp{5}& \difx{a}{b}{99} \\
        \difx{a}{b}{42} &\eqnp{6}& \difx{a}{b}{99} \\
    \end{array}
  $$

  E agora a igualdade (6) é claramente verdade -- confira!

}}


\newpage

{\bf O truque dos intervalos}

\scalebox{0.9}{\def\colwidth{10cm}\firstcol{

    Dê uma olhada nas primeiras páginas daqui:

    \ssk

    \Ca{Leit5p3} 5.1. Antidiferenciação

    \msk

    O Leithold usa expressões como ``num intervalo $I$'', ``para todo
    $x∈I$'' e ``definidas no mesmo intervalo'' um montão de vezes. O
    truque de usar sempre intervalos resolve esse esse problema daqui
    super bem:

    \ssk

    \Ca{2fT24} Meme: expanding brain, versão ln

    \bsk

    A minha definição preferida pra integral indefinida,

    \ssk

    \Ca{2fT23} Outra definição pra integral indefinida

    \ssk

    também resolve o problema -- de um modo bem mais simples, e que é
    suficiente pro tipo de conta que a gente tem que treinar em
    Cálculo 2.


}\anothercol{
}}



\newpage

%  __  ____     _____ 
% |  \/  \ \   / /_ _|
% | |\/| |\ \ / / | | 
% | |  | | \ V /  | | 
% |_|  |_|  \_/  |___|
%                     
% «MVI»  (to ".MVI")
% (c2m231mvp 14 "MVI")
% (c2m231mva    "MVI")

{\bf MVI}

\def\und#1#2{\underbrace{#1}_{#2}}
\def\P#1{\left(#1\right)}
\sa{MVA H short}{
   \D \intx{f'(g(x))g'(x)}
   \;=\;
   \D \intu{f'(u)}
  }
\sa{MVA H}{
   \D \intx{f'(g(x))g'(x)}
   \;=\;
   f(g(x))
   \;=\;
   f(u)
   \;=\;
   \D \intu{f'(u)}
  }
\sa{MVA Hund}{
   \und{ \D\ddx\P{ \intx{f'(g(x))g'(x)}} }{f'(g(x))g'(x)}
   \;=\;
   \und{ \D\ddx f(g(x)) }{f'(g(x))g'(x)}
   \;=\;
   \und{ \D\ddx \und{f(u)}{f(g(x))} }{f'(g(x))g'(x)}
   \;=\;
   \und{ \D\ddx \und{\und{\intu{f'(u)}}{f(u)}}{f(g(x))} }{f'(g(x))g'(x)}
  }


\scalebox{0.6}{\def\colwidth{15cm}\firstcol{

% (c2m222mvp 4 "justificando-cada")
% (c2m222mva   "justificando-cada")
% \Ca{2fT52} Justificando cada igualdade

    A nossa fórmula pra mudança de variável na integral indefinida vai
    ser esta aqui:
    %
    $$\ga{MVI} \;=\; \P{\ga{MVA H short}}$$

    Dá pra demonstrar ela deste jeito,
    %
    $$\ga{MVA H}$$

    onde a primeira e a terceira igualdades são consequências do
    $\ga{II}$, e a igualdade do meio só vale se tivermos $u=g(x)$.

    \msk

    Os livros demonstram a $\ga{MVI}$ de um jeitos que eu nunca achei
    muito convincentes -- ou fingindo que tudo é óbvio, ou ``derivando
    tudo em $x$''. As contas abaixo me ajudaram a entender o que
    acontece quando a gente ``deriva tudo em $x$'':

    % \Ca{Miranda189} 6.2: Integração por substituição

    $$\ga{MVA Hund}$$

}\anothercol{
}}


\newpage

% (c2m222strigp 3 "exercicio-1")
% (c2m222striga   "exercicio-1")

{\bf Simplificando raizes quadradas}

\scalebox{0.6}{\def\colwidth{9cm}\firstcol{

Na aula de 16/maio/2023 você aprendeu -- na prática, não vendo uma
definição formal -- o que é transformar uma integral mais difícil
numa integral mais fácil, que nós sabemos integrar...

\ssk

a) Digamos que você sabe integrar $\ints{\sqrt{1-s^2}}$.

Transforme $\intx{\sqrt{1-(5x)^2}}$ em algo que você sabe integrar.

\ssk

b) Transforme $\intx{\sqrt{1-(ax)^2}}$ em algo que você sabe integrar.

\ssk

c) Digamos que você sabe integrar $\ints{\sqrt{1-s^2}^{\,k}}$ para
qualquer valor de $k$.

Transforme $\intx{{\sqrt{1-(5x)^2}}^{\,42}}$ em algo que você sabe
integrar.

\ssk

d) Transforme $\intx{\sqrt{1-(ax)^2}^{\,42}}$ em algo que você sabe
integrar.

\ssk

e) Transforme $\intx{\sqrt{1-(ax)^2}^{\,k}}$ em algo que você sabe integrar.

\ssk

f) Transforme $\intx{\sqrt{1-(ax)^2}^{\,k}}$ em algo que você sabe integrar.

}\anothercol{

\ssk

g) Entenda este truque aqui:
%
$$\begin{array}{rcl}
  \sqrt{3^2 - x^2} &=& \sqrt{3^2 - 3^2 \frac{1}{3^2} x^2} \\
                   &=& \sqrt{3^2 - 3^2(\frac x3)^2} \\
                   &=& \sqrt{3^2(1 - (\frac x3)^2)} \\
                   &=& \sqrt{3^2}\sqrt{1 - (\frac x3)^2} \\
                   &=& 3\sqrt{1 - (\frac x3)^2} \\
  \end{array}
$$

Use ele -- com adaptações, óbvio -- pra transformar
$\intx{\sqrt{25-x^2}}$ em algo que você sabe integrar.

\ssk

h) Use ele pra transformar
$\intx{\sqrt{25-x^2}^{\,42}}$ em algo que você sabe integrar.

\ssk

i) Use ele pra transformar $\intx{\sqrt{a^2-x^2}}$ em algo que você
sabe integrar.

\ssk

j) Use ele pra transformar
$\intx{\sqrt{a^2-x^2}^{\,k}}$ em algo que você sabe integrar.

\ssk

j) Use ele pra transformar $\intx{x^{20} \sqrt{a^2-x^2}^{\,k}}$ em
algo que você sabe integrar.


}}



\newpage

{\bf Exercício 2}

\sa{[DFI]}{\CFname{DFI}{}}


\scalebox{0.58}{\def\colwidth{9cm}\firstcol{

(Obs: ainda não atualizei este slide!)

\ssk

No final da aula de 28/set/2022 -- veja a foto do quadro:

\ssk

{\scriptsize

% (find-angg ".emacs" "c2q222")
% (find-angg ".emacs" "c2q222" "22" "set28: substituição trigonométrica")
% http://angg.twu.net/2022.2-C2/C2-quadros.pdf#page=23
\url{http://angg.twu.net/2022.2-C2/C2-quadros.pdf\#page=23}

}

\ssk

nós vimos que a demonstração de que $\ddx \ln x = \frac1x$ pode ser
generalizada, e aí a gente obtém a ``fórmula da derivada da função
inversa'', que eu chamei de \ga{[DFI]}...

Essa generalização pode ser ``especializada'' pra obter outros casos
particulares diferentes de $\ddx \ln x = \frac1x$.

\msk

a) Faça o primeiro exercício que eu pus no quadro:
%
$$\ga{[DFI]}
  \bmat{
    g(x) := \arcsen x \\
    g'(x) := \arcsen' x \\
    f(x) := \sen x \\
    f'(x) := \cos x \\
  } = \Rq
$$

b) Faça o segundo exercício do quadro:
%
$$\ga{[DFI]}
  \bmat{
    g(x) := \arcsen x \\
    g'(x) := \arcsen' x \\
    f(x) := \sen x \\
    f'(x) := \sqrt{1 - (\sen x)^2} \\
  } = \Rq
$$

}\anothercol{



  c) Use as identidades trigonométricas que vamos ver em sala pra
  encontrar uma fórmula pra derivada do $\arctan$.

\msk

  d) Use as identidades trigonométricas que vamos ver em sala pra
  encontrar uma fórmula pra derivada do $\arcsec$.


}}



\newpage

{\bf Exercício 3}


\scalebox{0.54}{\def\colwidth{10cm}\firstcol{

Slogan:

\begin{quote}

  {\sl Toda integral que pode ser resolvida por uma sequência de mudanças
  de variável pode ser resolvida por uma mudança de variável só.}

\end{quote}

Durante a quarentena eu dei algumas questões de prova sobre este
slogan. Dê uma olhada:

\ssk

{\footnotesize

% (c2m202p1p 4 "questao-2")
% (c2m202p1a   "questao-2")
%    http://angg.twu.net/LATEX/2020-2-C2-P1.pdf#page=4
\url{http://angg.twu.net/LATEX/2020-2-C2-P1.pdf\#page=4}

% (c2m202p1p 9 "gabarito-2")
% (c2m202p1a   "gabarito-2")
%    http://angg.twu.net/LATEX/2020-2-C2-P1.pdf#page=9
\url{http://angg.twu.net/LATEX/2020-2-C2-P1.pdf\#page=9}

% (c2m211p1p 15 "gabarito-2-2020.2")
% (c2m211p1a    "gabarito-2-2020.2")
%    http://angg.twu.net/LATEX/2021-1-C2-P1.pdf#page=15
\url{http://angg.twu.net/LATEX/2021-1-C2-P1.pdf\#page=15}

}

\msk

a) Resolva a integral abaixo usando uma mudança de variável só (dica:
$u=g(h(x))$):
%
$$\intx{f'(g(h(x)))g'(h(x))h'(x)} = \Rq$$

b) Resolva a integral acima usando duas mudanças de variável. Dica:
comece com $u=h(x)$.

\bsk
\bsk

O Miranda e o Leithold preferem fazer em um passo só certas mudanças
de variáveis que eu prefiro fazer em dois ou três passos. Entenda o
exemplo 8.1 do Miranda -- o da seção 8.4, na página 264...

\ssk

{\scriptsize

% (find-books "__analysis/__analysis.el" "miranda")
% (find-books "__analysis/__analysis.el" "miranda" "8.4 Substituição Trigonométrica")
% (find-dmirandacalcpage 263 "8.4 Substituição Trigonométrica")
% http://hostel.ufabc.edu.br/~daniel.miranda/calculo/calculo.pdf#263
\url{http://hostel.ufabc.edu.br/~daniel.miranda/calculo/calculo.pdf\#263}

}

}\anothercol{

% a

  c) ...e descubra como resolver a integral dele fazendo duas mudanças
  de variáveis ao invés de uma só. A segunda mudança de variável vai
  ser $s = \sen θ$, e a primeira eu prefiro não contar qual é -- tente
  usar as idéias do exercício 1 pra descobrir qual ela tem que ser.


\bsk
\bsk
\bsk
\bsk
\bsk
\bsk
\bsk
\bsk
\bsk
\bsk
\bsk

{\sl (Obs: ainda não atualizei este slide!)}



}}





\GenericWarning{Success:}{Success!!!}  % Used by `M-x cv'

\end{document}


% Local Variables:
% coding: utf-8-unix
% ee-tla: "c2mv"
% ee-tla: "c2m231mv"
% End:
