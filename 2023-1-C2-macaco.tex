% (find-LATEX "2023-1-C2-macaco.tex")
% (defun c () (interactive) (find-LATEXsh "lualatex -record 2023-1-C2-macaco.tex" :end))
% (defun C () (interactive) (find-LATEXsh "lualatex 2023-1-C2-macaco.tex" "Success!!!"))
% (defun D () (interactive) (find-pdf-page      "~/LATEX/2023-1-C2-macaco.pdf"))
% (defun d () (interactive) (find-pdftools-page "~/LATEX/2023-1-C2-macaco.pdf"))
% (defun e () (interactive) (find-LATEX "2023-1-C2-macaco.tex"))
% (defun o () (interactive) (find-LATEX "2023-1-C2-macaco.tex"))
% (defun u () (interactive) (find-latex-upload-links "2023-1-C2-macaco"))
% (defun v () (interactive) (find-2a '(e) '(d)))
% (defun d0 () (interactive) (find-ebuffer "2023-1-C2-macaco.pdf"))
% (defun cv () (interactive) (C) (ee-kill-this-buffer) (v) (g))
%          (code-eec-LATEX "2023-1-C2-macaco")
% (find-pdf-page   "~/LATEX/2023-1-C2-macaco.pdf")
% (find-sh0 "cp -v  ~/LATEX/2023-1-C2-macaco.pdf /tmp/")
% (find-sh0 "cp -v  ~/LATEX/2023-1-C2-macaco.pdf /tmp/pen/")
%     (find-xournalpp "/tmp/2023-1-C2-macaco.pdf")
%   file:///home/edrx/LATEX/2023-1-C2-macaco.pdf
%               file:///tmp/2023-1-C2-macaco.pdf
%           file:///tmp/pen/2023-1-C2-macaco.pdf
%  http://anggtwu.net/LATEX/2023-1-C2-macaco.pdf
% (find-LATEX "2019.mk")
% (find-MM-aula-links "2023-1-C2-macaco" "C2" "c2m231macaco" "c2ma")
% (find-anggfile "LUA/ELpeg1.lua" "cp")

% «.defs»		(to "defs")
% «.title»		(to "title")
% «.links»		(to "links")
% «.banana»		(to "banana")
% «.contas-formais»	(to "contas-formais")
% «.EDOs-RC-TFC2»	(to "EDOs-RC-TFC2")



% <videos>
% Video (not yet):
% (find-ssr-links     "c2m231macaco" "2023-1-C2-macaco")
% (code-eevvideo      "c2m231macaco" "2023-1-C2-macaco")
% (code-eevlinksvideo "c2m231macaco" "2023-1-C2-macaco")
% (find-c2m231macacovideo "0:00")

\documentclass[oneside,12pt]{article}
\usepackage[colorlinks,citecolor=DarkRed,urlcolor=DarkRed]{hyperref} % (find-es "tex" "hyperref")
\usepackage{amsmath}
\usepackage{amsfonts}
\usepackage{amssymb}
\usepackage{pict2e}
\usepackage[x11names,svgnames]{xcolor} % (find-es "tex" "xcolor")
\usepackage{colorweb}                  % (find-es "tex" "colorweb")
%\usepackage{tikz}
%
% (find-dn6 "preamble6.lua" "preamble0")
%\usepackage{proof}   % For derivation trees ("%:" lines)
%\input diagxy        % For 2D diagrams ("%D" lines)
%\xyoption{curve}     % For the ".curve=" feature in 2D diagrams
%
\usepackage{edrx21}               % (find-LATEX "edrx21.sty")
\input edrxaccents.tex            % (find-LATEX "edrxaccents.tex")
\input edrx21chars.tex            % (find-LATEX "edrx21chars.tex")
\input edrxheadfoot.tex           % (find-LATEX "edrxheadfoot.tex")
\input edrxgac2.tex               % (find-LATEX "edrxgac2.tex")
%\usepackage{emaxima}              % (find-LATEX "emaxima.sty")
%
% (find-es "tex" "geometry")
\usepackage[a6paper, landscape,
            top=1.5cm, bottom=.25cm, left=1cm, right=1cm, includefoot
           ]{geometry}
%
\begin{document}

\catcode`\^^J=10
\directlua{dofile "dednat6load.lua"}  % (find-LATEX "dednat6load.lua")

% «defs»  (to ".defs")
% (find-LATEX "edrx21defs.tex" "colors")
% (find-LATEX "edrx21.sty")
\def\drafturl{http://anggtwu.net/LATEX/2023-1-C2.pdf}
\def\drafturl{http://anggtwu.net/2023.1-C2.html}
\def\draftfooter{\tiny \href{\drafturl}{\jobname{}} \ColorBrown{\shorttoday{} \hours}}

% (find-LATEX "2023-1-C2-carro.tex" "defs-caepro")
% (find-LATEX "2023-1-C2-carro.tex" "defs-pict2e")
%L dofile "Caepro5.lua"              -- (find-angg "LUA/Caepro5.lua" "LaTeX")
\pu
\def\Caurl   #1{\expr{Caurl("#1")}}
\def\Cahref#1#2{\href{\Caurl{#1}}{#2}}
\def\Ca      #1{\Cahref{#1}{#1}}

% (find-angg "LUA/ELpeg-cme1.lua" "Freeze")
%L require "ELpeg-cme1"              -- (find-angg "LUA/ELpeg-cme1.lua")
%L parse_cme = Parser.freeze("expr")
\pu
\def\cme#1{\expr{parse_cme("#1"):totex()}}
\def\CME#1{\expr{CME      ("#1"):totex()}}
% $\cme{ .(f(g(x))) s .[ f(y):=e^y ;; g(x):=4x] }$

% (find-LATEX "edrxgac2.tex" "C2-substnames")
% 2dT13 EDOs por chutar e testar
% (c2m212introp 12 "EDOs-chutar-testar")
% (c2m212introa    "EDOs-chutar-testar")
\def\redname#1{{\color{Red3}\text{#1}}}
\sa  {RC}{\redname{[RC]}}
\sa{TFC2}{\redname{[TFC2]}}
\sa   {4}{\redname{[4]}}
\sa   {5}{\redname{[5]}}
\sa   {6}{\redname{[6]}}
\sa   {7}{\redname{[7]}}
\sa   {8}{\redname{[8]}}
\sa  {II}{\redname{[II]}}

% \ga{1}
% \ga{RC}
% \GenericWarning{Success:}{Success!!!}  % Used by `M-x cv'
% \end{document}




%  _____ _ _   _                               
% |_   _(_) |_| | ___   _ __   __ _  __ _  ___ 
%   | | | | __| |/ _ \ | '_ \ / _` |/ _` |/ _ \
%   | | | | |_| |  __/ | |_) | (_| | (_| |  __/
%   |_| |_|\__|_|\___| | .__/ \__,_|\__, |\___|
%                      |_|          |___/      
%
% «title»  (to ".title")
% (c2m231macacop 1 "title")
% (c2m231macacoa   "title")

\thispagestyle{empty}

\begin{center}

\vspace*{1.2cm}

{\bf \Large Cálculo 2 - 2023.1}

\bsk

Aulas 4 até 9: o macaco integrador

e integração por partes


\bsk

Eduardo Ochs - RCN/PURO/UFF

\url{http://anggtwu.net/2023.1-C2.html}

\end{center}

\newpage

% «links»  (to ".links")
% (c2m231macacop 2 "links")
% (c2m231macacoa   "links")

{\bf Links}


\scalebox{0.9}{\def\colwidth{13cm}\firstcol{

\par \Ca{2fT2} PDF de 2022.2 sobre o `$[:=]$'
\par \Ca{2dT8} PDF de 2021.2 sobre o `$[:=]$' e justificativas
\par \Ca{Leit2p15} (p.68): Dois exemplos de contas com justificativas

\msk

\par \Ca{CalcEasy14:08} até 18:18: como o macaco deriva funções elementares
\par \Ca{2fQ1} Quadros de 2022.2 sobre árvores

\msk

\par \Ca{2fT23} Outra definição para integral indefinida
\par \Ca{2fT26} Integração por partes em 2022.2: pedaços do quadro
\par \Ca{2fT30} Exercícios pra casa

\msk

\par \Ca{2gQ5} Quadros da aula 4 (14/abr/2023)
\par \Ca{2gQ7} Quadros da aula 5 (18/abr/2023)
\par \Ca{2gQ13} Quadros da aula 7 (25/abr/2023)
\par \Ca{2gQ16} Quadros da aula 8 (28/abr/2023)
\par \Ca{2gQ18} Quadros da aula 9 (02/mai/2023)

}\anothercol{
}}




\newpage

%  ____                                
% | __ )  __ _ _ __   __ _ _ __   __ _ 
% |  _ \ / _` | '_ \ / _` | '_ \ / _` |
% | |_) | (_| | | | | (_| | | | | (_| |
% |____/ \__,_|_| |_|\__,_|_| |_|\__,_|
%                                      
% «banana»  (to ".banana")
% (c2m231macacop 3 "banana")
% (c2m231macacoa   "banana")

% $\cme{ .(f(g(x))) s .[ f(y):=e^y ;; g(x):=4x] }$

\def\t#1{\text{#1}}
\def\atoo{[\t{a}:=\t{o}]}


{\bf O macaco substituidor: banana}

\scalebox{0.5}{\def\colwidth{10.5cm}\firstcol{

Os quadros da aula de 25/abril/2023 estão aqui:
%
$$\text{\Ca{2gQ13}}$$

O objetivo deste slide é fazer você entender muito bem que a operação
de {\sl substituição}, `$[:=]$', é uma operação MUITO diferente da
{\sl igualdade}, `$=$'. Você provavelmente já sabe lidar bastante bem
com igualdades. Leia este slide aqui várias vezes,
%
$$\text{\Ca{2gT13} Banana}$$

até você entender bastante bem como o programa de substituir letras
descrito nele funciona. Digamos que a notação abaixo
%
$$\t{banona} \, \atoo = \t{bonono}$$

Seja uma {\sl abreviação} pra isto aqui:

\begin{quote}
  o resultado de substituir todas as letras `a' por `o' na palavra
  ``banona'' é a palavra ``bonono''.
\end{quote}

\msk

{\bf Dica pro exercício}

Lembre que em C {\tt "123"} é um string de comprimento 3 e {\tt 123} é
um número; {\tt 100+23} é igual a {\tt 123} mas {\tt "100+23"} não é
igual a {\tt "123"}.

}\anothercol{

Em Cálculo 2 a gente vai ter algo parecido com essa distinção entre
strings e números -- a gente vai distinguir entre {\sl expressões} e
{\sl resultados de expressões}. Por exemplo, 2+5, 3+4 e 7 são três
expressões diferentes mas com o mesmo resultado... e em alguns itens
do exercício abaixo você vai ter que considerar que, por exemplo,
bana+na é uma palavra com 7 caracteres, banana é uma palavra com 6
caracteres, e que em nenhum momento o exercício pede pra você tratar
bana+na como uma expressão cujo resultado vai ser banana se você
definir o `+' como concatenação.

\bsk

{\bf Exercício}

Quais das afirmações abaixo são verdadeiras?

$$\begin{tabular}{rl}
  a) & $\t{banana} \, \atoo = \t{bonono}$ \\
  b) & $\t{bonono} \, \atoo = \t{banana}$ \\
  c) & $\t{banana} \, \atoo = \t{bono}$ \\
  d) & $\t{banana} \, \atoo = \t{bonono}$ \\
  e) & $\t{banana} \, \atoo = \t{bono+no}$ \\
  f) & $\t{banana} \, \atoo = \t{bononono}$ \\
  g) & $\t{bonono} \, \atoo = \t{banona}$ \\
  h) & $\t{banona} \, \atoo = \t{bonono}$ \\
  \end{tabular}
$$

\bsk

{\bf Péssima notícia}

Em Cálculo 2 a gente vai trabalhar o tempo todo com expressões que são
transformadas passo a passo. {\sl Talvez isto seja algo completamente
  novo pra você.}

}}



\newpage

%   ____            _               __                            _     
%  / ___|___  _ __ | |_ __ _ ___   / _| ___  _ __ _ __ ___   __ _(_)___ 
% | |   / _ \| '_ \| __/ _` / __| | |_ / _ \| '__| '_ ` _ \ / _` | / __|
% | |__| (_) | | | | || (_| \__ \ |  _| (_) | |  | | | | | | (_| | \__ \
%  \____\___/|_| |_|\__\__,_|___/ |_|  \___/|_|  |_| |_| |_|\__,_|_|___/
%                                                                       
% «contas-formais»  (to ".contas-formais")
% (c2m231macacop 4 "contas-formais")
% (c2m231macacoa   "contas-formais")
{\bf O macaco e as contas formais}

\scalebox{0.5}{\def\colwidth{11cm}\firstcol{

Na aula de 25/abril nós passamos muito tempo revendo coisas que
deveriam ser básicas -- já vou dizer quais -- e eu passei um dever de
casa bem grande: {\sl leia o que você conseguir das seções do Miranda
  e do Leithold sobre a regra da cadeia e faça todos os exercícios que
  você puder.} Aqui tem links pra elas:

\msk

\Ca{Miranda87} Seção 3.5: regra da cadeia

\Ca{Miranda228} Seção 7.5.1: TFC2

\Ca{Leit3p45} (p.181) Seção 3.6: regra da cadeia

\Ca{Leit5p61} (p.344) Seção 5.8: Os teoremas fundamentais do Cálculo

\msk

Lembre que: 1) um dos objetivos do curso é fazer vocês se tornarem
capazes de estudar pelos livros, 2) as provas vão ter várias questões
que vocês só vão conseguir fazer se vocês tiverem muita prática de
fazer contas, e 3) o livro do Leithold é difícil em alguns lugares mas
ele é INCRIVELMENTE bom -- estudem por ele sempre que puderem!

\msk

Outra coisa: dê uma olhada na seção do Miranda sobre a regra da cadeia
-- você vai ver que essa fórmula tem uma demonstração, e que a fórmula
e a demonstração só funcionam quando certas hipóteses são obedecidas.
Aliás, uma questão da P1 do semestre passado foi sobre situações em
que a fórmula do TFC2 dá resultados errados. Dê uma olhada nela:

\msk

\Ca{2fT110} A fórmula do TFC2 nem sempre vale

\msk

{\sl A P1 deste semestre vai ter uma questão parecida com essa.}


}\anothercol{

{}

Em algumas situações nós vamos primeiro aplicar a fórmula como se ela
valesse sempre, e só depois que nós fizermos todas as contas nós vamos
descobrir quais são as hipóteses necessárias pra aquelas contas
valerem. O nome ``oficial'' pra essas contas sem a verificação das
hipóteses é ``contas formais'', mas eu vou usar a terminologia do
Mathologer... ele fala muito no macaco que faz contas automaticamente
sem fazer a menor idéia do que aquelas contas querem dizer, então eu
vou usar expressões como ``aqui vamos fazer contas como o macaco''.

\bsk

{\bf Exercício}

Use o que você lembra de Cálculo 1 pra obter boas fórmulas pras
derivadas abaixo:
%
$$\begin{tabular}{rl}
  a) & $\ddx \, e^{g(x)}$ \\
  b) & $\ddx \, g(x)^{1/2}$ \\
  c) & $\ddx \, \sqrt{g(x)}$ \\
  c) & $\ddx \, f(4x)$ \\
  \end{tabular}
$$


No próximo slide nós vamos ver como o macaco faz esssas contas usando
a operação `$[:=]$'.


}}

\newpage

%  _____ ____   ___          ____   ____     _____ _____ ____ ____  
% | ____|  _ \ / _ \ ___    |  _ \ / ___|   |_   _|  ___/ ___|___ \ 
% |  _| | | | | | | / __|   | |_) | |         | | | |_ | |     __) |
% | |___| |_| | |_| \__ \_  |  _ <| |___ _    | | |  _|| |___ / __/ 
% |_____|____/ \___/|___( ) |_| \_\\____( )   |_| |_|   \____|_____|
%                       |/              |/                          
%
% «EDOs-RC-TFC2»  (to ".EDOs-RC-TFC2")
% (c2m231macacop 5 "EDOs-RC-TFC2")
% (c2m231macacoa   "EDOs-RC-TFC2")

\def\bigeq#1{\Bigl(#1\Bigr)}
\def\bigeq#1{\Bigl(#1\Bigr)}

{\bf O macaco substituidor: EDOs, RC, TFC2}

\scalebox{0.55}{\def\colwidth{14cm}\firstcol{

Sejam:
%
$$\begin{array}{rcl}
  \ga{4}    &=& \bigeq{ f'(x) = x^4               } \\
  \ga{5}    &=& \bigeq{ f'(x) = 2f(x)             } \\
  \ga{6}    &=& \bigeq{ f''(x) + f'(x) = 6f(x)    } \\
  \ga{7}    &=& \bigeq{ f'(x) = -\frac{1}{f(x)}   } \\
  \ga{8}    &=& \bigeq{ f'(x) = -\frac{x}{f(x)}   } \\
  \ga{RC}   &=& \bigeq{ f(g(x))' = f'(g(x)) g'(x) } \\
  \ga{TFC2} &=& \bigeq{ \Intx{a}{b}{f'(x)} = f(b) - f(a) } \\
  \end{array}
$$

Note que as expressões $\ga{4}$, $\ga{5}$, $\ga{6}$, $\ga{7}$,
$\ga{8}$, são as EDOs deste problema aqui:

\ssk

\Ca{2dT13} EDOs por chutar e testar.

\bsk

{\bf Exercício}

Calcule o resultado de cada uma das substituições à direita. Lembre
que o resultado de uma substituição é sempre uma {\sl expressão} --
{\sl não simplifique ela}. Deixa eu fazer uma comparação com C: o
resultado de substituir cada ocorrência do caracter {\tt 'a'} pelo
caracter {\tt '2'} no string {\tt "a+5"} é o string {\tt "2+5"}, não o
string {\tt "7"}, e nem o número {\tt 7}.

}\anothercol{



$\begin{tabular}[t]{rl}
  a) & $f(g(x)) \CME{[x := 42]}$ \\
  b) & $f(g(x)) \CME{[g(x) := 200*x]}$ \\
  c) & $f(g(x)) \CME{[f(y) := y^2+y^3]}$ \\
  d) & $f(g(x)) \CME{[f(y) := e^{y}]}$ \\
  e) & $f(g(x)) \CME{[g(x) := 4*x]}$ \\
  f) & $f(g(x)) \CME{.[f(y) := e^{y} ;; g(x) := 4*x]}$ \\
  g) & $f(g(x)) \CME{[f(y) := y^{1/2}]}$ \\
  h) & $f(g(x)) \CME{[f(y) := sqrt{y}]}$ \\
  i) & $f(g(x)) \CME{[f(y) := sqrt{y}]}$ \\
  j) & $f(g(x)) \CME{.[g(x) := e^x ;; g'(x) := e^x]}$ \\
  k) & $f(g(x))' \CME{.[g(x) := e^x ;; g'(x) := e^x]}$ \\
  l) & $\ga{RC}  \CME{.[f(y) := e^y ;; f'(y) := e^y]}$ \\
  m) & $\ga{RC}  \CME{.[f(y) := y^{1/2} ;; f'(y) := {1//2} mul y^{-1/2}]}$ \\
  n) & $\ga{RC}  \CME{.[f(y) := sqrt{y} ;; f'(y) := 1 // 2 mul sqrt{y}]}$ \\
  o) & $\ga{6}   \CME{.[f  (x) :=       e^{2 mul x} ;;
                        f' (x) := 2 mul e^{2 mul x} ;;
                        f''(x) := 4 mul e^{2 mul x}]}$ \\
  p) & $\ga{6}   \CME{.[f  (x) :=       e^{3 mul x} ;;
                        f' (x) := 3 mul e^{3 mul x} ;;
                        f''(x) := 9 mul e^{3 mul x}]}$ \\
  q) & $\ga{TFC2}\CME{.[f  (x) := {1//2} mul x^2 ;;
                        f' (x) := x ;;
                        a      := 0 ;;
                        b      := 2 ]}$ \\
  r) & $\ga{8}   \CME{.[f  (x) := sqrt{1 - x^2} ;;
                        f' (x) := -1 // x mul sqrt{1 - x^2}]}$ \\
  \end{tabular}
$

}}

\newpage

{\bf Diferença}

\scalebox{0.6}{\def\colwidth{9cm}\firstcol{

Lembre que esta notação aqui
%
$$\Intx{a}{b}{f(x)}$$

tem várias pronúncias:

``a integral da função $f(x)$ entre $x=a$ e $x=b$'',

``a área sob a curva $f(x)$ entre $x=a$ e $x=b$'',

``a área sob a curva $f(x)$ desde $x=a$ até $x=b$'',

etc...

\msk

A pronúncia desta operação daqui
%
$$\Difx{a}{b}{f(x)}$$

vai ser ``a diferença da $f(x)$ entre $x=a$ e $x=b$'',

e a definição formal dela vai ser esta:
%
$$\Difx{a}{b}{f(x)} = f(b)-f(a)$$


}\anothercol{

{\bf Exercício}

O Leithold e o Miranda usam notações ligeiramente diferentes da minha
para a operação diferença. Dê uma olhada nestas páginas aqui,

\msk

\Ca{Leit5p65} p.348

\Ca{Miranda344}

\msk

e traduza a expressão

$$\difx{3}{4}{(\sin 2x)}$$

da minha notação para

\msk

a) a notação do Miranda,

b) a notação do Leithold.

}}

% (find-leitholdptpage (+ 17 344) "5.8. Os Teoremas Fundamentais do Cálculo")
% (find-leitholdptpage (+ 17 348)   "notação para diferença")
% (find-dmirandacalcpage 228     "notação para diferença")

\newpage

%  ___       _     _           _ 
% |_ _|_ __ | |_  (_)_ __   __| |
%  | || '_ \| __| | | '_ \ / _` |
%  | || | | | |_  | | | | | (_| |
% |___|_| |_|\__| |_|_| |_|\__,_|
%                                

{\bf Integral indefinida}

\scalebox{0.55}{\def\colwidth{10cm}\firstcol{

Tanto o Leithold quanto o Miranda explicam a {\sl integral indefinida}
antes da {\sl integral definida}. Dê uma olhada:

\msk

\Ca{Miranda181} 6. Integral Indefinida

\Ca{Miranda207} 7. Integração definida

\Ca{Leit5p3} (p.286) 5.1. Antidiferenciação

\Ca{Leit5p41} (p.324) 5.5. A integral definida

\msk

{\sl Todos os modos fáceis de atribuir um significado intuitivo para
  expressões como esta aqui}
%
$$\intx{f(x)}$$

{\sl são gambiarras que funcionam mal.}

\msk

Eu vou usar esta definição aqui,

\ssk

\Ca{2fT23} (p.4) Outra definição para a integral indefinida

\ssk

e aqui tem um caso em que a definição usual quebra:

\ssk

\Ca{2fT24} (p.5) Meme: expanding brain, versão ln

\msk

}\anothercol{

Nós vamos começar usando a integral indefinida como o macaco que faz
contas sem ter idéia do significado do que está fazendo, e só depois
que tivermos bastante prática nós vamos discutir os vários jeitos de
atribuir significados intuitivos para % $\intx{f(x)}$.

\msk

A regra básica vai ser esta aqui:

$$\ga{II} = \left( \intx{f'(x)} = f(x) \right)$$

\bsk

{\bf Exercícios}

\msk

Calcule:

\ssk

a) $\ga{II} \CME{.[f(x) := x+42 ;; f'(x) := 1]}$

b) $\ga{II} \CME{.[f(x) := {1//2} mul x^2 ;; f'(x) := x]}$

\msk

c) Resolva os exercícios 1 a 10 daqui por chutar e testar:

\Ca{Miranda185} Exercícios 6.1

\msk

d) Entenda tudo que esta nesta página:

\Ca{Leit5p6} (p.289) 5.1.8. Teorema

}}

\newpage

{\bf Integração por partes}

\scalebox{0.9}{\def\colwidth{14cm}\firstcol{

Vou usar isto, de 2022.2:

\Ca{2fT25} (p.6) Pedaços do quadro

% (find-books "__analysis/__analysis.el" "leithold")
% (find-books "__analysis/__analysis.el" "leithold" "9.1. Integração por partes")
% (find-books "__analysis/__analysis.el" "miranda")
% (find-books "__analysis/__analysis.el" "miranda" "por Partes")
% (find-dmirandacalcpage 182 "6.1.1 Regras Básicas de Integração")
% (find-dmirandacalcpage 199 "6.3 Integração por Partes")
% (find-leitholdptpage (+ 17 286) "5.1. Antidiferenciação")
% (find-leitholdptpage (+ 17 294)   "Exercícios 5.1")
% (find-leitholdptpage (+ 17 531) "9.1. Integração por partes")

\msk

E:

\par \Ca{Leit9} 9. Técnicas de integração
\par \Ca{Leit9p4} (p.531) 9.1. Integração por partes
\par \Ca{Miranda182} 6.1.1 Regras Básicas de Integração
\par \Ca{Miranda199} 6.3 Integração por partes

\msk

Ainda não \LaTeX ei as contas desta aula!

Mas os quadros dela -- os sobre integração por partes --

estão aqui: \Ca{2gQ20}.

\msk



}\anothercol{
}}



% (find-es "tex" "big-delimiters")

% Se nós assumirmos que a=o então todas estas palavras ``são iguais'':
% 
% \begin{quote}
%   banana, banano, banona, banono, \\
%   bonana, bonano, bonona, bonono
% \end{quote}





\GenericWarning{Success:}{Success!!!}  % Used by `M-x cv'

\end{document}



% Local Variables:
% coding: utf-8-unix
% ee-tla: "c2ma"
% ee-tla: "c2m231macaco"
% End:
