% (find-LATEX "2023-1-ES-P1.tex")
% (defun c () (interactive) (find-LATEXsh "lualatex -record 2023-1-ES-P1.tex" :end))
% (defun C () (interactive) (find-LATEXsh "lualatex 2023-1-ES-P1.tex" "Success!!!"))
% (defun D () (interactive) (find-pdf-page      "~/LATEX/2023-1-ES-P1.pdf"))
% (defun d () (interactive) (find-pdftools-page "~/LATEX/2023-1-ES-P1.pdf"))
% (defun e () (interactive) (find-LATEX "2023-1-ES-P1.tex"))
% (defun o () (interactive) (find-LATEX "2023-1-ES-P1.tex"))
% (defun u () (interactive) (find-latex-upload-links "2023-1-ES-P1"))
% (defun v () (interactive) (find-2a '(e) '(d)))
% (defun d0 () (interactive) (find-ebuffer "2023-1-ES-P1.pdf"))
% (defun cv () (interactive) (C) (ee-kill-this-buffer) (v) (g))
%          (code-eec-LATEX "2023-1-ES-P1")
% (find-pdf-page   "~/LATEX/2023-1-ES-P1.pdf")
% (find-sh0 "cp -v  ~/LATEX/2023-1-ES-P1.pdf /tmp/")
% (find-sh0 "cp -v  ~/LATEX/2023-1-ES-P1.pdf /tmp/pen/")
%     (find-xournalpp "/tmp/2023-1-ES-P1.pdf")
%   file:///home/edrx/LATEX/2023-1-ES-P1.pdf
%               file:///tmp/2023-1-ES-P1.pdf
%           file:///tmp/pen/2023-1-ES-P1.pdf
%  http://anggtwu.net/LATEX/2023-1-ES-P1.pdf
% (find-LATEX "2019.mk")
% (find-Deps1-links "Caepro5 Estatistica1")
% (find-Deps1-cps   "Caepro5 Estatistica1")
% (find-Deps1-anggs "Caepro5 Estatistica1")
% (find-MM-aula-links "2023-1-ES-P1" "ES" "esm231p1" "esp1")

% «.defs»			(to "defs")
% «.defs-T-and-B»		(to "defs-T-and-B")
% «.defs-dots»			(to "defs-dots")
% «.defs-pacocas»		(to "defs-pacocas")
% «.defs-balas»			(to "defs-balas")
% «.title»			(to "title")
% «.questao-1»			(to "questao-1")
% «.questao-2»			(to "questao-2")
% «.questao-3»			(to "questao-3")
% «.pacocas»			(to "pacocas")
% «.dicas-medias»		(to "dicas-medias")
% «.dicas-dm-e-var»		(to "dicas-dm-e-var")
% «.dicas-probabilidade»	(to "dicas-probabilidade")
% «.dicas-conjuntos»		(to "dicas-conjuntos")
% «.dicas-prob-cond»		(to "dicas-prob-cond")
% «.questao-1-gab»		(to "questao-1-gab")
% «.questao-2-gab»		(to "questao-2-gab")
% «.links»			(to "links")
%
% «.djvuize»		(to "djvuize")



\documentclass[oneside,12pt]{article}
\usepackage[colorlinks,citecolor=DarkRed,urlcolor=DarkRed]{hyperref} % (find-es "tex" "hyperref")
\usepackage{amsmath}
\usepackage{amsfonts}
\usepackage{amssymb}
\usepackage{pict2e}
\usepackage[x11names,svgnames]{xcolor} % (find-es "tex" "xcolor")
\usepackage{colorweb}                  % (find-es "tex" "colorweb")
%\usepackage{tikz}
%
% (find-dn6 "preamble6.lua" "preamble0")
%\usepackage{proof}   % For derivation trees ("%:" lines)
%\input diagxy        % For 2D diagrams ("%D" lines)
%\xyoption{curve}     % For the ".curve=" feature in 2D diagrams
%
\usepackage{edrx21}               % (find-LATEX "edrx21.sty")
\input edrxaccents.tex            % (find-LATEX "edrxaccents.tex")
\input edrx21chars.tex            % (find-LATEX "edrx21chars.tex")
\input edrxheadfoot.tex           % (find-LATEX "edrxheadfoot.tex")
\input edrxgac2.tex               % (find-LATEX "edrxgac2.tex")
%\usepackage{emaxima}              % (find-LATEX "emaxima.sty")
%\input 2022pict2e.tex             % (find-fline "~/LATEX/2022pict2e.tex")
%
% (find-es "tex" "geometry")
\usepackage[a6paper, landscape,
            top=1.5cm, bottom=.25cm, left=1cm, right=1cm, includefoot
           ]{geometry}
%
\begin{document}

% «defs»  (to ".defs")
% (find-LATEX "edrx21defs.tex" "colors")
% (find-LATEX "edrx21.sty")

\def\drafturl{http://anggtwu.net/LATEX/2023-1-ES.pdf}
\def\drafturl{http://anggtwu.net/2023.1-ES.html}
\def\draftfooter{\tiny \href{\drafturl}{\jobname{}} \ColorBrown{\shorttoday{} \hours}}

% «defs-T-and-B»  (to ".defs-T-and-B")
\long\def\ColorOrange#1{{\color{orange!90!black}#1}}
\def\T(Total: #1 pts){{\bf(Total: #1)}}
\def\T(Total: #1 pts){{\bf(Total: #1 pts)}}
\def\T(Total: #1 pts){\ColorRed{\bf(Total: #1 pts)}}
\def\B       (#1 pts){\ColorOrange{\bf(#1 pts)}}

\def\Media{\textsf{Média}}
\def\DM   {\textsf{dm}}
\def\Var  {\textsf{var}}
\def\nome {\text  {nome}}
\def\P#1{\left(#1\right)}

\catcode`\^^J=10
\directlua{dofile "dednat6load.lua"}  % (find-LATEX "dednat6load.lua")

% (find-LATEX "2023-1-C2-carro.tex" "defs-caepro")
% (find-LATEX "2023-1-C2-carro.tex" "defs-pict2e")

% «defs-caepro»  (to ".defs-caepro")
%L dofile "Caepro5.lua"              -- (find-angg "LUA/Caepro5.lua" "LaTeX")
%L dofile "Estatistica1.lua"         -- (find-angg "LUA/Estatistica1.lua")
\pu
\def\Caurl   #1{\expr{Caurl("#1")}}
\def\Cahref#1#2{\href{\Caurl{#1}}{#2}}
\def\Ca      #1{\Cahref{#1}{#1}}

\def\pictgridstyle{\color{GrayPale}\linethickness{0.3pt}}
\def\pictaxesstyle{\linethickness{0.5pt}}
\def\pictnaxesstyle{\color{GrayPale}\linethickness{0.5pt}}
\celllower=2.5pt
\celllower=4.5pt
\unitlength=10pt

% «defs-dots»  (to ".defs-dots")
\def\pdots       #1{\directlua{pdots "#1":pgat("patc"):output()}}
\def\und       #1#2{\underbrace{#1}_{#2}}
\def\unddots   #1#2{\underbrace{#1}_{\pdots{#2}}}
\def\setdotdims#1#2{
  \def\closeddot{\circle*{#1}}
  \def\opendot  {\circle*{#1}\color{white}\circle*{#2}}}

\setdotdims{0.6}{0.5}

% «defs-pacocas»  (to ".defs-pacocas")
% (find-angg "LUA/Estatistica1.lua" "pacocas-test")
%L
%L Pict.__index.enslower = -0.25
%L Pict.__index.enslower = -0.4
%L
%L -- Exemplo das paçocas:
%L PictBounds.setbounds(v(0,0), v(6,2))
%L spec1 = "1:A 1:B 4:C 5:D"
%L spec2 = "1:A 2:B 3:C 5:D"
%L p1    = SqP.from(spec1):topict():scalebox(0.6):sa("pacocas 1")
%L p2    = SqP.from(spec2):topict():scalebox(0.6):sa("pacocas 2")
%L Pict { p1, p2 } :output()
%L
%L -- Histogramas com numerozinhos pra desvio médio e variância:
%L PictBounds.setbounds(v(0,0), v(6,1))
%L specdm  = "1:2 2:1 3:0 4:1 5:2"
%L specvar = "1:4 2:1 3:0 4:1 5:4"
%L p1      = SqP.from(specdm ):topict():scalebox(0.6):sa("dm")
%L p2      = SqP.from(specvar):topict():scalebox(0.6):sa("var")
%L Pict { p1, p2 } :output()
%L
%L -- Questão 1:
%L PictBounds.setbounds(v(0,0), v(6,3))
%L def = function (sa, spec) return SqP.from(spec):topict():scalebox(0.6):sa(sa):output() end
%L def("Q1a", "1 2 2 2 3");   def("Q1b", "2 3 3 3 4")
%L def("Q1c", "1 1 2 3 3");   def("Q1d", "2 2 3 4 4")
%L def("Q1e", "0 1 2 3 4");   def("Q1f", "1 2 3 4 5")
%L def("Q1g", "0 0 2 4 4");   def("Q1h", "1 1 3 5 5")
%L def("Q1i", "0 0 3 3 4");   def("Q1j", "1 1 4 4 5")
%L
%L def("Q1a_DM",  "1:1 2:0 2:0 2:0 3:1");   def("Q1b_DM",  "2:1 3:0 3:0 3:0 4:1")
%L def("Q1c_DM",  "1:1 1:1 2:0 3:1 3:1");   def("Q1d_DM",  "2:1 2:1 3:0 4:1 4:1")
%L def("Q1e_DM",  "0:2 1:1 2:0 3:1 4:2");   def("Q1f_DM",  "1:2 2:1 3:0 4:1 5:2")
%L def("Q1g_DM",  "0:2 0:2 2:0 4:2 4:2");   def("Q1h_DM",  "1:2 1:2 3:0 5:2 5:2")
%L def("Q1i_DM",  "0:2 0:2 3:1 3:1 4:2");   def("Q1j_DM",  "1:2 1:2 4:1 4:1 5:2")
%L
%L def("Q1a_Var", "1:1 2:0 2:0 2:0 3:1");   def("Q1b_Var", "2:1 3:0 3:0 3:0 4:1")
%L def("Q1c_Var", "1:1 1:1 2:0 3:1 3:1");   def("Q1d_Var", "2:1 2:1 3:0 4:1 4:1")
%L def("Q1e_Var", "0:4 1:1 2:0 3:1 4:4");   def("Q1f_Var", "1:4 2:1 3:0 4:1 5:4")
%L def("Q1g_Var", "0:4 0:4 2:0 4:4 4:4");   def("Q1h_Var", "1:4 1:4 3:0 5:4 5:4")
%L def("Q1i_Var", "0:4 0:4 3:1 3:1 4:4");   def("Q1j_Var", "1:4 1:4 4:1 4:1 5:4")

\pu

% «defs-balas»  (to ".defs-balas")

%L PictBounds.setbounds(v(0,0), v(5,4))
%L spec = [[
%L   1,3,A  2,3,B  3,3,C
%L   1,2,D  2,2,E  3,2,F
%L   1,1,G  2,1,H  3,1,I  4,1,J
%L ]]
%L p = Pict {}
%L for x,y,name in spec:gmatch("(.),(.),(.)") do
%L   print(x,y,name)
%L   p:addcloseddotat(v(x+0, y+0))
%L   p:puttcellat(v(x+0.3, y+0.3), name)
%L end
%L p:pgat("pat"):sa("balas0"):output()
\pu

\sa{balas}{
  \unitlength=20pt
  \setdotdims{0.25}{0.2}
  \scalebox{0.7}{$
    \ga{balas0}
  $}}


%L defminibalas = function (sa, spec)
%L     local p = Pict {}
%L     for x,y,name in spec:gmatch("%s(%d),(%d),?(%S*)") do
%L       print(x,y,name)
%L       p:addcloseddotat(v(x+0, y+0))
%L       p:puttcellat(v(x+0.3, y+0.3), name)
%L     end
%L     p:pgat("patc"):sa(sa):output()
%L   end
%L
%L defminibalas("X<=2", [[
%L    1,3  2,3  _3,3
%L    1,2  2,2  _3,2
%L    1,1  2,1  _3,1  _4,1
%L ]])
%L defminibalas("X>=2", [[
%L   _1,3  2,3  3,3
%L   _1,2  2,2  3,2
%L   _1,1  2,1  3,1  4,1
%L ]])
%L defminibalas("Y<=2", [[
%L   _1,3  _2,3  _3,3
%L    1,2   2,2   3,2
%L    1,1   2,1   3,1   4,1
%L ]])
%L defminibalas("Y>=2", [[
%L    1,3   2,3   3,3
%L    1,2   2,2   3,2
%L   _1,1  _2,1  _3,1  _4,1
%L ]])
%L defminibalas("Z>=0", [[
%L   _1,3  _2,3   3,3
%L   _1,2   2,2   3,2
%L    1,1   2,1   3,1   4,1
%L ]])
%L defminibalas("Z>=1", [[
%L   _1,3  _2,3  _3,3
%L   _1,2  _2,2   3,2
%L   _1,1   2,1   3,1   4,1
%L ]])
%L defminibalas("X<=2 and Z>=0", [[
%L   _1,3 _2,3  _3,3
%L   _1,2  2,2  _3,2
%L    1,1  2,1  _3,1  _4,1
%L ]])
%L defminibalas("Z>=0 and X>=2", [[
%L   _1,3 _2,3   3,3
%L   _1,2  2,2   3,2
%L   _1,1  2,1   3,1   4,1
%L ]])
\pu

\def\minibalas#1{{
  \unitlength=5pt
  \setdotdims{0.5}{0.2}
  \scalebox{0.7}{$
    \ga{#1}
  $}}}
\def\MBA#1#2#3{#1 \; ⇒ \; \minibalas{#2}}
\def\MBB#1#2#3{P(#1) = #3}
\def\MBC#1#2#3{\begin{array}{rcl}
   #1\phantom{i} &⇒& \minibalas{#2} \\[-2pt]
  P(#1) &=& #3 \\
  \end{array}}



%  _____ _ _   _                               
% |_   _(_) |_| | ___   _ __   __ _  __ _  ___ 
%   | | | | __| |/ _ \ | '_ \ / _` |/ _` |/ _ \
%   | | | | |_| |  __/ | |_) | (_| | (_| |  __/
%   |_| |_|\__|_|\___| | .__/ \__,_|\__, |\___|
%                      |_|          |___/      
%
% «title»  (to ".title")
% (esm231p1p 1 "title")
% (esm231p1a   "title")

\thispagestyle{empty}

\begin{center}

\vspace*{1.2cm}

{\bf \Large Estatística - 2023.1}

\bsk

Primeira prova (P1)

\bsk

Eduardo Ochs - RCN/PURO/UFF

\url{http://anggtwu.net/2023.1-ES.html}

\end{center}

\newpage

\scalebox{0.55}{\def\colwidth{9cm}\firstcol{

% «questao-1»  (to ".questao-1")
% (esm231p1p 2 "questao-1")
% (esm231p1a   "questao-1")

{\bf Questão 1}

\T(Total: 2.0 pts)

\msk

Calcule a média, o desvio médio e a variância de cada uma das
distribuições abaixo.

\unitlength=13pt

$$\def\Item#1{ \text{#1)}\;\ga{Q1#1} }
  \begin{array}{cc}
  \Item{a} & \Item{b} \\
  \Item{c} & \Item{d} \\
  \Item{e} & \Item{f} \\
  \Item{g} & \Item{h} \\
  \Item{i} & \Item{j} \\
  \end{array}
$$

Não é preciso simplificar as frações. Arrume os seus resultados em
três tabelas com essa cara aqui:

\msk

\def\tabela#1{%
  \begin{tabular}{lll}
    \multicolumn{3}{l}{#1} \\
    \hline
    a) && b) \\
    c) && d) \\
    e) && f) \\
    g) && h) \\
    i) && j) \\
  \end{tabular}}

\tabela{Média}
\quad
\tabela{Desvio médio}
\quad
\tabela{Variância}

}\anothercol{

% «questao-2»  (to ".questao-2")
% (esm231p1p 2 "questao-2")
% (esm231p1a   "questao-2")

{\bf Questão 2}

\T(Total: 4.0 pts)

\msk

Na turma que eu usei num monte de exemplos as crianças se chamavam
Ana, Bia, Carlos, Dani, Eduardo, Fábio, Geraldo, Heloá, Inês e Joana.
O diagrama abaixo diz quantas balas ``X'' e quantas balas ``Y'' cada
criança tem -- por exemplo, a Joana tem quatro balas X e uma bala Y.

% 5gQ1

$$\ga{balas}
$$

a) Transforme as informações do diagrama acima numa tabela. A sua
tabela deve ter pelo menos estas colunas aqui: $i$, $\nome_i$, $X_i$,
$Y_i$, e $Z_i=(X_i-Y_i)$ -- mas você pode acrescentar outras se você
achar que isto vai te ajudar a responder os outros itens.

\msk

b) Calcule estas probabilidades:
%
$$\begin{array}{ccc}
  P(X≤0), & P(Y≤0), & P(Z=0), \\
  P(X≤1), & P(Y≤1), & P(Z=1), \\
  P(X≤2), & P(Y≤2), & P(Z=2), \\
  P(X≤3), & P(Y≤3), & \\
  P(X≤4), & P(Y≤4). & \\
  \end{array}
$$

}}





\newpage

\scalebox{0.6}{\def\colwidth{12cm}\firstcol{

% «questao-3»  (to ".questao-3")
% (esm231p1p 3 "questao-3")
% (esm231p1a   "questao-3")

{\bf Questão 3}

\T(Total: 4.0 pts)

\ssk

Aqui nós vamos usar o mesmo diagrama da questão 2. Nesta questão você
pode usar tanto os diagramas que só tem bolinhas da terceira coluna do
anexo quanto os diagramas que têm bolinhas e blobs que eu desenhei à
mão e pus na coluna da direita. O último desenho, com um blob
pequenininho dentro de um blob maior, é um jeito de visualizar
probabilidades condicionais.

$$\ga{balas}
$$

a) Represente graficamente estes eventos:

$X≤2$, $X≥2$,

$Y≥2$, $Y≤2$,

$Z≥0$, $Z≥1$,

e diga a probabilidade de cada um deles.

\msk

b) Para cada uma das probabilidades condicionais abaixo represente-a
graficamente e diga quanto ela vale como um número:

$P(X≤2 | Z≥0)$,

$P(Z≥0 | X≥2)$.



}\anothercol{

\vspace*{0cm}

\includegraphics[height=2cm]{2023-1-ES/est-P1-X3.pdf}

\includegraphics[height=2cm]{2023-1-ES/est-P1-Y1.pdf}

\includegraphics[height=2cm]{2023-1-ES/est-P1-Y1-X3.pdf}



}}



% (find-latexscan-links "ES" "est-P1-X3")
% (find-latexscan-links "ES" "est-P1-Y1-X3")
% (find-latexscan-links "ES" "est-P1-Y1")

% (find-latexscan-links "ES" "est-P1-X3")
% (find-xpdf-page "~/LATEX/2023-1-ES/est-P1-X3.pdf")
% (find-latexscan-links "ES" "est-P1-Y1-X3")
% (find-xpdf-page "~/LATEX/2023-1-ES/est-P1-Y1-X3.pdf")
% (find-latexscan-links "ES" "est-P1-Y1")
% (find-xpdf-page "~/LATEX/2023-1-ES/est-P1-Y1.pdf")




\newpage

% «pacocas»  (to ".pacocas")

\vspace*{-0.5cm}

\scalebox{0.45}{\def\colwidth{8cm}\firstcol{

% «dicas-medias»  (to ".dicas-medias")
% (esm231p1p 4 "dicas-medias")
% (esm231p1a   "dicas-medias")

{\bf Médias}

\unitlength=13pt

\ssk

Definição: $\ovl{X} = \frac{1}{N} \sum_{i=1}^N X_i$.

\ssk

Digamos que temos duas variáveis, $A$ e $D$ -- ``antes'' e ``depois''
-- que dizem o número de paçocas de cada criança antes e depois do
Carlos dar uma paçoca pra Beatriz. As distribuições de $A$ e de $D$
são diferentes, mas como o número total de paçocas não mudou as médias
dessas duas distribuições são iguais: $\ovl{A} = \ovl{D}$. Por
exemplo:


$$\begin{array}{clccc}
  i & \nome_i & A_i & D_i \\
  \hline
  1 & \text{Ana}     & 1 & 1 \\
  2 & \text{Beatriz} & 1 & 2 \\
  3 & \text{Carlos}  & 4 & 3 \\
  4 & \text{Dani}    & 5 & 5 \\
  \end{array}
$$
$$\Media\P{\ga{pacocas 1}} =
  \Media\P{\ga{pacocas 2}}
$$

\bsk

% «dicas-dm-e-var»  (to ".dicas-dm-e-var")
% (esm231p1p 4 "dicas-dm-e-var")
% (esm231p1a   "dicas-dm-e-var")

{\bf Desvio médio e variância}

\ssk

$\DM(X)  = \frac{1}{N} \sum_{i=1}^N |X_i - \ovl{X}|$

$\Var(X) = \frac{1}{N} \sum_{i=1}^N (X_i - \ovl{X})^2$

\ssk

Lembre que $|42|=42$, $|-42|=42$, e que nas aulas a gente calculou o
desvio médio e a variância usando ``histogramas com numerozinhos'',
como esses aqui:
%
$$\scalebox{1.5}{$
  \ga{dm} \qquad \ga{var}
  $}
$$


}\anothercol{

% «dicas-probabilidade»  (to ".dicas-probabilidade")
% (esm231p1p 4 "dicas-probabilidade")
% (esm231p1a   "dicas-probabilidade")

{\bf Probabilidade}

Uma variável que só pode assumir os valores `$\True$' (verdadeiro) ou
`$\False$' (falso) é uma variável {\sl booleana}.

A operação $[·]$ (``colchete'') transforma boole\-anos nos valores 0 e
1. Por exemplo:
%
$$\begin{array}{c}
  [2<3] = [\True] = 1,   \\{}
  [2>3] = [\False] = 0.  \\
  \end{array}
$$
%
Se $B$ é uma variável booleana e todas as linhas da nossa tabela são
``equiprováveis'' então a probabilidade de $B$, $P(B)$ é definida
assim:
%
$$P(B) = \frac{1}{N} \sum_{i=1}^N [B_i].$$
%
Às vezes a gente intepreta expressões como `$A<42$' como {\sl
  variáveis com nomes longos} -- e aí $(A<42)_i = (A_i<42)$. E às
vezes a gente coloca definições na primeira linha da tabela. Por
exemplo, em
%
$$\begin{array}{cccc}
  i & A_i & B_i=(A_i<42) & C_i=[B_i] \\
  \hline
  1 & 200 & \False & 0 \\
  2 & 20  & \True  & 1 \\
  3 & 99  & \False & 0 \\
  \end{array}
$$
%
a segunda coluna diz que cada $B_i$ vai ser definido como o resultado
do $A_i<42$ correspondente e lista os valores dos `$B_i$'s, e a
terceira coluna faz a mesma coisa pros `$C_i$'s.

Neste caso temos $\ovl{C} = \ovl{[B]} = \frac13$ e:
%
$$P(B) = P(A<42) = \frac13.$$


}\anothercol{

% «dicas-conjuntos»  (to ".dicas-conjuntos")
% (esm231p1p 4 "dicas-conjuntos")
% (esm231p1a   "dicas-conjuntos")

\sa {A}          {\unddots{A}{1,2 2,2}}
\sa       {B}    {\unddots{B}{2,2 2,1}}
\sa {A cap B}    {\unddots{\ga{A}∩\ga{B}}{2,2}}
\sa{(A cap B)^c} {\unddots{(\ga{A cap B})^c}{2,1 1,1 1,2}}
\sa{A^c}         {\unddots{(\ga{A})^c}{1,1 2,1}}
\sa        {B^c} {\unddots{(\ga{B})^c}{1,1 1,2}}
\sa{A^c cup B^c} {\unddots{\ga{A^c}∪\ga{B^c}}{2,1 1,1 1,2}}

{\bf Conjuntos}  

\ssk

\unitlength=4pt

Se $Ω=\{1,2,3,4\}$, $A=\{1,2\}$ e $B=\{2,3\}$

então $A∪B = \{1,2,3\}$, $A∩B=\{2\},$

e $A^c = \{3,4\}$.

\ssk

%L PictBounds.setbounds(v(0,0), v(2,2))
\pu

\unitlength=6pt

Se $Ω = \pdots{1,1 1,2 2,1 2,2}$,
   $A = \pdots{    1,2     2,2}$ e
   $B = \pdots{        2,1 2,2}$,
   então:

\unitlength=4pt

$$%
  \und{
    \ga{(A cap B)^c} \;=\;
    \ga{A^c cup B^c}
  }{\True}
$$

\msk

Às vezes a gente diz qual é a probabilidade de cada ``evento''. Nós
usamos este exemplo aqui várias vezes:

\unitlength=6pt

$$\def\PP#1#2{P \P{\pdots{#1}}=\frac{#2}{10}}
  \begin{array}{c}
  \PP{1,2}{1}, \; \PP{2,2}{2}, \\
  \PP{1,1}{3}, \; \PP{2,1}{4}.
  \end{array}
$$

Quando a gente não diz a probabilidade de cada evento fica implícito
que eles são equiprováveis.

\bsk
\bsk

% «dicas-prob-cond»  (to ".dicas-prob-cond")
% (esm231p1p 4 "dicas-prob-cond")
% (esm231p1a   "dicas-prob-cond")

{\bf Probabilidade condicional}

A definição é: $P(A|B) = P(A∩B) / P(B)$.

%L PictBounds.setbounds(v(0,0), v(3,3))
\pu
\unitlength=6pt

Por exemplo:

$P\P{\pdots{1,3 2,2 3,1} \;\; | \;\;
     \pdots{2,3 3,3 2,2 3,2 2,1 3,1}}
 =
 P\P{\pdots{    2,2 3,1}} /
 P\P{\pdots{2,3 3,3 2,2 3,2 2,1 3,1}}.
$

}}


\newpage

% «questao-1-gab»  (to ".questao-1-gab")
% (esm231p1p 5 "questao-1-gab")
% (esm231p1a   "questao-1-gab")

\scalebox{0.6}{\def\colwidth{6.5cm}\firstcol{

{\bf Questão 1: gabarito}

Dá pra passar de cada distribuição na coluna da esquerda pra
distribuição abaixo dela transferindo uma paçoca de uma criança pra
outra, e idem na coluna da direita... então todas as distribuições à
esquerda têm a mesma média, que é 2, e todas as distribuições à
esquerda também têm a mesma média, que é 3. E aí dá pra calcular os
desvios médios e as variâncias fazendo histogramas com numerozinhos --
veja os diagramas à direita.


}\anothercol{

\unitlength=13pt

\vspace*{0cm}

$\def\Item#1{ \text{#1)}\;\ga{Q1#1_DM} }
  \begin{array}{cc}
  \Item{a} & \Item{b} \\
  \Item{c} & \Item{d} \\
  \Item{e} & \Item{f} \\
  \Item{g} & \Item{h} \\
  \Item{i} & \Item{j} \\
  \end{array}
  \qquad
  \def\Item#1{ \text{#1)}\;\ga{Q1#1_Var} }
  \begin{array}{cc}
  \Item{a} & \Item{b} \\
  \Item{c} & \Item{d} \\
  \Item{e} & \Item{f} \\
  \Item{g} & \Item{h} \\
  \Item{i} & \Item{j} \\
  \end{array}
$

\def\tabela#1{%
  \begin{tabular}{lll}
    \multicolumn{3}{l}{#1} \\
    \hline
    a) && b) \\
    c) && d) \\
    e) && f) \\
    g) && h) \\
    i) && j) \\
  \end{tabular}}

\bsk
\bsk

$
  \begin{tabular}{lll}
    \multicolumn{3}{l}{Média} \\
    \hline
    a) 2 && b) 3 \\
    c) 2 && d) 3 \\
    e) 2 && f) 3 \\
    g) 2 && h) 3 \\
    i) 2 && j) 3 \\
  \end{tabular}
  \qquad
  \begin{tabular}{lll}
    \multicolumn{3}{l}{Desvio médio} \\
    \hline
    a) 2/5 && b) 2/5 \\
    c) 4/5 && d) 4/5 \\
    e) 6/5 && f) 6/5 \\
    g) 8/5 && h) 8/5 \\
    i) 8/5 && j) 8/5 \\
  \end{tabular}
  \qquad
  \begin{tabular}{lll}
    \multicolumn{3}{l}{Variância} \\
    \hline
    a)  2/5 && b)  2/5 \\
    c)  4/5 && d)  4/5 \\
    e) 10/5 && f) 10/5 \\
    g) 16/5 && h) 16/5 \\
    i) 14/5 && j) 14/5 \\
  \end{tabular}
$


}}

\newpage

% «questao-2-gab»  (to ".questao-2-gab")
% (esm231p1p 6 "questao-2-gab")
% (esm231p1a   "questao-2-gab")

\scalebox{0.6}{\def\colwidth{7.5cm}\firstcol{

{\bf Questão 2: gabarito}

Figura original:

$$\ga{balas}
$$

\msk

O item (a) dá esta tabela:
%
$$\begin{array}{rlcccccc}
  i & \nome_i & X_i & Y_i & Z_i=(X_i-Y_i) \\
  \hline
   1 & \text{Ana}     & 1 & 3 & -2 & \\
   2 & \text{Bia}     & 2 & 3 & -1 & \\
   3 & \text{Carlos}  & 3 & 3 &  0 & \\
   4 & \text{Dani}    & 1 & 2 & -1 & \\
   5 & \text{Eduardo} & 2 & 2 &  0 & \\
   6 & \text{Fábio}   & 3 & 2 &  1 & \\
   7 & \text{Geraldo} & 1 & 1 &  0 & \\
   8 & \text{Heloá}   & 2 & 1 &  1 & \\
   9 & \text{Inês}    & 3 & 1 &  2 & \\
  10 & \text{Joana}   & 4 & 1 &  3 & \\
  \end{array}
$$


}\anothercol{



\vspace*{0cm}

O item (b) dá isto aqui:

\bsk

$\begin{array}{lll}
  P(X≤0)=  0,   & P(Y≤0)=0,     & P(Z=0)=3/10, \\
  P(X≤1)= 3/10, & P(Y≤1)=4/10,  & P(Z=1)=2/10, \\
  P(X≤2)= 6/10, & P(Y≤2)=7/10,  & P(Z=2)=1/10, \\
  P(X≤3)= 9/10, & P(Y≤3)=10/10, & \\
  P(X≤4)=10/10, & P(Y≤4)=10/10. & \\
  \end{array}
$




}}


\newpage


\scalebox{0.6}{\def\colwidth{9cm}\firstcol{

{\bf Questão 3: gabarito}

Item a:
%
$$\begin{array}{cc}
  \MBC{X≤2}{X<=2}{6/10} &
  \MBC{X≥2}{X>=2}{7/10} \\[15pt]
  \MBC{Y≥2}{Y>=2}{6/10} &
  \MBC{Y≤2}{Y<=2}{7/10} \\[15pt]
  \MBC{Z≥0}{Z>=0}{7/10} &
  \MBC{Z≥1}{Z>=1}{4/10} \\
  \end{array}
$$

Item b:

\bsk

$\def\und#1#2{\underbrace{#1}_{#2}}
 \sa{p X<=2}{\und{X≤2}{\minibalas{X<=2}}}
 \sa{p X>=2}{\und{X≥2}{\minibalas{X>=2}}}
 \sa{p Z>=0}{\und{Z≥0}{\minibalas{Z>=0}}}
 %
 \sa{p X<=2 and Z>=0}{\und{(\ga{p X<=2})∩(\ga{p Z>=0})}
                          {\minibalas{X<=2 and Z>=0}}}
 \sa{p Z>=0 and X>=2}{\und{(\ga{p Z>=0})∩(\ga{p X>=2})}
                          {\minibalas{Z>=0 and X>=2}}}
 %
 \sa{P(X<=2  |  Z>=0)}    {P(\ga{p X<=2} | \ga{p Z>=0})}
 \sa{P(X<=2 and Z>=0)}    {\und{P(\ga{p X<=2 and Z>=0})}{3/10}}
 \sa         {P(Z>=0)}    {\und{P(\ga{p Z>=0})}{7/10}}
 %
 \sa{P(Z>=0  |  X>=2)}    {P(\ga{p Z>=0} | \ga{p X>=2})}
 \sa{P(Z>=0 and X>=2)}    {\und{P(\ga{p Z>=0 and X>=2})}{6/10}}
 \sa         {P(X>=2)}    {\und{P(\ga{p X>=2})}{7/10}}
 %
 \begin{array}{c}
   \ga{P(X<=2  |  Z>=0)} \;=\;
   \ga{P(X<=2 and Z>=0)} /
   \ga         {P(Z>=0)} \;=\; 3/7
   \\
   \\\\[-17.5pt]
   \ga{P(Z>=0  |  X>=2)} \;=\;
   \ga{P(Z>=0 and X>=2)} /
   \ga         {P(X>=2)} \;=\; 6/7
 \end{array}
$

}\anothercol{
}}




% \par Calcular médias
% \par Calcular variância
% \par Variáveis com nomes longos
% 
% Assuntos:
% 
% \par 1. Distribuições. Tabelas.
% \par 2. Histogramas identificados e não identificados.
% \par 3. Espaço amostral. Variável. Índice.
% \par 4. Notação matemática: funções, somatórios. Média.
% \par 5. Distribuições em duas variáveis.
% \par 6. Medidas de dispersão: amplitude, variância, desvio padrão.
% \par 7. Eventos. Probabilidade.
% \par 8. Conjuntos. Eventos como conjuntos.
% \par 9. Restrições. Probabilidade condicional.
% \par 10. Correlação.
% \par 11. Probabilidade acumulada.
% \par 12. Mediana. Moda. Quartis e percentis.
% \par 13. Boxplots.
% \par 14. Curva normal.
% \par 15. Distribuições contínuas.
% \par 16. Distribuição uniforme.
% \par 17. Distribuição normal padrão.
% \par 18. Distribuição normal não-padrão.
% \par 19. Desvio padrão.


\newpage

{\bf Avisos (do dia anterior à prova)}

{\sl Eu ainda não fiz as questões da prova!}

A próxima folha tem o anexo que eu prometi - uma folha com um monte de
definições e exemplos que podem ajudar vocês a fazerem as questões da
prova.

A parte em que as pessoas tiveram mais dificuldade foi a das
``variáveis com nomes longos'' da coluna do meio do anexo. A gente
discutiu isso no dia 10/maio, nas páginas 16 e 17 do PDF com as fotos
dos quadros. O link é este: \Ca{5gQ16}. {\sl Revisem isso!!!} \quad
\smile






% «links»  (to ".links")

\GenericWarning{Success:}{Success!!!}  % Used by `M-x cv'

\end{document}

%  ____  _             _         
% |  _ \(_)_   ___   _(_)_______ 
% | | | | \ \ / / | | | |_  / _ \
% | |_| | |\ V /| |_| | |/ /  __/
% |____// | \_/  \__,_|_/___\___|
%     |__/                       
%
% «djvuize»  (to ".djvuize")
% (find-LATEXgrep "grep --color -nH --null -e djvuize 2020-1*.tex")

 (eepitch-shell)
 (eepitch-kill)
 (eepitch-shell)
# (find-fline "~/2023.1-ES/")
# (find-fline "~/LATEX/2023-1-ES/")
# (find-fline "~/bin/djvuize")

cd /tmp/
for i in *.jpg; do echo f $(basename $i .jpg); done

f () { rm -v $1.pdf;  textcleaner -f 50 -o  5 $1.jpg $1.png; djvuize $1.pdf; xpdf $1.pdf }
f () { rm -v $1.pdf;  textcleaner -f 50 -o 10 $1.jpg $1.png; djvuize $1.pdf; xpdf $1.pdf }
f () { rm -v $1.pdf;  textcleaner -f 50 -o 20 $1.jpg $1.png; djvuize $1.pdf; xpdf $1.pdf }

f () { rm -fv $1.png $1.pdf; djvuize $1.pdf }
f () { rm -fv $1.png $1.pdf; djvuize WHITEBOARDOPTS="-m 1.0 -f 15" $1.pdf; xpdf $1.pdf }
f () { rm -fv $1.png $1.pdf; djvuize WHITEBOARDOPTS="-m 1.0 -f 30" $1.pdf; xpdf $1.pdf }
f () { rm -fv $1.png $1.pdf; djvuize WHITEBOARDOPTS="-m 1.0 -f 45" $1.pdf; xpdf $1.pdf }
f () { rm -fv $1.png $1.pdf; djvuize WHITEBOARDOPTS="-m 0.5" $1.pdf; xpdf $1.pdf }
f () { rm -fv $1.png $1.pdf; djvuize WHITEBOARDOPTS="-m 0.25" $1.pdf; xpdf $1.pdf }
f () { cp -fv $1.png $1.pdf       ~/2023.1-ES/
       cp -fv        $1.pdf ~/LATEX/2023-1-ES/
       cat <<%%%
% (find-latexscan-links "ES" "$1")
%%%
}

f est-P1-X3
f est-P1-Y1-X3
f est-P1-Y1

f 20201213_area_em_funcao_de_theta
f 20201213_area_em_funcao_de_x
f 20201213_area_fatias_pizza

% Local Variables:
% coding: utf-8-unix
% ee-tla: "esp1"
% ee-tla: "esm231p1"
% End:
