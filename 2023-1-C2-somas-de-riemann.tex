% (find-LATEX "2023-1-C2-somas-de-riemann.tex")
% (defun c () (interactive) (find-LATEXsh "lualatex -record 2023-1-C2-somas-de-riemann.tex" :end))
% (defun C () (interactive) (find-LATEXsh "lualatex 2023-1-C2-somas-de-riemann.tex" "Success!!!"))
% (defun D () (interactive) (find-pdf-page      "~/LATEX/2023-1-C2-somas-de-riemann.pdf"))
% (defun d () (interactive) (find-pdftools-page "~/LATEX/2023-1-C2-somas-de-riemann.pdf"))
% (defun e () (interactive) (find-LATEX "2023-1-C2-somas-de-riemann.tex"))
% (defun o  () (interactive) (find-LATEX "2022-2-C2-somas-de-riemann.tex"))
% (defun ot () (interactive) (find-LATEX "2022-2-C2-TFC1-e-TFC2.tex"))
% (defun oi () (interactive) (find-LATEX "2022-1-C2-infs-e-sups.tex"))
% (defun u  () (interactive) (find-latex-upload-links "2023-1-C2-somas-de-riemann"))
% (defun v  () (interactive) (find-2a '(e) '(d)))
% (defun d0 () (interactive) (find-ebuffer "2023-1-C2-somas-de-riemann.pdf"))
% (defun cv () (interactive) (C) (ee-kill-this-buffer) (v) (g))
%          (code-eec-LATEX "2023-1-C2-somas-de-riemann")
% (find-pdf-page   "~/LATEX/2023-1-C2-somas-de-riemann.pdf")
% (find-sh0 "cp -v  ~/LATEX/2023-1-C2-somas-de-riemann.pdf /tmp/")
% (find-sh0 "cp -v  ~/LATEX/2023-1-C2-somas-de-riemann.pdf /tmp/pen/")
%     (find-xournalpp "/tmp/2023-1-C2-somas-de-riemann.pdf")
%   file:///home/edrx/LATEX/2023-1-C2-somas-de-riemann.pdf
%               file:///tmp/2023-1-C2-somas-de-riemann.pdf
%           file:///tmp/pen/2023-1-C2-somas-de-riemann.pdf
%  http://anggtwu.net/LATEX/2023-1-C2-somas-de-riemann.pdf
% (find-LATEX "2019.mk")
% (find-sh0 "cd ~/LUA/; cp -v Pict2e1.lua Pict2e1-1.lua Piecewise1.lua ~/LATEX/")
% (find-sh0 "cd ~/LUA/; cp -v Pict2e1.lua Pict2e1-1.lua Pict3D1.lua ~/LATEX/")
% (find-sh0 "cd ~/LUA/; cp -v C2Subst1.lua C2Formulas1.lua ~/LATEX/")
% (find-sh0 "cd ~/LUA/; cp -v Gram2.lua Tree1.lua Caepro5.lua ~/LATEX/")
% (find-MM-aula-links "2023-1-C2-somas-de-riemann" "C2" "c2m231sr" "c2sr")

% «.defs»			(to "defs")
% «.title»			(to "title")
% «.links»			(to "links")
% «.montanhas»			(to "montanhas")
% «.montanhas-figs»		(to "montanhas-figs")
% «.miranda-sup-inf»		(to "miranda-sup-inf")
% «.aviso»			(to "aviso")
% «.particoes»			(to "particoes")
% «.def-particao»		(to "def-particao")
% «.def-inf-e-sup»		(to "def-inf-e-sup")
% «.algumas-somas»		(to "algumas-somas")
% «.def-integral»		(to "def-integral")
% «.particoes-exercs»		(to "particoes-exercs")
% «.dica-simplificacao»		(to "dica-simplificacao")
% «.imagens-figuras»		(to "imagens-figuras")
% «.imagens-de-intervalos»	(to "imagens-de-intervalos")
% «.imagens-exercicio»		(to "imagens-exercicio")
% «.descontinua»		(to "descontinua")

% «.instrucoes-des-defs»	(to "instrucoes-des-defs")
% «.instrucoes-des-1»		(to "instrucoes-des-1")
% «.instrucoes-des-2»		(to "instrucoes-des-2")
% «.instrucoes-des-ex»		(to "instrucoes-des-ex")
% «.instrucoes-des-grid»	(to "instrucoes-des-grid")
% «.instrucoes-des-ex-2»	(to "instrucoes-des-ex-2")
% «.into-e-intu»		(to "into-e-intu")
% «.um-jogo»			(to "um-jogo")
% «.um-jogo-2»			(to "um-jogo-2")
% «.dirichlet»			(to "dirichlet")
% «.integral-como-limite»	(to "integral-como-limite")
% «.TFC1»			(to "TFC1")



% <videos>
% Video (not yet):
% (find-ssr-links     "c2m231sr" "2023-1-C2-somas-de-riemann")
% (code-eevvideo      "c2m231sr" "2023-1-C2-somas-de-riemann")
% (code-eevlinksvideo "c2m231sr" "2023-1-C2-somas-de-riemann")
% (find-c2m231srvideo "0:00")

\documentclass[oneside,12pt]{article}
\usepackage[colorlinks,citecolor=DarkRed,urlcolor=DarkRed]{hyperref} % (find-es "tex" "hyperref")
\usepackage{amsmath}
\usepackage{amsfonts}
\usepackage{amssymb}
\usepackage{pict2e}
\usepackage[x11names,svgnames]{xcolor} % (find-es "tex" "xcolor")
\usepackage{colorweb}                  % (find-es "tex" "colorweb")
%\usepackage{tikz}
%
% (find-dn6 "preamble6.lua" "preamble0")
%\usepackage{proof}   % For derivation trees ("%:" lines)
%\input diagxy        % For 2D diagrams ("%D" lines)
%\xyoption{curve}     % For the ".curve=" feature in 2D diagrams
%
\usepackage{edrx21}               % (find-LATEX "edrx21.sty")
\input edrxaccents.tex            % (find-LATEX "edrxaccents.tex")
\input edrx21chars.tex            % (find-LATEX "edrx21chars.tex")
\input edrxheadfoot.tex           % (find-LATEX "edrxheadfoot.tex")
\input edrxgac2.tex               % (find-LATEX "edrxgac2.tex")
%\usepackage{emaxima}              % (find-LATEX "emaxima.sty")
%
% (find-es "tex" "geometry")
\usepackage[a6paper, landscape,
            top=1.5cm, bottom=.25cm, left=1cm, right=1cm, includefoot
           ]{geometry}
%
\begin{document}

% «defs»  (to ".defs")
% (find-LATEX "edrx21defs.tex" "colors")
% (find-LATEX "edrx21.sty")

\def\drafturl{http://anggtwu.net/LATEX/2023-1-C2.pdf}
\def\drafturl{http://anggtwu.net/2023.1-C2.html}
\def\draftfooter{\tiny \href{\drafturl}{\jobname{}} \ColorBrown{\shorttoday{} \hours}}

\def\Rext{\overline{\R}}
\def\V{\mathbf{V}}
\def\F{\mathbf{F}}

\def\into{\overline ∫}
\def\intu{\underline∫}
\def\intou{\overline{\underline∫}}
\def\INTx#1#2#3#4{#1_{x=#2}^{x=#3} #4 \, dx}
\def\INTP  #1#2#3{#1_{#2}          #3 \, dx}

\def\mname#1{\ensuremath{[\text{#1}]}}
\def\minf{\mname{inf}}
\def\msup{\mname{sup}}
\def\sse {\text{sse}}

\def\sumiN#1{\sum_{i=1}^N #1 (b_i-a_i)}

\sa{into_P     f(x) dx}{\INTP{\into} {P}{f(x)}}
\sa{intu_P     f(x) dx}{\INTP{\intu} {P}{f(x)}}
\sa{intou_P    f(x) dx}{\INTP{\intou}{P}{f(x)}}
\sa{into_ab2k  f(x) dx}{\INTP{\into} {[a,b]_{2^k}}{f(x)}}
\sa{intu_ab2k  f(x) dx}{\INTP{\intu} {[a,b]_{2^k}}{f(x)}}
\sa{intou_ab2k f(x) dx}{\INTP{\intou}{[a,b]_{2^k}}{f(x)}}
\sa{into_xab   f(x) dx}{\INTx{\into} {a}{b}{f(x)}}
\sa{intu_xab   f(x) dx}{\INTx{\intu} {a}{b}{f(x)}}
\sa{intou_xab  f(x) dx}{\INTx{\intou}{a}{b}{f(x)}}
\sa{int_xab    f(x) dx}{\INTx{\int}  {a}{b}{f(x)}}


% (find-LATEX "2023-1-C2-carro.tex" "defs-caepro")
% (find-LATEX "2023-1-C2-carro.tex" "defs-pict2e")

\catcode`\^^J=10
\directlua{dofile "dednat6load.lua"}  % (find-LATEX "dednat6load.lua")

% «defs-caepro»  (to ".defs-caepro")
%L dofile "Caepro5.lua"              -- (find-angg "LUA/Caepro5.lua" "LaTeX")
\def\Caurl   #1{\expr{Caurl("#1")}}
\def\Cahref#1#2{\href{\Caurl{#1}}{#2}}
\def\Ca      #1{\Cahref{#1}{#1}}

% «defs-pict2e»  (to ".defs-pict2e")
%L V = nil                           -- (find-angg "LUA/Pict2e1.lua" "MiniV")
%L dofile "Piecewise1.lua"           -- (find-LATEX "Piecewise1.lua")
%L Pict2e.__index.suffix = "%"
\def\pictgridstyle{\color{GrayPale}\linethickness{0.3pt}}
\def\pictaxesstyle{\linethickness{0.5pt}}
\def\pictnaxesstyle{\color{GrayPale}\linethickness{0.5pt}}
\celllower=2.5pt

\pu





%  _____ _ _   _                               
% |_   _(_) |_| | ___   _ __   __ _  __ _  ___ 
%   | | | | __| |/ _ \ | '_ \ / _` |/ _` |/ _ \
%   | | | | |_| |  __/ | |_) | (_| | (_| |  __/
%   |_| |_|\__|_|\___| | .__/ \__,_|\__, |\___|
%                      |_|          |___/      
%
% «title»  (to ".title")
% (c2m231srp 1 "title")
% (c2m231sra   "title")

\thispagestyle{empty}

\begin{center}

\vspace*{1.2cm}

{\bf \Large Cálculo C2 - 2023.1}

\bsk

Aulas 15 até 19: Somas de Riemann

\bsk

Eduardo Ochs - RCN/PURO/UFF

\url{http://anggtwu.net/2023.1-C2.html}

\end{center}

\newpage

% «links»  (to ".links")
% (c2m231srp 2 "links")
% (c2m231sra   "links")

{\bf Links}

\scalebox{0.6}{\def\colwidth{12cm}\firstcol{

\par Leithold:
\par \Ca{Leit5p35} (p.318) Figura 3
\par \Ca{Leit5p36} (p.319) Figura 4
\par \Ca{Leit5p41} (p.324) 5.5. A integral definida

\ssk

\par Miranda:
\par \Ca{Miranda207} 7.1 Áreas e somas de Riemann
\par \Ca{Miranda212} 7.2 Integral definida
\par \Ca{Miranda217} 7.3. Definição 3: Soma superior e inferior

\ssk

\par Livro de Análise do Ross:
\par \Ca{RossAp16} (p.269) The Riemann Integral

\msk

\par Vou (re)usar muito material destes PDFzinhos:
\par \Ca{2fT60} 2022.2, aulas 13, 14 e 16: Somas de Riemann
\par \Ca{2fT89} 2022.2, aula 19, 14 e 16: o TFC1 e o TFC2
\par \Ca{2eT39} 2022.1, aula 15: infs e sups

\msk

\par Algumas figuras importantes:
\par \Ca{2eT95} A integral como limite
\par \Ca{2fT91} Algumas definições: partição, inf e sup, integral

\bsk

\par \Ca{2gQ32} Quadros da aula 15 (23/maio/2023)
\par \Ca{2gQ34} Quadros da Aula 16 (26/maio/2023)
\par \Ca{2gQ37} Quadros da Aula 18 (02/junho/2023)
\par \Ca{2gQ39} Quadros da Aula 19 (06/junho/2023)

}\anothercol{
}}


\newpage

% «montanhas»  (to ".montanhas")
% (c2m231srp 3 "montanhas")
% (c2m231sra   "montanhas")

% (c2m222srp 8 "exercicio-4")
% (c2m222sra   "exercicio-4")
% (c2m221somas3p 4 "exercicio-1")
% (c2m221somas3a   "exercicio-1")

{\bf Montanhas}

\def\sumo{\sum_{i=1}^{8}}
\def\sumoo#1{\sumo #1 (x_i - x_{i-1})}

\scalebox{0.9}{\def\colwidth{12cm}\firstcol{

Seja $f(x)$ a função da próxima página -- ``as montanhas''.

Você vai receber (pelo menos) uma cópia dessa página.

Faça cada item abaixo em um dos 12 gráficos da $f(x)$.

\msk

Represente graficamente cada um dos somatórios abaixo.

Se você tiver dificuldade com algum desses somatórios

comece expandindo ele em dois passos, como na página 7.

\msk

a) $\sumoo{f(x_i)}$

\ssk

b) $\sumoo{f(x_{i-1})}$

\ssk

c) $\sumoo{\max(f(x_{i-1}), f(x_i))}$

\ssk

d) $\sumoo{\min(f(x_{i-1}), f(x_i))}$

\ssk

e) $\sumoo{f(\frac{x_{i-1} + x_i}{2})}$

\ssk

f) $\sumoo{\frac{f(x_{i-1}) + f(x_i)}{2}}$

}\anothercol{
}}


\newpage


%  __  __                   _        _           
% |  \/  | ___  _   _ _ __ | |_ __ _(_)_ __  ___ 
% | |\/| |/ _ \| | | | '_ \| __/ _` | | '_ \/ __|
% | |  | | (_) | |_| | | | | || (_| | | | | \__ \
% |_|  |_|\___/ \__,_|_| |_|\__\__,_|_|_| |_|___/
%                                                
% «montanhas-figs»  (to ".montanhas-figs")
% (c2m231srp 4 "montanhas-figs")
% (c2m231sra   "montanhas-figs")
% (c2m222srp 9 "mountains")
% (c2m222sra   "mountains")
% (c2m221somas3p 3 "mountains")
% (c2m221somas3a   "mountains")

% (find-angg "LUA/Piecewise1.lua" "Xtoxytoy-test2")
%
%L Pict2e.bounds = PictBounds.new(v(0,0), v(23,9))
%L spec   = "(0,1)--(5,6)--(7,4)--(11,8)--(15,4)--(17,6)--(23,0)"
%L xs     = {    1,3,    6,     9,  11, 13,   16,19,    21      }
%L labely = -1
%L pws    = PwSpec.from(spec)
%L xtos   = Xtoxytoy.from(pws:fun(), xs)
%L vlines = xtos:topict("v")
%L curve  = pws:topict()
%L labels = PictList {}
%L for i,x in ipairs(xs) do
%L   labels:addputstrat(v(x,labely), "\\cell{x_"..(i-1).."}")
%L end
%L p = PictList { vlines, curve:prethickness("2pt"), labels }
%L p:pgat("pA", "mountain"):output()
\pu

\unitlength=8pt

\vspace*{-0.25cm}
\hspace*{-0.5cm}
$\scalebox{0.55}{$
 \begin{array}{ccccc}
 \mountain && \mountain && \mountain \\[20pt]
 \mountain && \mountain && \mountain \\[20pt]
 \mountain && \mountain && \mountain \\[20pt]
 \mountain && \mountain && \mountain \\[20pt]
 \end{array}
 $}
$

\newpage

%            _                                               
%  _ __ ___ (_)_ __          ___   _ __ ___   __ ___  __     
% | '_ ` _ \| | '_ \        / _ \ | '_ ` _ \ / _` \ \/ /     
% | | | | | | | | | |      |  __/ | | | | | | (_| |>  <      
% |_| |_| |_|_|_| |_|____   \___| |_| |_| |_|\__,_/_/\_\____ 
%                  |_____|                            |_____|
%
% «miranda-sup-inf»  (to ".miranda-sup-inf")
% (c2m231srp 5 "montanhas-min-e-max")
% (c2m231sra   "montanhas-min-e-max")
% (c2m222srp 10 "soma-superior-e")
% (c2m222sra    "soma-superior-e")

{\bf Miranda: somas inferiores e superiores}

\scalebox{0.65}{\def\colwidth{12cm}\firstcol{

Nas páginas 217 e 218 o Miranda define as notações $I(f,P)$ e
$S(f,P)$, e lá no meio dessas definições ele define
%
$$\min_{x∈I} f(x)
  \qquad
  \text{e}
  \qquad
  \max_{x∈I} f(x)
$$

usando o truque do ``vire-se'': ele mostra uma figura e o leitor tem
que se virar pra entender o que essas notações querem dizer... veja:

% (find-books "__analysis/__analysis.el" "miranda")
% (find-books "__analysis/__analysis.el" "miranda" "soma superior")
% (find-dmirandacalcpage 217   "soma superior e inferior")
% (find-dmirandacalcpage 218   "min_")
\Ca{Miranda217} (Definição 3)

\bsk

{\bf Mais itens pra fazer na figura das montanhas}

a) Entenda o que essas notações do Miranda querem dizer

e verifique que na figura das montanhas temos:
%
$$\begin{array}{ccccc}
   && \D \max(f(x_1),f(x_2))
     &\lneqq& \D \max_{x∈[x_1,x_2]}f(x) \\
   \D     \min_{x∈[x_2,x_3]}f(x)
     &\lneqq& \min(f(x_2),f(x_3)) \\
  \end{array}
$$

e depois represente nas montanhas:

\ssk

b) $\sumoo{(\max_{x∈[x_{i-1},x_i]} f(x))}$

\ssk

c) $\sumoo{(\min_{x∈[x_{i-1},x_i]} f(x))}$

}\anothercol{
}}



\newpage

% «aviso»  (to ".aviso")
% (c2m231srp 6 "aviso")
% (c2m231sra   "aviso")

{\bf Aviso}

\scalebox{0.9}{\def\colwidth{10cm}\firstcol{

As próximas páginas têm definições precisas de:

partição, inf e sup, $\minf$ e $\msup$, integral definida,

e um monte de definições intermediárias que a

gente vai precisar pra entender as definições

mais importantes...

\msk

{\sl O objetivo desta parte do curso é fazer vocês aprenderem um monte
  de truques pra entenderem definições complicadas ``visualizando o
  que elas querem dizer''. Estes truques vão ser uma das partes do
  curso que vão ser mais úteis pras matérias seguintes, mas esses
  assuntos vão valer bem poucos pontos na prova.}

% 2gQ32

}\anothercol{
}}


\newpage

% «particoes»  (to ".particoes")
% (c2m231srp 7 "particoes")
% (c2m231sra   "particoes")
% (c2m212somas1p 9 "particoes")
% (c2m212somas1a   "particoes")

{\bf Partições, informalmente}

\scalebox{0.47}{\def\colwidth{12cm}\firstcol{

Informalmente uma partição de um intervalo $[a,b]$ é um modo de
decompor $[a,b]$ em intervalos menores consecutivos. Por exemplo,
%
$$[2,7] = [2,3.5]∪[3.5,4]∪[4,6]∪[6,7]$$

A definição ``certa'' é mais complicada... vamos vê-la daqui a pouco.
O caso geral da igualdade acima é:
%
$$[a,b] = [a_1,b_1]∪[a_2,b_2]∪\ldots∪[a_N,b_N],$$

onde:

$N$ é o número de intervalos,

$a=a_1$, $b=b_N$, (``extremidades'')

$a_i<b_i$ para todo $i$ em que isto faz sentido ($i=1,\ldots,N$)

$b_i=a_{i+1}$ para todo $i$ e.q.i.f.s.; neste caso, $i=1,\ldots,N-1$

\bsk

Um jeito prático de definir uma partição é usando uma tabela.

Por exemplo, esta tabela
%
$$\begin{array}{cccc}
  i & a_i & b_i & I_i \\\hline
  1 & 2   & 3.5 & [2, 3.5] \\
  2 & 3.5 & 4   & [3.5, 4] \\
  3 & 4   & 6   & [4,   6] \\
  4 & 6   & 7   & [6,   7] \\
  \end{array}
$$

corresponde à partição de $[2,7]$ do início deste slide.

\bsk

% (find-dmirandacalcpage 212 "7.2. Integral definida")
% (find-leitholdptpage (+ 17 324) "5.5. A integral definida")
Veja: \Ca{Miranda212}, \Ca{Leit5p41} (p.324, seção 5.5).

}\anothercol{

Uma definição um pouco melhor de partição é a seguinte.

Digamos que $P$ seja um subconjunto não-vazio e finito de $\R$,

e que o menor elemento de $P$ seja $a$ e o maior seja $b$.

\ColorRed{Então $P$ é uma partição do intervalo $[a,b]$.}

\msk

Exemplo: a partição $P=\{2,3.5,4,6,7\}$ corresponde a:
%
$$[2,7] = [2,3.5]∪[3.5,4]∪[4,6]∪[6,7]$$



Pra fazer a tradução da ``versão conjunto'' pra ``versão tabela''
ponha os elementos de $P$ em ordem e chame-os de $b_0,\ldots,b_N$;
defina cada $a_i$ como sendo $b_{i-1}$ -- por exemplo, $a_1 = b_0$ --
e encontre $a$, $b$, e $N$. Depois que você tem a ``versão tabela'' é
bem fácil obter a ``versão união de intervalos''.

\bsk

Quando dizemos algo como ``Seja $P$ a partição $\{2.5,4,6\}$'' estamos
criando um contexto no qual há uma partição ``default'' definida... e
neste contexto vamos ter valores definidos para $N$, $a$, $b$, e para
cada $a_i$ e $b_i$. Por exemplo...

\msk

Seja $P$ a partição $\{2.5,4,6\}$. Então
%
$$\begin{array}{rcl}
  \sum_{i=1}^N f(b_i)·(b_i-a_i)
     &=& \sum_{i=1}^2 f(b_i)·(b_i-a_i) \\
     &=& f(b_1)·(b_1-a_1) \\
     &+& f(b_2)·(b_2-a_2) \\
     &=& f(4)·(4-2.5) \\
     &+& f(6)·(6-4) \\
  \end{array}
$$


}}


\newpage

% «exercicio-4»  (to ".exercicio-4")
% (c2m212somas1p 10 "exercicio-4")
% (c2m212somas1a    "exercicio-4")
% (c2sop 10 "exercicio-4")
% (c2soa    "exercicio-4")
% (c2m202somas1p 8 "exercicio-4")
% (c2m202somas1    "exercicio-4")

% «exercicio-5»  (to ".exercicio-5")
% (c2m212somas1p 11 "exercicio-5")
% (c2m212somas1a    "exercicio-5")
% (c2m202somas1p 9 "exercicio-5")
% (c2m202somas1    "exercicio-5")

% «ponto-decimal»  (to ".ponto-decimal")
% Ah, obs, repara que eu vou usar a convencao internacional e vou
% sempre escrever "1.5" ao inves de "1,5" - e recomendo que voces usem
% ela tambem pra gente poder usar a virgula pra outras coisas. Por
% exemplo, na pagina 9 temos P = {2, 3.5, 4, 6, 7}, e se a gente
% escrever "3,5" ao inves de "3.5" vamos ter que usar ";"s como
% separadores entres os numeros...

% «partition-sum»  (to ".partition-sum")
% (c2m211somas1p 12 "partition-sum")
% (c2m211somas1a    "partition-sum")

% «subst»  (to ".subst")
% (c2m202somas1p 11 "subst")
% (c2m202somas1     "subst")


\newpage

%  ____        __                    _   _                 
% |  _ \  ___ / _|  _ __   __ _ _ __| |_(_) ___ __ _  ___  
% | | | |/ _ \ |_  | '_ \ / _` | '__| __| |/ __/ _` |/ _ \ 
% | |_| |  __/  _| | |_) | (_| | |  | |_| | (_| (_| | (_) |
% |____/ \___|_|   | .__/ \__,_|_|   \__|_|\___\__,_|\___/ 
%                  |_|                                     
%
% «def-particao»  (to ".def-particao")
% (c2m231srp 7 "def-particao")
% (c2m231sra   "def-particao")
% (c2m222tfcsp 3 "def-particao")
% (c2m222tfcsa   "def-particao")

{\bf A definição de partição}

\scalebox{0.85}{\def\colwidth{11cm}\firstcol{

Se $P$ é um subconjunto \ColorRed{finito} e \ColorRed{não-vazio} de $\R$,

então podemos interpretar $P$ como uma partição...

Por exemplo, se $P=\{200,20,42,99,63,33,20,20\}$

então $P=\{20,33,42,63,99,200\}$, e aí vamos interpretar

esse conjunto de 6 pontos -- ordenados em ordem crescente --

como uma partição do intervalo $I = [a,b] = [20,200]$ em

5 subintervalos (``$N=5$''), assim:

$$\begin{array}{ccccccl}
  20 & 33 & 42 & 63 & 99 & 200 \\
  x_0 & x_1 & x_2 & x_3 & x_4 & x_5 \\
  a_1 & b_1 &     &     &     &     & I_1=[a_1,b_1] \\
      & a_2 & b_2 &     &     &     & I_2=[a_2,b_2] \\
      &     & a_3 & b_3 &     &     & I_3=[a_3,b_3] \\
      &     &     & a_4 & b_4 &     & I_4=[a_4,b_4] \\
      &     &     &     & a_5 & b_5 & I_5=[a_5,b_5] \\
   a  &     &     &     &     &  b  & I  = [a,b] = [x_0,x_N]\\
  \end{array}
$$

}\anothercol{
}}


\newpage

%  ____        __   _        __                          
% |  _ \  ___ / _| (_)_ __  / _|   ___   ___ _   _ _ __  
% | | | |/ _ \ |_  | | '_ \| |_   / _ \ / __| | | | '_ \ 
% | |_| |  __/  _| | | | | |  _| |  __/ \__ \ |_| | |_) |
% |____/ \___|_|   |_|_| |_|_|    \___| |___/\__,_| .__/ 
%                                                 |_|    
% «def-inf-e-sup»  (to ".def-inf-e-sup")
% (c2m231srp 9 "def-inf-e-sup")
% (c2m231sra   "def-inf-e-sup")
% (c2m222tfcsp 4 "def-inf-e-sup")
% (c2m222tfcsa   "def-inf-e-sup")
% (c2m221isp 3 "algumas-definicoes")
% (c2m221isa   "algumas-definicoes")

{\bf As definições de inf e sup}

\scalebox{0.9}{\def\colwidth{10cm}\firstcol{

Digamos que $f:\R→\R$ e $B⊂\R$.

Vamos definir $\inf(f(B))$ e $\sup(f(B))$ ---

e também $\inf(D)$ e $\sup(D)$, pra $D⊂\R$ ---

desta forma:
%
$$\begin{array}{rcl}
  \Rext &=& \R∪\{-∞,+∞\} \\
  C  &=& \setofst{(x,f(x))}{x∈B} \\
  D  &=& \setofst{f(x)}{x∈B} \\
  D' &=& \setofst{y∈\R}{∃x∈B.\ f(x)=y} \\
  L &=& \setofst{y∈\Rext}{∀d∈D.\;y≤d} \\
  U &=& \setofst{y∈\Rext}{∀d∈D.\;d≤y} \\
  (α=\inf(D)) &=& α∈L ∧ (∀ℓ∈L.\;ℓ \le α) \\
  (β=\sup(D)) &=& β∈U ∧ (∀u∈U.\;β \le u) \\
  \end{array}
$$

% Com isto podemos definir a integral definida.

% A definição formal dela está na próxima página.

}\anothercol{
}}

\newpage

%     _    _                                                                 
%    / \  | | __ _ _   _ _ __ ___   __ _ ___   ___  ___  _ __ ___   __ _ ___ 
%   / _ \ | |/ _` | | | | '_ ` _ \ / _` / __| / __|/ _ \| '_ ` _ \ / _` / __|
%  / ___ \| | (_| | |_| | | | | | | (_| \__ \ \__ \ (_) | | | | | | (_| \__ \
% /_/   \_\_|\__, |\__,_|_| |_| |_|\__,_|___/ |___/\___/|_| |_| |_|\__,_|___/
%            |___/                                                           
%
% «algumas-somas»  (to ".algumas-somas")
% (c2m231srp 10 "algumas-somas")
% (c2m231sra    "algumas-somas")
% (c2m221somas3p 13 "metodos-nomes")
% (c2m221somas3a    "metodos-nomes")

{\bf Algumas somas de Riemann}

\scalebox{0.65}{\def\colwidth{9cm}\firstcol{

Vou definir:
%
$$\begin{array}{ccl}
  \mname{L}    &=& \sumiN {f(a_i)}                    \\[2pt]
  \mname{R}    &=& \sumiN {f(b_i)}                    \\[2pt]
  \mname{Trap} &=& \sumiN {\frac{f(a_i) + f(b_i)}{2}} \\[2pt]
  \mname{M}    &=& \sumiN {f(\frac{a_i+b_i}{2})}      \\[2pt]
  \mname{min}  &=& \sumiN {\min(f(a_i), f(b_i))}      \\[2pt]
  \mname{max}  &=& \sumiN {\max(f(a_i), f(b_i))}      \\[2pt]
  \mname{inf}  &=& \sumiN {\inf(f([a_i,b_i]))}        \\[2pt]
  \mname{sup}  &=& \sumiN {\sup(f([a_i,b_i]))}        \\
  \end{array}
$$

Compare com: os exercícios das montanhas, 

as páginas 208--210 do Miranda (\Ca{Miranda208}), e:

{\footnotesize

\url{https://pt.wikipedia.org/wiki/Soma_de_Riemann}

}

\msk

Nas duas últimas linhas o $f([a_i,b_i])$ é a \ColorRed{imagem

de um intervalo}. Temos:
%
$$\begin{array}{rcl}
  f(A) &=& \setofst{f(a)}{a∈A} \\
  f(\{7,8,9\}) &=& \setofst{f(a)}{a∈\{7,8,9\}} \\
               &=& \{f(7), f(8), f(9)\} \\
  \end{array}
$$

% (find-LATEXgrep "grep --color=auto -niH --null -e imagens 202*.tex")

}\anothercol{
}}

% (c2m222p2p 4 "questao-3")
% (c2m222p2a   "questao-3")
% (find-dmirandacalcpage 212 "7.2. Integral definida")
% (find-leitholdptpage (+ 17 324) "5.5. A integral definida")
% Veja: \Ca{Miranda212}, \Ca{Leit5p41} (p.324, seção 5.5).

% 2fT125


\newpage

%  ____        __   _       _                       _ 
% |  _ \  ___ / _| (_)_ __ | |_ ___  __ _ _ __ __ _| |
% | | | |/ _ \ |_  | | '_ \| __/ _ \/ _` | '__/ _` | |
% | |_| |  __/  _| | | | | | ||  __/ (_| | | | (_| | |
% |____/ \___|_|   |_|_| |_|\__\___|\__, |_|  \__,_|_|
%                                   |___/             
%
% «def-integral»  (to ".def-integral")
% (c2m231srp 11 "def-integral")
% (c2m231sra    "def-integral")
% (c2m222tfcsp 5 "def-integral")
% (c2m222tfcsa   "def-integral")

\vspace*{-0.3cm}

$$\scalebox{0.44}{$
  \begin{array}{rcl}
  [a,b]_N &=& \setofst{a+k(\frac{b-a}{N})}{k∈\{0,\ldots,N\}} \\
          &=& \{ a+0(\frac{b-a}{N}),
              \; a+1(\frac{b-a}{N}),
              \; \ldots,
              \; a+N(\frac{b-a}{N}) \} \\
          &=& \{ a,
              \; a + \frac{b-a}{N},
              \; a + 2\frac{b-a}{N},
              \; a + 3\frac{b-a}{N},
              \; \ldots, \; b\} \\
  \D \ga{into_P  f(x) dx} &=&    \msup_P \\[-5pt]
                          &=& \D \sum_{i=1}^{N} \sup(f([a_i,b_i])) (b_i-a_i) \\
  \D \ga{intu_P  f(x) dx} &=&    \minf_P \\[-5pt]
                          &=& \D \sum_{i=1}^{N} \inf(f([a_i,b_i])) (b_i-a_i) \\ \\[-5pt]
  \D \ga{intou_P f(x) dx} &=& \D \INTP{\into}{P}{f(x)}
                               - \INTP{\intu}{P}{f(x)} \\ \\[-5pt]
  \D \ga{into_xab  f(x) dx} &=& \D \lim_{k→∞} \ga{into_ab2k f(x) dx} \\
  \D \ga{intu_xab  f(x) dx} &=& \D \lim_{k→∞} \ga{intu_ab2k f(x) dx} \\ \\[-5pt]
  \D \ga{intou_xab f(x) dx} &=& \D \ga{into_xab f(x) dx}
                                 - \ga{intu_xab f(x) dx} \\ \\[-5pt]
  \D \left( \ga{int_xab f(x) dx} \text{\;\;existe} \right)
                             &=& \D \left( \ga{into_xab  f(x) dx}
                                         = \ga{intu_xab  f(x) dx} \right)     \\ \\[-7pt]
                             &=& \D \left( \ga{intou_xab f(x) dx} = 0 \right) \\ \\[-7pt]
  \D \ga{int_xab f(x) dx}    &=& \D \ga{into_xab f(x) dx}
                                 \qquad \text{(se a integral existir)} \\  \\[-7pt]
                             &=& \D \ga{intu_xab f(x) dx}
                                 \qquad \text{(se a integral existir)} \\
  \end{array}
  $}
$$



\newpage

% «particoes-exercs»  (to ".particoes-exercs")
% (c2m231srp 11 "particoes-exercs")
% (c2m231sra    "particoes-exercs")
% (c2m212somas1p 9 "particoes")
% (c2m212somas1a   "particoes")

{\bf Exercícios sobre partições}

\scalebox{0.55}{\def\colwidth{10cm}\firstcol{



a) Converta esta ``partição''
%
$$[4,12] = [4,5]∪[5,6]∪[6,9]∪[9,10]∪[10,12]$$

para uma tabela. Neste caso quem são $a$, $b$ e $N$?

\bsk

b) Seja $P=\{2.5,3,4,6,10\}$.

Converta $P$ para o ``formato tabela'' e para o

``formato união de subintervalos'', que é este aqui:
%
$$[a,b] = [a_1,b_1]∪\ldots∪[a_N,b_N].$$


\msk

c) Seja $P=\{4,2,1,1.5\}$.

Interprete $P$ como uma partição. Diga quem são o $N$,

o $a$ e o $b$ dela e monte a tabela dos subintervalos dela.

\msk

d) Seja $P=[2,4]_6$.

Diga quem são os pontos da partição $P$.

\msk

e) Seja $P=[2,5]_{2^3}$.

Diga quem são os pontos da partição $P$.

}\anothercol{

% «dica-simplificacao»  (to ".dica-simplificacao")
% (c2m231srp 12 "dica-simplificacao")
% (c2m231sra    "dica-simplificacao")
% (c2m211somas1p 16 "exercicio-9-dicas")
% (c2m211somas1a    "exercicio-9-dicas")

{\bf Uma dica sobre simplificação}

No Ensino Médio às vezes convencem a gente de que uma fração como
$\frac64$ \ColorRed{\underline{\underline{tem}}} que ser simplificada
pra $\frac32$, mas se a gente tem que listar uma sequência de números
começando em 0 em que cada número novo é o anterior mais $\frac14$ eu
acho bem melhor escrever essa sequência como
%
$$\frac04, \frac14, \frac24, \frac34, \frac44, \frac54, \frac64, \ldots$$
%
do que como:
%
$$0, \frac14, \frac12, \frac34, 1, \frac54, \frac32, \ldots$$

\bsk

Lembre destes trechos da Dica 7: \Ca{2gT4}

% (c2m231introp 3 "releia-a-dica-7")
% (c2m231introa   "releia-a-dica-7")

``Uma solução bem escrita é fácil de ler e fácil de verificar'', e
``Se as outras pessoas acharem que ler a sua solução é um sofrimento,
isso é mau sinal; se as outras pessoas acharem que a sua solução está
claríssima e que elas devem estudar com você, isso é bom sinal''.

}}

\newpage

% (c2m231introp 11 "imagens-de-intervalos")
% (c2m231introa    "imagens-de-intervalos")


\newpage

%  ___                                           __ _           
% |_ _|_ __ ___   __ _  __ _  ___ _ __  ___ _   / _(_) __ _ ___ 
%  | || '_ ` _ \ / _` |/ _` |/ _ \ '_ \/ __(_) | |_| |/ _` / __|
%  | || | | | | | (_| | (_| |  __/ | | \__ \_  |  _| | (_| \__ \
% |___|_| |_| |_|\__,_|\__, |\___|_| |_|___(_) |_| |_|\__, |___/
%                      |___/                          |___/     
%
% «imagens-figuras»  (to ".imagens-figuras")
% (c2m231srp 13 "imagens-figuras")
% (c2m231sra    "imagens-figuras")
% (c2m221somas3p 5 "imagens-figuras")
% (c2m221somas3a   "imagens-figuras")

% (find-angg "LUA/Piecewise1.lua" "Xtoxytoy-test3")

%L Pict2e.bounds = PictBounds.new(v(0,0), v(8,4))
%L 
%L cthick = "2pt"     -- curve
%L dthick = "0.25pt"  -- dots
%L sthick = "4pt"     -- segments
%L 
%L -- Curve:
%L cspec   = "(0,2)--(2,4)--(6,0)--(8,2)"
%L cpws    = PwSpec.from(cspec)
%L curve   = cpws:topict():prethickness(cthick)
%L 
%L -- Segments:
%L sspec = "(1,0)c--(2,0)--(4,0)c" ..
%L        " (1,3)c--(2,4)--(4,2)c" ..
%L        " (0,2)c--(0,4)c"
%L spws   = PwSpec.from(sspec)
%L segs   = spws:topict():prethickness(sthick):Color("Orange")
%L 
%L -- Dots:
%L dotsn = function (nsubsegs)
%L     local xs    = seqn(1, 4, nsubsegs)
%L     local dots0 = Xtoxytoy.from(cpws:fun(), xs)
%L     local dots  = dots0:topict("vhxpy"):prethickness(dthick):Color("Red")
%L     return dots
%L   end
%L 
%L PictList { curve, dotsn(1)  } :pgat("pgatc", "ImageOne")   :output()
%L PictList { curve, dotsn(3)  } :pgat("pgatc", "ImageThree") :output()
%L PictList { curve, dotsn(6)  } :pgat("pgatc", "ImageSix")   :output()
%L PictList { curve, dotsn(12) } :pgat("pgatc", "ImageTwelve"):output()
%L PictList { curve, segs }      :pgat("pgatc", "ImageSegs")  :output()
\pu



\newpage

% «imagens-de-intervalos»  (to ".imagens-de-intervalos")
% (c2m231srp 13 "imagens-de-intervalos")
% (c2m231sra    "imagens-de-intervalos")
% (c2m221somas3p 6 "imagens-figuras-2")
% (c2m221somas3a   "imagens-figuras-2")

{\bf Imagens de intervalos}

\scalebox{0.5}{\def\colwidth{12.5cm}\firstcol{

{}

Se $f:\R→\R$ então em princípio a expressão $f(\{7,8,9\})$ deveria dar
um erro, porque $f$ é uma função que espera receber um número, e
$\{7,8,9\}$ é um conjunto... mas aí normalmente a gente define que o
comportamento da $f$ quando ela recebe um conjunto vai ser este aqui:
%
$$f(A) \;\;=\;\; \setofst{f(a)}{a∈A}$$

A gente diz que $f(A)$ é \ColorRed{a imagem do conjunto $A$}.

\bsk

Algumas pessoas -- como o Carlos, aqui: \Ca{2gT12} --

acham que isto é sempre verdade:
%
$$f([a,b]) = [f(a),f(b)].$$

\standout{Não seja como o Carlos!!! Seja como o Bob!!!}

\bsk

Nas figuras à direita temos:
%
$$\begin{array}{rcl}
  f(\{1,4\}) &=& \{f(1),f(4)\} \\
             &=& \{3,2\} \\
             &=& \{2,3\} \\
  f(\{1,2,3,4\}) &=& \{f(1),f(2),f(3),f(4)\} \\
                 &=& \{2,3,4,3\} \\
                 &=& \{2,3,4\} \\
  f([1,4]) &=& [2,4] \\{}
  [f(1),f(4)] &=& [3,2] \\
              &=& \setofst{y∈\R}{3≤y≤2} \\
              &=& ∅ \\
              &≠& f([1,4]) \\
  \end{array}
$$

}\anothercol{

\vspace*{0cm}

$\phantom{mmmm}
 \scalebox{2}{$
  \begin{array}{l}
    {\ImageOne}    \\
    {\ImageThree}  \\
    {\ImageSix}  \\
    {\ImageTwelve} \\
    {\ImageSegs}   \\
  \end{array}
 $}
$

}}



% (c2m222srp 14 "exercicio-7")
% (c2m222sra    "exercicio-7")

\newpage


%  ___                                                               
% |_ _|_ __ ___   __ _  __ _  ___ _ __  ___    _____  _____ _ __ ___ 
%  | || '_ ` _ \ / _` |/ _` |/ _ \ '_ \/ __|  / _ \ \/ / _ \ '__/ __|
%  | || | | | | | (_| | (_| |  __/ | | \__ \ |  __/>  <  __/ | | (__ 
% |___|_| |_| |_|\__,_|\__, |\___|_| |_|___/  \___/_/\_\___|_|  \___|
%                      |___/                                         
%
% «imagens-exercicio»  (to ".imagens-exercicio")
% (c2m231srp 14 "imagens-exercicio")
% (c2m231sra    "imagens-exercicio")
% (c2m221somas3p 7 "exercicio-2")
% (c2m221somas3a   "exercicio-2")

{\bf Imagens de intervalos: exercício}

\scalebox{0.85}{\def\colwidth{6.5cm}\firstcol{

Seja $f(x)$ esta função:

\msk

%L Pict2e.bounds = PictBounds.new(v(0,0), v(8,4))
%L spec   = "(0,2)--(2,4)--(6,0)--(8,2)"
%L pws    = PwSpec.from(spec)
%L curve  = pws:topict()
%L p = PictList { curve:prethickness("2pt") }
%L p:pgat("pgatc", "falsoseno"):output()
\pu
%
$f(x) = \falsoseno$

\msk

Calcule estas imagens de intervalos:

\msk

\begin{tabular}[t]{l}
a) $f([0,1])$ \\
b) $f([1,2])$ \\
c) $f([0,2])$ \\
d) $f([2,3])$ \\
e) $f([1,3])$ \\
f) $f([0,3])$ \\
g) $f([0,4])$ \\
h) $f([4,8])$ \\
i) $f([0,8])$ \\
j) $f([1,7])$ \\
\end{tabular}
\qquad
\begin{tabular}[t]{l}
a') $f((0,1))$ \\
b') $f((1,2))$ \\
c') $f((0,2))$ \\
d') $f((2,3))$ \\
e') $f((1,3))$ \\
f') $f((0,3))$ \\
g') $f((0,4))$ \\
h') $f((4,8))$ \\
i') $f((0,8))$ \\
j') $f((1,7))$ \\
\end{tabular}

}\anothercol{

Dicas:

\msk

Faça os itens (a) até (j) primeiro. Os itens (a') até (j') são bem
mais difíceis, e em alguns deles os resultados vão ser conjuntos
fechados ou ``semi-abertos''.

\msk

O Leithold define intervalos

semi-abertos aqui: \Ca{Leit1p7}

\msk

Daqui a pouco nós vamos ver um modo de testar as respostas dos itens
desse exercício, e um modo de resolver ele por chutar e testar... mas
aguente um pouquinho!


}}

\newpage

%  ____                            _   _                   
% |  _ \  ___  ___  ___ ___  _ __ | |_(_)_ __  _   _  __ _ 
% | | | |/ _ \/ __|/ __/ _ \| '_ \| __| | '_ \| | | |/ _` |
% | |_| |  __/\__ \ (_| (_) | | | | |_| | | | | |_| | (_| |
% |____/ \___||___/\___\___/|_| |_|\__|_|_| |_|\__,_|\__,_|
%                                                          
% «descontinua»  (to ".descontinua")
% (c2m231srp 15 "descontinua")
% (c2m231sra    "descontinua")
% (c2m221isp 12 "exercicio-5")
% (c2m221isa    "exercicio-5")

{\bf Agora uma função descontínua}

%L Pict2e.bounds = PictBounds.new(v(0,0), v(9,7))
%L spec   = "(0,3)--(2,1)o (2,3)c (2,5)o--(7,0)"
%L pws    = PwSpec.from(spec)
%L curve  = pws:topict()
%L p = PictList { curve:prethickness("2pt") }
%L p:addputstrat(v(2.7,5.5), "\\cell{(2,5)}")
%L p:addputstrat(v(7.7,0.5), "\\cell{(7,0)}")
%L p:pgat("pgatc"):preunitlength("17pt"):sa("Exercicio 5"):output()
\pu

\msk

Sejam
%
$f(x) = \scalebox{0.5}{$\ga{Exercicio 5}$}$

e $B=[1,3]$.

\msk

Represente graficamente estes conjuntos ---

as definições deles são as mesmas do slide 3:
%
$$\begin{array}{rcl}
  % \Rext &=& \R∪\{-∞,+∞\} \\
  C  &=& \setofst{(x,f(x))}{x∈B} \\
  D  &=& \setofst{f(x)}{x∈B} \\
  D' &=& \setofst{y∈\R}{∃x∈B.\ f(x)=y} \\
  L &=& \setofst{y∈\Rext}{∀d∈D.\;y≤d} \\
  U &=& \setofst{y∈\Rext}{∀d∈D.\;d≤y} \\
  % (α=\inf(D)) &=& α∈L ∧ (∀ℓ∈L.\;ℓ \le α) \\
  % (β=\sup(D)) &=& β∈U ∧ (∀u∈U.\;β \le u) \\
  \end{array}
$$


% (c2m222tfcsp 3 "def-particao")
% (c2m222tfcsa   "def-particao")
% \Ca{2fT91} A definição de partição
% \Ca{2fT93} A definição do $[a,b]_n$
% \Ca{2fT94} Alguns exercícios sobre partições

% 2fT73

%%%%%%%%%%%%%%%%%%%%%%%%%%%%%%%%%%%%%%%%
%%%%%%%%%%%%%%%%%%%%%%%%%%%%%%%%%%%%%%%%
%%%%%%%%%%%%%%%%%%%%%%%%%%%%%%%%%%%%%%%%
%%%%%%%%%%%%%%%%%%%%%%%%%%%%%%%%%%%%%%%%
%%%%%%%%%%%%%%%%%%%%%%%%%%%%%%%%%%%%%%%%
%%%%%%%%%%%%%%%%%%%%%%%%%%%%%%%%%%%%%%%%
%%%%%%%%%%%%%%%%%%%%%%%%%%%%%%%%%%%%%%%%
%%%%%%%%%%%%%%%%%%%%%%%%%%%%%%%%%%%%%%%%
%%%%%%%%%%%%%%%%%%%%%%%%%%%%%%%%%%%%%%%%
%%%%%%%%%%%%%%%%%%%%%%%%%%%%%%%%%%%%%%%%

\newpage

% «acima-e-abaixo»  (to ".acima-e-abaixo")
% (c2m222srp 15 "acima-e-abaixo")
% (c2m222sra    "acima-e-abaixo")

{\bf Retângulos acima e abaixo}

\scalebox{0.9}{\def\colwidth{12cm}\firstcol{

Lembre que eu contei que em cursos tradicionais de Cálculo 2 --
aqueles em que as pessoas passam centenas de horas fazendo contas à
mão, e mais outras centenas de horas estudando por aqueles livros que
fingem que certas coisas dificílimas são óbvias -- as pessoas acabam
aprendendo algumas coisas super úteis que não aparecem listadas
explicitamente no programa do curso...

\msk

Uma dessas coisas é aprender a entender definições que {\sl
  aparentemente} envolvem um número infinito de contas. Se a gente for
como o Bob a gente consegue visualizar o que essas definições ``querem
dizer''.

\msk

As definições formais de ``retângulo acima (ou abaixo) da curva'' e
``melhor retângulo acima (ou abaixo) da curva'' são assim -- elas
aparentemente precisam de infinitas contas.

}\anothercol{
}}

\newpage

% «para-todo-e-existe»  (to ".para-todo-e-existe")
% (c2m222srp 16 "para-todo-e-existe")
% (c2m222sra    "para-todo-e-existe")
% (c2m212somas2p 14 "para-todo-e-existe")
% (c2m212somas2a    "para-todo-e-existe")

{\bf ``Para todo'' ($∀$) e ``existe'' ($∃$)}

\msk

$\scalebox{0.9}{$
  \begin{array}{rcl}
  (∀a∈\{2,3,5\}.a^2<10) &=& (a^2<10)[a:=2] \;∧ \\&&
                            (a^2<10)[a:=3] \;∧ \\&&
                            (a^2<10)[a:=5] \\
                        &=& (2^2<10) ∧
                            (3^2<10) ∧
                            (5^2<10) \\
                        &=& (4<10) ∧
                            (9<10) ∧
                            (25<10) \\
                        &=& \V ∧ \V ∧ \F \\
                        &=& \F \\[5pt]
  (∃a∈\{2,3,5\}.a^2<10) &=& (a^2<10)[a:=2] \;∨ \\&&
                            (a^2<10)[a:=3] \;∨ \\&&
                            (a^2<10)[a:=5] \\
                        &=& (2^2<10) ∨
                            (3^2<10) ∨
                            (5^2<10) \\
                        &=& (4<10) ∨
                            (9<10) ∨
                            (25<10) \\
                        &=& \V ∨ \V ∨ \F \\
                        &=& \V \\
  \end{array}
 $}
$

\newpage

% «visualizando-fas-e-exs»  (to ".visualizando-fas-e-exs")
% (c2m222srp 17 "visualizando-fas-e-exs")
% (c2m222sra    "visualizando-fas-e-exs")
% (c2m212somas2p 15 "visualizando-fas-e-exs")
% (c2m212somas2a    "visualizando-fas-e-exs")
% (c2m211substp 24 "visualizando-fas-e-exs")
% (c2m211substa    "visualizando-fas-e-exs")

{\bf Visualizando `$∀$'s e `$∃$'s}

Repare...

\msk

{
\def\V    {\mathbf{V}}
\def\F    {\mathbf{F}}
\def\mbc#1{\hbox to 8pt{\hss$#1$\hss}}
\def\V    {\mbc{\mathbf{V}}}
\def\F    {\mbc{\mathbf{F}}}

$\scalebox{0.9}{$
  \begin{array}{lcl}
  (∀x∈\{1,\ldots,7\}.2≤x)            &=& \F∧\V∧\V∧\V∧\V∧\V∧\V \\
  (∀x∈\{1,\ldots,7\}.\ph{mm}x<4)     &=& \V∧\V∧\V∧\F∧\F∧\F∧\F \\
  (∀x∈\{1,\ldots,7\}.2≤x<4)          &=& \F∧\V∧\V∧\F∧\F∧\F∧\F \\
  (∀x∈\{1,\ldots,7\}.\ph{mmmmmm}x=6) &=& \F∧\F∧\F∧\F∧\F∧\V∧\F \\
  (∀x∈\{1,\ldots,7\}.2≤x<4∨     x=6) &=& \F∧\V∧\V∧\F∧\F∧\V∧\F \\
  \end{array}
  $}
$
}

\msk

...que dá pra {\sl visualizar} o que a expressão

$(∀x∈\{1,\ldots,7\}.2≤x<4∨x=6)$

``quer dizer'' visualizando os `$\V$'s e `$\F$'s

de expressões mais simples, e combinando

esses ``mapas'' de `$\V$'s e `$\F$'s.

\newpage

% «visualizando-fas-e-exs-2»  (to ".visualizando-fas-e-exs-2")
% (c2m222srp 18 "visualizando-fas-e-exs-2")
% (c2m222sra    "visualizando-fas-e-exs-2")
% (c2m212somas2p 16 "visualizando-fas-e-exs-2")
% (c2m212somas2a    "visualizando-fas-e-exs-2")
% (c2m211substp 20 "visualizando-fas-e-exs-2")
% (c2m211substa    "visualizando-fas-e-exs-2")

{\bf Visualizando `$∀$'s e `$∃$'s (2)}

Às vezes vai valer a pena \ColorRed{definir proposições}

como nomes mais curtos, como $F(x) = (2≤x)$,

$G(x) = (x≤4)$, $H(x) = (x=6)$... Aí:

\msk

{
\def\mbc#1{\hbox to 8pt{\hss$#1$\hss}}
\def\V    {\mbc{\mathbf{V}}}
\def\F    {\mbc{\mathbf{F}}}

$\scalebox{0.9}{$
  \begin{array}{lcl}
  (∀x∈\{1,\ldots,7\}.F(x))              &=& \F∧\V∧\V∧\V∧\V∧\V∧\V \\
  (∀x∈\{1,\ldots,7\}.\ph{mmmii}G(x))    &=& \V∧\V∧\V∧\F∧\F∧\F∧\F \\
  (∀x∈\{1,\ldots,7\}.F(x)∧G(x))         &=& \F∧\V∧\V∧\F∧\F∧\F∧\F \\
  (∀x∈\{1,\ldots,7\}.\ph{mmmmmmmi}H(x)) &=& \F∧\F∧\F∧\F∧\F∧\V∧\F \\
  (∀x∈\{1,\ldots,7\}.F(x)∧G(x)∨ H(x))   &=& \F∧\V∧\V∧\F∧\F∧\V∧\F \\
  \end{array}
  $}
$
}

\msk

É isso que a gente vai fazer pra analisar expressões

como $(∀x∈A.▁▁▁)$ e $(∃x∈A.▁▁▁)$ e descobrir quais

são verdadeiras e quais não --- \ColorRed{mesmo quando o conjunto

$A$ é um conjunto infinito}, como $\N$, $\R$ ou $[2,10]$.


\newpage

% «visualizando-fas-e-exs-3»  (to ".visualizando-fas-e-exs-3")
% (c2m222srp 19 "visualizando-fas-e-exs-3")
% (c2m222sra    "visualizando-fas-e-exs-3")
% (c2m212somas2p 17 "visualizando-fas-e-exs-3")
% (c2m212somas2a    "visualizando-fas-e-exs-3")
% (c2m211substp 26 "visualizando-fas-e-exs-3")
% (c2m211substa    "visualizando-fas-e-exs-3")

{\bf Visualizando `$∀$'s e `$∃$'s (3)}

\scalebox{0.8}{\def\colwidth{12cm}\firstcol{

Às vezes vamos ter que fazer figuras com muitos `$\V$'s e `$\F$'s,

e vai ser mais fácil visualizar onde estão os `$\V$'s e `$\F$'s

delas se usarmos sinais mais fáceis de distinguir...

\msk

Vou usar essa convenção aqui:

O $\V$ é uma bolinha preta, ou sólida: $•$

O $\F$ é uma bolinha branca, ou oca: $∘$

\msk

{
\def\mbc#1{\hbox to 8pt{\hss$#1$\hss}}
\def\V    {\mbc{\mathbf{V}}}
\def\V    {\mbc{•}}
\def\F    {\mbc{∘}}

$\scalebox{0.9}{$
  \begin{array}{lcl}
  (∀x∈\{1,\ldots,7\}.F(x))              &=& \F∧\V∧\V∧\V∧\V∧\V∧\V \\
  (∀x∈\{1,\ldots,7\}.\ph{mmmii}G(x))    &=& \V∧\V∧\V∧\F∧\F∧\F∧\F \\
  (∀x∈\{1,\ldots,7\}.F(x)∧G(x))         &=& \F∧\V∧\V∧\F∧\F∧\F∧\F \\
  (∀x∈\{1,\ldots,7\}.\ph{mmmmmmmi}H(x)) &=& \F∧\F∧\F∧\F∧\F∧\V∧\F \\
  (∀x∈\{1,\ldots,7\}.F(x)∧G(x)∨ H(x))   &=& \F∧\V∧\V∧\F∧\F∧\V∧\F \\
  \end{array}
  $}
$
}

\bsk

Você \ColorRed{pode} fazer as suas próprias definições ---

como o meu ``$•:=\V$ e $∘:=\F$'' acima --- mas elas

\standout{têm} que ficar claras o suficiente... releia a dica 7:

\ssk

{\footnotesize

% (c2m212introp 3 "dica-7")
% (c2m212introa   "dica-7")
%    http://angg.twu.net/LATEX/2021-2-C2-intro.pdf#page=3
\url{http://angg.twu.net/LATEX/2021-2-C2-intro.pdf\#page=3}

}

}\anothercol{
}}

\newpage

%  ___           _                  _                 _       __     
% |_ _|_ __  ___| |_ _ __ ___    __| | ___  ___    __| | ___ / _|___ 
%  | || '_ \/ __| __| '__/ __|  / _` |/ _ \/ __|  / _` |/ _ \ |_/ __|
%  | || | | \__ \ |_| |  \__ \ | (_| |  __/\__ \ | (_| |  __/  _\__ \
% |___|_| |_|___/\__|_|  |___/  \__,_|\___||___/  \__,_|\___|_| |___/
%                                                                    
% «instrucoes-des-defs»  (to ".instrucoes-des-defs")
% (c2m231srp 21 "instrucoes-des-defs")
% (c2m231sra    "instrucoes-des-defs")
%
%L Pict2e.bounds = PictBounds.new(v(0,0), v(7,5))
%L spec   = "(0,2)--(2,4)--(5,1)--(7,3)"
%L pws    = PwSpec.from(spec)
%L curve  = pws:topict()
%L p = PictList { curve:prethickness("2pt") }
%L p:pgat("pgatc", "falsoseno"):output()
\pu
%
\sa{Color A}{\ColorRed}
\sa{Color B}{\ColorOrange}
\sa{Color C}{\ColorGreen}
\def\COLOR#1#2{\ga{Color #1}{#2}}
\def\undem#1#2{\underbrace{#1}_{\text{em }#2}}
\def\undemc#1#2#3{\underbrace{#2}_{\COLOR{#1}{\text{em }#3}}}
%
\def\fx #1{f(\undemc{A}{\mathstrut #1}{(#1,0)})}
\def\Fx #1{  \undemc{A}{\mathstrut #1}{(#1,0)} }
\def\fxy#1#2{\undemc{B}{\fx{#1}<#2}{(#1,f(#1))}}
\def\fafxy#1{\undemc{C}{∀x∈\{1,2,3\}. \fxy{x}{#1}}{(0,#1)}}
\def\LAND{\;\;∧\;\;}

\newpage

%  ___           _                  _           
% |_ _|_ __  ___| |_ _ __ ___    __| | ___  ___ 
%  | || '_ \/ __| __| '__/ __|  / _` |/ _ \/ __|
%  | || | | \__ \ |_| |  \__ \ | (_| |  __/\__ \
% |___|_| |_|___/\__|_|  |___/  \__,_|\___||___/
%                                               
% «instrucoes-des-1»  (to ".instrucoes-des-1")
% (c2m231srp 21 "instrucoes-des-1")
% (c2m231sra    "instrucoes-des-1")
% (c2m222srp 20 "instrucoes-des-1")
% (c2m222sra    "instrucoes-des-1")

{\bf Instruções de desenho (explícitas)}

\msk

Sejam $f(x) = \falsoseno$ ,

\msk

e $P(y) \;=\; \fafxy{y} .$

\bsk

As anotações sob as chaves são ``instruções de desenho''

que o Bob vai usar pra calcular cada $P(y)$ de cabeça,

e pra visualizar o que $P(y)$ ``quer dizer''...

\ssk

Na próxima página eu fiz as figuras pra $P(3.5)$.


% (c2m221isp 5 "exercicio-1")
% (c2m221isa   "exercicio-1")

\newpage

%  ___           _                  _              __ _       
% |_ _|_ __  ___| |_ _ __ ___    __| | ___  ___   / _(_) __ _ 
%  | || '_ \/ __| __| '__/ __|  / _` |/ _ \/ __| | |_| |/ _` |
%  | || | | \__ \ |_| |  \__ \ | (_| |  __/\__ \ |  _| | (_| |
% |___|_| |_|___/\__|_|  |___/  \__,_|\___||___/ |_| |_|\__, |
%                                                       |___/ 
% «instrucoes-des-2»  (to ".instrucoes-des-2")
% (c2m231srp 22 "instrucoes-des-2")
% (c2m231sra    "instrucoes-des-2")
% (c2m222srp 21 "instrucoes-des-2")
% (c2m222sra    "instrucoes-des-2")

% (c2m221isp 2 "uma-figura")
% (c2m221isa   "uma-figura")
%
%L fromep    = PwSpec.fromep
%L thick     = function (th) return "\\linethickness{"..th.."}" end
%L
%L p = PictList {
%L   thick("1pt"),
%L   fromep(" (0,2)--(2,4)--(5,1)--(7,3)  "),
%L   thick("2pt"),
%L   fromep(" (1,0)c (2,0)c (3,0)c          "):color("red"),
%L   fromep(" (1,3)c (2,4)o (3,3)c          "):color("orange"),
%L   fromep(" (0,3.5)o                      "):Color("Green"),
%L }
%L p = (p
%L       :setbounds(v(0,0), v(7,5))
%L       :pgat("gat")
%L       :pgat("p")
%L       :preunitlength("10pt")
%L       :sa("instrucoes des")
%L     )
%L p:output()
\pu

\def\Fxy#1#2#3#4{\undemc{B}{\mathstrut #1<#2}{(#3,#4)}}
\def\Bxy#1#2#3{\undemc{B}{\mathstrut\COLOR{B}{#1}}{(#2,#3)}}

\scalebox{0.65}{\def\colwidth{11cm}\firstcol{

$\begin{array}[t]{rcl}
 P(3.5) &=& \fafxy{3.5} \\
 \\[-5pt]
 &=& \undemc{C}{ (\fxy{1}{3.5})
           \LAND (\fxy{2}{3.5})
           \LAND (\fxy{3}{3.5})}
                      {(0,3.5)} \\
 \\[-5pt]
 &=& \undemc{C}{ (\Fxy 3{3.5}13)
           \LAND (\Fxy 4{3.5}24)
           \LAND (\Fxy 3{3.5}33)}
                     {(0,3.5)} \\
 \\[-5pt]
 &=& \undemc{C}{ (\Bxy{•}{1}{3}) \LAND (\Bxy{∘}{2}{4}) \LAND (\Bxy{•}{3}{3})}
             {(0,3.5)} \\
 \\[-5pt]
 &=& \undemc{C}{ \mathstrut{\COLOR{C}{∘}} }{(0,3.5)} \\
 \end{array}
$

}\anothercol{

\vspace*{6cm}

\def\closeddot{\circle*{0.3}}
\def\opendot  {\circle*{0.3}\color{white}\circle*{0.2}}

\def\closeddot{\circle*{0.5}}
\def\opendot  {\circle*{0.5}\color{white}\circle*{0.3}}

$\scalebox{2}{$
  \ga{instrucoes des}
 $}
$

}}

\newpage

%  ___           _                  _                       
% |_ _|_ __  ___| |_ _ __ ___    __| | ___  ___    _____  __
%  | || '_ \/ __| __| '__/ __|  / _` |/ _ \/ __|  / _ \ \/ /
%  | || | | \__ \ |_| |  \__ \ | (_| |  __/\__ \ |  __/>  < 
% |___|_| |_|___/\__|_|  |___/  \__,_|\___||___/  \___/_/\_\
%                                                           
% «instrucoes-des-ex»  (to ".instrucoes-des-ex")
% (c2m231srp 23 "instrucoes-des-ex")
% (c2m231sra    "instrucoes-des-ex")
% (c2m222srp 19 "exercicio-8")
% (c2m222sra    "exercicio-8")

%L Pict2e.bounds = PictBounds.new(v(0,0), v(6,4))
%L spec   = "(0,1)--(2,3)--(4,1)--(6,3)"
%L pws    = PwSpec.from(spec)
%L curve  = pws:topict()
%L p = PictList { curve:prethickness("0.5pt") }
%L p:pgat("pgatc"):sa("instrthin"):output()
\pu

{\bf Instruções de desenho: exercício}

\scalebox{0.6}{\def\colwidth{9cm}\firstcol{

Sejam:

$\begin{array}{rcl}
 f(x) &=& \ga{instrthin}       \;, \\
 \\[-7pt]
 P(y) &=& ∀x∈\{1,2,3\}. f(x)<y \;, \\
 Q(y) &=& ∀x∈\{1,2,3\}. f(x)≤y \;, \\
 R(y) &=& ∀x∈\{1,2,3\}. f(x)≥y \;, \\
 S(y) &=& ∀x∈\{1,2,3\}. f(x)>y \;, \\
 \\[-7pt]
 P'(y) &=& ∀x∈[3,5]. f(x)<y \;, \\
 Q'(y) &=& ∀x∈[3,5]. f(x)≤y \;, \\
 R'(y) &=& ∀x∈[3,5]. f(x)≥y \;, \\
 S'(y) &=& ∀x∈[3,5]. f(x)>y \;. \\
 \end{array}
$

\bsk

Para cada uma das expressões à direita visualize-a, represente-a
graficamente numa das cópias do gráfico da $f(x)$ da próxima página, e
dê o resultado dela.

Note que aqui eu não estou dando instruções de desenho {\sl
  explícitas} -- você vai ter que escolher como você vai fazer pra
visualizar cada expressão.


}\anothercol{

a) $P(3.5), P(3.0), \ldots, P(0.5)$  

b) $Q(3.5), Q(3.0), \ldots, Q(0.5)$  

c) $R(3.5), R(3.0), \ldots, R(0.5)$  

d) $S(3.5), S(3.0), \ldots, S(0.5)$  

\msk

e) $P'(3.5), P'(3.0), \ldots, P'(0.5)$

f) $Q'(3.5), Q'(3.0), \ldots, Q'(0.5)$  

g) $R'(3.5), R'(3.0), \ldots, R'(0.5)$  

h) $S'(3.5), S'(3.0), \ldots, S'(0.5)$  

\bsk

Nos itens (e) até (f) os seus desenhos vão ter infinitas bolinhas...
aliás, você vai ter que fazer desenhos que {\sl finjam} que têm
infinitas bolinhas, e nos quais o leitor consiga entender o que você
quis representar... veja este slide antigo:

\ssk

% (c2m212somas2p 53 "dirichlet-3")
% (c2m212somas2a    "dirichlet-3")
% (c2m211somas24p 34 "que-finja-ter-infinitas")
% (c2m211somas24a    "que-finja-ter-infinitas")
%      http://angg.twu.net/LATEX/2021-1-C2-somas-2-4.pdf#page=29
% \url{http://angg.twu.net/LATEX/2021-1-C2-somas-2-4.pdf\#page=29}
%
\Ca{2dT142} ``E pra conjuntos infinitos?''


}}

\newpage

%  ___           _                  _                        _     _ 
% |_ _|_ __  ___| |_ _ __ ___    __| | ___  ___    __ _ _ __(_) __| |
%  | || '_ \/ __| __| '__/ __|  / _` |/ _ \/ __|  / _` | '__| |/ _` |
%  | || | | \__ \ |_| |  \__ \ | (_| |  __/\__ \ | (_| | |  | | (_| |
% |___|_| |_|___/\__|_|  |___/  \__,_|\___||___/  \__, |_|  |_|\__,_|
%                                                 |___/              
%
% «instrucoes-des-grid»  (to ".instrucoes-des-grid")
% (c2m231srp 24 "instrucoes-des-grid")
% (c2m231sra    "instrucoes-des-grid")
% (c2m222srp 23 "exercicio-8-figs")
% (c2m222sra    "exercicio-8-figs")

\def\IT{\ga{instrthin}}
\def\ITS{\IT & \IT & \IT & \IT & \IT & \IT & \IT & \IT & \IT }

$\scalebox{0.6}{$
 \begin{matrix}
 \ITS \\
 \ITS \\
 \ITS \\
 \ITS \\
 \ITS \\
 \ITS \\
 \ITS \\
 \ITS \\
 \ITS \\
 \end{matrix}
 $}
$


\newpage

% «instrucoes-des-ex-2»  (to ".instrucoes-des-ex-2")
% (c2m231srp 25 "instrucoes-des-ex-2")
% (c2m231sra    "instrucoes-des-ex-2")
% (c2m222srp 24 "exercicio-9")
% (c2m222sra    "exercicio-9")

{\bf Instruções de desenho: outro exercício}

\scalebox{0.6}{\def\colwidth{10cm}\firstcol{

A seção ``Mais sobre bolinhas'' daqui:

\ssk

{\scriptsize

% (c2m212somas2p 53 "dirichlet-3")
% (c2m212somas2a    "dirichlet-3")
% (c2m211somas24p 29 "mais-sobre-bolinhas")
% (c2m211somas24a    "mais-sobre-bolinhas")
%    http://angg.twu.net/LATEX/2021-1-C2-somas-2-4.pdf#page=29
\url{http://angg.twu.net/LATEX/2021-1-C2-somas-2-4.pdf\#page=29}

}

\ssk

tem dicas sobre como visualizar subconjuntos

``definidos por proposições'', como este aqui:
%
$$\setofst{x∈A}{P(a)}$$

A gente primeiro marca cada ponto de $A$ com uma

bolinha ou preta ou branca, e depois a gente pega

o conjunto das bolinhas pretas e interpreta ele

como um outro conjunto -- o resultado.

\msk

Use isto pra visualizar cada um dos conjuntos

à direita e pra encontrar uma descrição mais simples

para cada um deles. Geralmente essas ``descrições

mais simples'' vão ser em notação de intervalos.

\msk

As funções $P, \ldots, S, P', \ldots, S'$ são as do exercício 8.

O símbolo $\Rext$ denota a ``reta real estendida'':
%
$$\begin{array}{rcl}
  \Rext &=& \R ∪ \{-∞,+∞\} \\
        &=& (-∞,+∞) ∪ \{-∞,+∞\} \\
        &=& [-∞,+∞] \\
  \end{array}
$$

Para mais detalhes, veja:

{\scriptsize

% https://en.wikipedia.org/wiki/Extended_real_number_line
\url{https://en.wikipedia.org/wiki/Extended_real_number_line}

}



}\anothercol{

a) $\setofst{y∈[0,3]}{P(y)}$

b) $\setofst{y∈[0,3]}{Q(y)}$

c) $\setofst{y∈[0,3]}{R(y)}$

d) $\setofst{y∈[0,3]}{S(y)}$

\msk

a') $\setofst{y∈[0,3]}{P'(y)}$

b') $\setofst{y∈[0,3]}{Q'(y)}$

c') $\setofst{y∈[0,3]}{R'(y)}$

d') $\setofst{y∈[0,3]}{S'(y)}$

\msk

e) $\setofst{y∈\R}{P(y)}$

f) $\setofst{y∈\R}{Q(y)}$

g) $\setofst{y∈\R}{R(y)}$

h) $\setofst{y∈\R}{S(y)}$

\msk

i) $\setofst{y∈\Rext}{P(y)}$

j) $\setofst{y∈\Rext}{Q(y)}$

k) $\setofst{y∈\Rext}{R(y)}$

l) $\setofst{y∈\Rext}{S(y)}$


}}

\newpage

%  ___       _                 ___       _         
% |_ _|_ __ | |_ ___     ___  |_ _|_ __ | |_ _   _ 
%  | || '_ \| __/ _ \   / _ \  | || '_ \| __| | | |
%  | || | | | || (_) | |  __/  | || | | | |_| |_| |
% |___|_| |_|\__\___/   \___| |___|_| |_|\__|\__,_|
%                                                  
% «into-e-intu»  (to ".into-e-intu")
% (c2m231srp 26 "into-e-intu")
% (c2m231sra    "into-e-intu")
% (c2m222tfcsp 8 "exercicio-3")
% (c2m222tfcsa   "exercicio-3")

{\bf Aproximações por cima e por baixo}

%L Pict2e.bounds = PictBounds.new(v(0,0), v(9,7))
%L spec   = "(0,3)--(2,1)o (2,3)c (2,5)o--(7,0)"
%L pws    = PwSpec.from(spec)
%L curve  = pws:topict()
%L p = PictList { curve:prethickness("2pt") }
%L p:addputstrat(v(2.7,5.5), "\\cell{(2,5)}")
%L p:addputstrat(v(7.7,0.5), "\\cell{(7,0)}")
%L p:pgat("pgatc"):preunitlength("17pt"):sa("Exercicio 2"):output()
\pu

\scalebox{0.8}{\def\colwidth{8cm}\firstcol{

\vspace*{0cm}

Sejam
%
$f(x) = \scalebox{0.5}{$\ga{Exercicio 2}$}$

\msk

e $P=\{1,3,4,5\}$,

e \ColorRed{por enquanto} considere que:
%
$$\begin{array}{rcl}
  \sup(f(B)) &=& \max_{x∈B} f(x) \quad \text{e} \\
  \inf(f(B)) &=& \min_{x∈B} f(x).
  \end{array}
$$

\msk


}\anothercol{

Represente graficamente:

\msk

a) $\ga{into_P  f(x) dx}$

\msk

b) $\ga{intu_P  f(x) dx}$

\msk

c) $\ga{intou_P f(x) dx}$

\msk

d) $\INTP{\intou}{[1,5]_2}{f(x)}$

\msk

e) $\INTP{\intou}{[1,5]_4}{f(x)}$

}}


\newpage

% «um-jogo»  (to ".um-jogo")
% (c2m231srp 27 "um-jogo")
% (c2m231sra    "um-jogo")

% (c2m221isp 6 "exercicio-2")
% (c2m221isa   "exercicio-2")
% (c2m221isp 10 "exercicio-2-dica")
% (c2m221isa    "exercicio-2-dica")

% (c2m221isp 10 "exercicio-2-dica")
% (c2m221isa    "exercicio-2-dica")

{\bf Um jogo colaborativo}

%L Pict2e.bounds = PictBounds.new(v(0,0), v(2,2))
%L spec   = [[ (0,0)--(0,1)--(1,1)--(1,0)--(0,0)--(0,1)
%L          ]]
%L pws    = PwSpec.from(spec)
%L curve  = pws:topict()
%L p = PictList {
%L     [[ \def\closeddot{\circle*{0.4}}% ]],
%L     [[ \def\opendot  {\circle*{0.4}\color{white}\circle*{0.3}}% ]],
%L     Pict2e.region0(v(0,0), v(1,0), v(1,1), v(0,1)):Color("Orange"),
%L     -- curve:prethickness("3pt")
%L  }
%L p:pgat("pgatc"):preunitlength("10pt"):sa("Prop A'"):output()
%L
%L Pict2e.bounds = PictBounds.new(v(0,0), v(2,2))
%L spec   = [[ (0,0)--(0,1)--(1,1)--(1,0)--(0,0)--(0,1)
%L          ]]
%L pws    = PwSpec.from(spec)
%L curve  = pws:topict()
%L p = PictList {
%L     [[ \def\closeddot{\circle*{0.4}}% ]],
%L     [[ \def\opendot  {\circle*{0.4}\color{white}\circle*{0.3}}% ]],
%L     Pict2e.region0(v(0,0), v(1,0), v(1,1), v(0,1)):Color("Orange"),
%L     curve:prethickness("2pt")
%L  }
%L p:pgat("pgatc"):preunitlength("10pt"):sa("Prop A''"):output()
\pu

\scalebox{0.52}{\def\colwidth{10.5cm}\firstcol{

...ou: como debugar representações gráficas.

Pense num jogo colaborativo. Os jogadores se chamam $P$
(``proponente''), e $O$ (``oponente''). O $P$ quer encontrar uma
representação gráfica pro conjunto $A$, e à primeira vista o $O$ quer
mostrar que o $P$ está errado... mas na verdade o objetivo dos dois é
fazer com que o $P$ chegue numa representação gráfica que não tem erro
nenhum.

\msk

Digamos que
%
$$A = \setofxyst{x∈[1,2), \; y∈[1,2)}.$$

O $P$ desenha uma representação gráfica \ColorRed{com um nome
  diferente de $A$} e ``propõe'' ela --- por exemplo, o $P$ diz isso
aqui:
%
\pu
%
$$A' = \ga{Prop A'}$$

O oponente $O$ diz: ``verifica o ponto $(1,1)$''. Os dois verificam o
ponto $(1,1)$ do $A'$ e vêem que o desenho do $A'$ é ambíguo no ponto
$(1,1)$, já que esse é um ponto de fronteira e o $P$ não desenhou ele
nem como linha grossa sólida nem com linha tracejada... então a
resposta pra pergunta ``$(1,1)∈A'$?'' não é nem $\True$ nem $\False$,
é ``erro'', e portanto $A≠A'$, e o $P$ ainda não conseguiu a
representação gráfica certa. O oponente $O$ ganha essa rodada, e o $P$
tem que propôr outra representação gráfica.

}\anothercol{

  Aí o $P$ propõe uma outra representação gráfica, \ColorRed{com um
    outro nome, diferente de $A$ e de $A'$}. Por exemplo, $P$ propõe
  isso aqui:
%
$$A'' = \ga{Prop A''}$$

O oponente $O$ diz: ``verifica o ponto $(0,0)$''. Os dois verificam, e
vêem que:
%
$$(0,0)\not∈A, \quad (0,0)∈A''$$

E portanto $A≠A''$, e o $P$ ainda não conseguiu a representação
gráfica certa. O oponente $O$ ganha mais essa rodada.

\bsk

Quando o $P$ propõe um desenho que o $O$ não consegue mostrar que está
errado o $P$ ganha a rodada.

\bsk

Até vocês terem prática vocês vão jogar como o $P$, vão me mostrar as
representações gráficas de vocês, e eu vou jogar como o $O$. Quando
vocês tiverem mais prática vocês vão conseguir chutar representações
gráficas (como o jogador $P$) e testá-las (fazendo o papel do jogador
$O$ vocês mesmos).

}}


\newpage

% «um-jogo-2»  (to ".um-jogo-2")
% (c2m221isp 9 "exercicio-2")
% (c2m221isa   "exercicio-2")
% (find-pdf-page "~/2022.1-C2/C2-quadros-manha.pdf" 10)
% (c2m212somas24p 4 "subconjunto-do-plano")
% (c2m212somas24a   "subconjunto-do-plano")

%L Pict2e.bounds = PictBounds.new(v(-1,-1), v(4,3))
%L spec   = [[ (0,0)--(0,2)--(2,2)
%L             (0.3,0)--(0.75,0) (1.25,0)--(1.7,0)
%L             (2,0.3)--(2,0.75) (2,1.25)--(2,1.7)
%L             (0,0)o (2,0)o (2,2)c (0,2)c
%L          ]]
%L pws    = PwSpec.from(spec)
%L curve  = pws:topict()
%L p = PictList {
%L     [[ \def\closeddot{\circle*{0.4}}% ]],
%L     [[ \def\opendot  {\circle*{0.4}\color{white}\circle*{0.3}}% ]],
%L     Pict2e.region0(v(0,0), v(2,0), v(2,2), v(0,2)):Color("Orange"),
%L     curve:prethickness("3pt")
%L  }
%L p:pgat("pgatc"):preunitlength("20pt"):sa("Exercicio 2 exemplo"):output()
\pu

{\bf Um jogo colaborativo (2)}

\scalebox{0.60}{\def\colwidth{12cm}\firstcol{

Represente graficamente os seguintes conjuntos:
%
$$\begin{array}{rcl}
  A &=& \setofxyst{x∈[1,2), \; y∈[1,2)} \\
  B &=& \setofst{(x,2x)}{x∈[1,2)} \\
  C &=& \setofxyst{0≤x \;∧\; x+y<2} \\
  \end{array}
$$

Dica: todos eles vão dar subconjuntos do plano feitos de

infinitos pontos, e você vai ter que adaptar as convenções

que usamos pra desenhar intervalos pra desenhar {\sl regiões}.

\msk

Use bolinhas cheias pra indicar ``este ponto pertence ao

conjunto'', bolinhas ocas pra indicar ``este ponto não

pertence ao conjunto'', linhas grossas contínuas pra

indicar ``esse trecho da fronteira pertence ao conjunto''

e linhas tracejadas pra indicar ``esse trecho da fronteira

não pertence ao conjunto''. Por exemplo:
%
$$\ga{Exercicio 2 exemplo}$$

%}\anothercol{
}}

\newpage

% «dirichlet»  (to ".dirichlet")
% (c2m231srp 29 "dirichlet")
% (c2m231sra    "dirichlet")
% (c2m222tfcsp 9 "exercicio-4")
% (c2m222tfcsa   "exercicio-4")
% (c2m212somas2p 51 "dirichlet")
% (c2m212somas2a    "dirichlet")

% «integral-como-limite»  (to ".integral-como-limite")
% 2eT95 - Integral como limite
% (c2m221tfc1p 34 "descontinuidades")
% (c2m221tfc1p 36 "descontinuidades")
% (c2m221tfc1a    "descontinuidades")

% «TFC1»  (to ".TFC1")
% 2eT74 - TFC1:
% (c2m221tfc1p 15 "exemplo-1")
% (c2m221tfc1a    "exemplo-1")





\GenericWarning{Success:}{Success!!!}  % Used by `M-x cv'

\end{document}






% (find-dmirandacalcpage 212 "7.2. Integral definida")


% (c2m222srp 2 "somas-de-riemann-0")
% (c2m222sra   "somas-de-riemann-0")


\GenericWarning{Success:}{Success!!!}  % Used by `M-x cv'

\end{document}



% Local Variables:
% coding: utf-8-unix
% ee-tla: "c2sr"
% ee-tla: "c2m231sr"
% End:
