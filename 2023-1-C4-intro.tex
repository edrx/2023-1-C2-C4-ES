% (find-LATEX "2023-1-C4-intro.tex")
% (defun c () (interactive) (find-LATEXsh "lualatex -record 2023-1-C4-intro.tex" :end))
% (defun C () (interactive) (find-LATEXsh "lualatex 2023-1-C4-intro.tex" "Success!!!"))
% (defun D () (interactive) (find-pdf-page      "~/LATEX/2023-1-C4-intro.pdf"))
% (defun d () (interactive) (find-pdftools-page "~/LATEX/2023-1-C4-intro.pdf"))
% (defun e () (interactive) (find-LATEX "2023-1-C4-intro.tex"))
% (defun o () (interactive) (find-LATEX "2023-1-C4-intro.tex"))
% (defun u () (interactive) (find-latex-upload-links "2023-1-C4-intro"))
% (defun v () (interactive) (find-2a '(e) '(d)))
% (defun d0 () (interactive) (find-ebuffer "2023-1-C4-intro.pdf"))
% (defun cv () (interactive) (C) (ee-kill-this-buffer) (v) (g))
%          (code-eec-LATEX "2023-1-C4-intro")
% (find-pdf-page   "~/LATEX/2023-1-C4-intro.pdf")
% (find-sh0 "cp -v  ~/LATEX/2023-1-C4-intro.pdf /tmp/")
% (find-sh0 "cp -v  ~/LATEX/2023-1-C4-intro.pdf /tmp/pen/")
%     (find-xournalpp "/tmp/2023-1-C4-intro.pdf")
%   file:///home/edrx/LATEX/2023-1-C4-intro.pdf
%               file:///tmp/2023-1-C4-intro.pdf
%           file:///tmp/pen/2023-1-C4-intro.pdf
%  http://anggtwu.net/LATEX/2023-1-C4-intro.pdf
% (find-LATEX "2019.mk")
% (find-sh0 "cd ~/LUA/; cp -v Pict2e1.lua Pict2e1-1.lua Piecewise1.lua ~/LATEX/")
% (find-sh0 "cd ~/LUA/; cp -v Pict2e1.lua Pict2e1-1.lua Pict3D1.lua ~/LATEX/")
% (find-sh0 "cd ~/LUA/; cp -v C2Subst1.lua C2Formulas1.lua ~/LATEX/")
% (find-sh0 "cd ~/LUA/; cp -v Gram2.lua Tree1.lua Caepro5.lua ~/LATEX/")
% (find-MM-aula-links "2023-1-C4-intro" "C4" "c4m231intro" "c4mi")

% «.defs»		(to "defs")
% «.title»		(to "title")
% «.links»		(to "links")
% «.exercicio-1»	(to "exercicio-1")
% «.geogebra-1»		(to "geogebra-1")
% «.geogebra-2»		(to "geogebra-2")
% «.exercicio-3»	(to "exercicio-3")
% «.exercicio-4»	(to "exercicio-4")



% <videos>
% Video (not yet):
% (find-ssr-links     "c4m231intro" "2023-1-C4-intro")
% (code-eevvideo      "c4m231intro" "2023-1-C4-intro")
% (code-eevlinksvideo "c4m231intro" "2023-1-C4-intro")
% (find-c4m231introvideo "0:00")

\documentclass[oneside,12pt]{article}
\usepackage[colorlinks,citecolor=DarkRed,urlcolor=DarkRed]{hyperref} % (find-es "tex" "hyperref")
\usepackage{amsmath}
\usepackage{amsfonts}
\usepackage{amssymb}
\usepackage{pict2e}
\usepackage[x11names,svgnames]{xcolor} % (find-es "tex" "xcolor")
\usepackage{colorweb}                  % (find-es "tex" "colorweb")
%\usepackage{tikz}
%
% (find-dn6 "preamble6.lua" "preamble0")
%\usepackage{proof}   % For derivation trees ("%:" lines)
%\input diagxy        % For 2D diagrams ("%D" lines)
%\xyoption{curve}     % For the ".curve=" feature in 2D diagrams
%
\usepackage{edrx21}               % (find-LATEX "edrx21.sty")
\input edrxaccents.tex            % (find-LATEX "edrxaccents.tex")
\input edrx21chars.tex            % (find-LATEX "edrx21chars.tex")
\input edrxheadfoot.tex           % (find-LATEX "edrxheadfoot.tex")
\input edrxgac2.tex               % (find-LATEX "edrxgac2.tex")
%\usepackage{emaxima}              % (find-LATEX "emaxima.sty")
%
% (find-es "tex" "geometry")
\usepackage[a6paper, landscape,
            top=1.5cm, bottom=.25cm, left=1cm, right=1cm, includefoot
           ]{geometry}
%
\begin{document}

\catcode`\^^J=10
\directlua{dofile "dednat6load.lua"}  % (find-LATEX "dednat6load.lua")
%L dofile "Caepro5.lua"              -- (find-LATEX "Caepro5.lua")
\def\Caurl   #1{\expr{Caurl("#1")}}
\def\Cahref#1#2{\href{\Caurl{#1}}{#2}}
\def\Ca      #1{\Cahref{#1}{#1}}
\pu
%L V = nil                           -- (find-angg "LUA/Pict2e1.lua" "MiniV")
%L dofile "Piecewise1.lua"           -- (find-LATEX "Piecewise1.lua")
%L Pict2e.__index.suffix = "%"
\pu
\def\pictgridstyle{\color{GrayPale}\linethickness{0.3pt}}
\def\pictaxesstyle{\linethickness{0.5pt}}
\def\pictnaxesstyle{\color{GrayPale}\linethickness{0.5pt}}
\celllower=2.5pt

% %L dofile "QVis1.lua"                -- (find-LATEX "QVis1.lua")
% %L dofile "Pict3D1.lua"              -- (find-LATEX "Pict3D1.lua")
% %L dofile "C2Formulas1.lua"          -- (find-LATEX "C2Formulas1.lua")


% «defs»  (to ".defs")
% (find-LATEX "edrx21defs.tex" "colors")
% (find-LATEX "edrx21.sty")

\def\u#1{\par{\footnotesize \url{#1}}}

\def\drafturl{http://anggtwu.net/LATEX/2023-1-C4.pdf}
\def\drafturl{http://anggtwu.net/2023.1-C4.html}
\def\draftfooter{\tiny \href{\drafturl}{\jobname{}} \ColorBrown{\shorttoday{} \hours}}



%  _____ _ _   _                               
% |_   _(_) |_| | ___   _ __   __ _  __ _  ___ 
%   | | | | __| |/ _ \ | '_ \ / _` |/ _` |/ _ \
%   | | | | |_| |  __/ | |_) | (_| | (_| |  __/
%   |_| |_|\__|_|\___| | .__/ \__,_|\__, |\___|
%                      |_|          |___/      
%
% «title»  (to ".title")
% (c4m231introp 1 "title")
% (c4m231introa   "title")

\thispagestyle{empty}

\begin{center}

\vspace*{1.2cm}

{\bf \Large Cálculo 4 - 2023.1}

\bsk

Aulas 1 e 4: Introdução ao curso,

revisão de Cálculo 2 e Cálculo 3

\bsk

Eduardo Ochs - RCN/PURO/UFF

\url{http://anggtwu.net/2023.1-C4.html}

\end{center}

\newpage

%  _     _       _        
% | |   (_)_ __ | | _____ 
% | |   | | '_ \| |/ / __|
% | |___| | | | |   <\__ \
% |_____|_|_| |_|_|\_\___/
%                         
% «links»  (to ".links")
% (c2m231introp 16 "retas-reversas")
% (c2m231introa    "retas-reversas")

{\bf Links}


\scalebox{0.9}{\def\colwidth{12cm}\firstcol{

Página do curso:

\url{http://anggtwu.net/2023.1-C4.html}

\msk

\Ca{4gQ1} quadros da primeira aula

\Ca{Slogans01:10} até 08:51: sobre chutar e testar

\Ca{Slogans07:17} até 07:48: ...do tamanho de um apartamento

\Ca{Slogans1:11:02} até 1:17:42: seja o seu prório Geogebra

% \Ca{Visaud45:14} até 52:24: ajustar o nível de detalhe

\Ca{3fT16} (tipos, p.4): Tipos

\msk

\Ca{Leit6p17} (p.388: 6.3 Comprimento de arco)

\Ca{MirandaP301} (p.301: 9.5 Comprimento de arco)

\Ca{Stew8p3} (p.562: 8.1 Arc Length)

\Ca{Stew10p15} (p.672: 10.2, fig.4: Arc Length)

\Ca{Stew13p16} (p.877: 13.3 Arc Length and Curvature)

}\anothercol{
}}

\newpage

% «exercicio-1»  (to ".exercicio-1")
% (c4m231introp 3 "exercicio-1")
% (c4m231introa   "exercicio-1")

{\bf Exercício 1.}

\scalebox{0.55}{\def\colwidth{9.5cm}\firstcol{

    Digamos que a função $F(t)$ é a que eu desenhei no quadro na
    primeira aula. Ela obedecia

$$\begin{array}{rcl}
  F(0) &=& (1,1) \\
  F(1) &=& (2,1) \\
  F(2) &=& (3,2) \\
  F(3) &=& (3,3) \\
  \end{array}
$$

e o gráfico dela era formado por três segmentos de reta.

\msk

a) Encontre uma definição por casos pra $F(x)$ que ``tenha a forma da
função $F_1(t)$ da coluna da direita''. Note que você vai ter que
mudar todos os números da $F_1(t)$, e note que o modo normal, usual,
correto e formal de enunciar este problema seria usando variáveis ao
invés dos números $2, 3, \ldots, 17$... mas se eu disser ``troque
todos os números da definição pelos números corretos'' todo mundo
entende.

\msk

b) Faça a mesma coisa para a função $F_2(t)$.

\msk

c) Faça a mesma coisa para a função $F_3(t)$. Aqui há muitas soluções
possíveis; encontre uma na qual os números 4, 5, 6, 10, 11, 12, 17, 18
e 19 sejam trocados por números que tenham um significado geométrico e
olhométrico claro.

}\anothercol{

$$F_1(t) =
  \begin{cases}
     (2t+3, 4t+5)  & \text{$t<6$}, \\
     (7t+8, 9t+10) & \text{$11≤t≤12$}, \\
     (13t+14, 15t+16) & \text{$17<t$} \\
  \end{cases}
$$

$$F_2(t) =
  \begin{cases}
     (2,3) + t\VEC{4,5}  & \text{$t<6$}, \\
     (7,8) + t\VEC{9,10} & \text{$11≤t≤12$}, \\
     (13,14) + t\VEC{15,16} & \text{$17<t$} \\
  \end{cases}
$$

$$F_3(t) =
  \begin{cases}
     (2,3) + (t-4)\VEC{5,6}    & \text{$t<7$}, \\
     (8,9) + (t-10)\VEC{11,12} & \text{$13≤t≤14$}, \\
     (15,16) + (t-17)\VEC{18,19} & \text{$20<t$} \\
  \end{cases}
$$


}}

\newpage

%   ____             ____      _               
%  / ___| ___  ___  / ___| ___| |__  _ __ __ _ 
% | |  _ / _ \/ _ \| |  _ / _ \ '_ \| '__/ _` |
% | |_| |  __/ (_) | |_| |  __/ |_) | | | (_| |
%  \____|\___|\___/ \____|\___|_.__/|_|  \__,_|
%                                              
% «geogebra-1»  (to ".geogebra-1")
% (c4m231introp 4 "geogebra-1")
% (c4m231introa   "geogebra-1")

{\bf Seja o seu próprio GeoGebra}

\scalebox{0.58}{\def\colwidth{9cm}\firstcol{

    Na aula de 14/abril/2023 eu descobri que nenhuma das pessoas que
    veio sabia os truques do ``Seja o seu próprio GeoGebra''... a
    idéia está explicada por alto neste trecho de um vídeo:

    \ssk

    \Ca{Slogans1:11:02} até 1:17:42

    \bsk

    {\bf Exercício 2.}

    a) Relembre como usar esta notação de ``underbraces'' para
    escrever os resultados intermediários de uma expressão:
    %
    $$\def\und#1#2{\underbrace{#1}_{#2}}
      \und{(1,2)+\und{3\VEC{4,5}}{\VEC{12,15}}}{\VEC{13,17}}
    $$

    Dica: releia este slide:

    \ssk

    \Ca{3fT14} (p.2: C)

    \bsk

    b) Tente calcular de cabeça os pontos da reta $r$ -- definida à
    direita -- para estes valores de $t$: $t=0$, $t=1$, $t=4$, $t=5$,
    $t=1.23$. Para quais destes valores as contas são mais fáceis de
    fazer de cabeça?

}\anothercol{

  $$\begin{array}{rcl}
      r   &=& \setofst{(0,3)+(t-4)\VEC{2,0}}{t∈\R} \\
      r_1 &=& \setofst{(α,3)+(t-4)\VEC{2,0}}{t∈\R} \\
      r_2 &=& \setofst{(0,β)+(t-4)\VEC{2,0}}{t∈\R} \\
      r_3 &=& \setofst{(0,3)+(t-γ)\VEC{2,0}}{t∈\R} \\
      r_4 &=& \setofst{(0,3)+(t-4)\VEC{δ,0}}{t∈\R} \\
      r_5 &=& \setofst{(0,3)+(t-4)\VEC{2,ε}}{t∈\R} \\
    \end{array}
    $$

    \bsk

    c) Digamos que $α=5$ e que queremos desenhar a reta $r_1$
    desenhando dois pontos fáceis de calcular dela e escrevendo do
    lado de cada um deles o $t$ correspondente a eles. É fácil ver que
    o ponto com $t=1.23$ é difícil de calcular de cabeça. {\sl
      Descubra quais são os dois `$t$'s em que as contas são mais
      fáceis, desenhe estes dois pontos no plano, e desenhe o resto da
      reta.}

    \msk

    d) Use este truque dos pontos mais fáceis pra desenhar $r_1$
    quando $α=0$, quando $α=1$, e quando $α=2$. {\sl Descubra o que
      muda no desenho da $r_1$ quando o $α$ varia.}


}}

\newpage

% «geogebra-2»  (to ".geogebra-2")
% (c4m231introp 5 "geogebra-2")
% (c4m231introa   "geogebra-2")

{\bf Seja o seu próprio GeoGebra (2)}

\scalebox{0.6}{\def\colwidth{9.5cm}\firstcol{

    (Continuação do exercício 2...)

    \msk

    e) Use este truque dos pontos mais fáceis pra desenhar $r_2$
    quando $β=0$, quando $β=1$, e quando $β=2$. {\sl Descubra o que
      muda no desenho da $r_2$ quando o $β$ varia.}

    \msk

    f) Use este truque dos pontos mais fáceis pra desenhar $r_3$
    quando $γ=0$, quando $γ=1$, e quando $γ=2$. {\sl Descubra o que
      muda no desenho da $r_3$ quando o $γ$ varia. IMPORTANTE: aqui os
      `$t$'s mais fáceis vão ser diferentes para cada valor de $γ$.}

    \msk

    g) Use o truque dos pontos mais fáceis pra desenhar $r_4$ para
    três valores de $δ$ diferentes -- mas aqui você é que vai ter que
    escolher os valores de $δ$. IMPORTANTE: descubre três valores de
    $δ$ que deixam as contas e os desenhos bem fáceis de fazer, e use
    estes valores. Depois que você tiver feito os desenhos descubra o
    que no desenho da $r_4$ varia quando o $δ$ varia.

    \msk

    h) Use estes mesmos truques -- todos eles! -- pra desenhar a reta
    $r_5$ para três valores fáceis de $ε$ e para descobrir o que muda
    no desenho da $r_5$ quando o $ε$ varia.

}\anothercol{

    \vspace*{0.1cm}

    Digamos que a reta $r_6$ tem esta definição aqui,
    %
    $$\begin{array}{rcl}
        r_6 &=& \setofst{(α,β)+(t-γ)\VEC{δ,ε}}{t∈\R} \\
    \end{array}
    $$
    %
    e imagine que cada um dos parâmetros $α$, $β$, $γ$, $δ$ e $ε$ pode
    ser controlado por um slider, como neste trecho do vídeo:

    \ssk

    \Ca{Slogans1:11:32} até 1:11:59

    \bsk

    i) Releia tudo o que você fez até agora várias vezes, até você
    conseguir visualizar mentalmente, {\sl sem escrever nada e
      (quase?) sem fazer contas de cabeça,} como a reta $r_6$ muda
    quando você varia os parâmetros $α$, $β$, $γ$, $δ$ e $ε$.

    \msk

    j) Descubra, {\sl sem escrever nada e quase sem fazer contas de
      cabeça}, qual é a reta desta forma aqui
    %
    $$\begin{array}{rcl}
        r_7 &=& \setofst{(α,β)+(t-γ)\VEC{δ,ε}}{t∈\R} \\
    \end{array}
    $$

    que passa pelo ponto $(2,5)$ quando $t=6$ e pelo ponto
    $(2+20,5+42)$ quando $t=7$; quando você conseguir uma hipótese
    bastante boa escreva-a e teste-a.




}}






\newpage

% «exercicio-3»  (to ".exercicio-3")
% (c4m231introp 6 "exercicio-3")
% (c4m231introa   "exercicio-3")

{\bf Exercício 3.}

\scalebox{0.65}{\def\colwidth{10cm}\firstcol{

    Seja $F(t)$ a função do exercício 1, e digamos que
    $F(t) = (x(t),y(t))$.

    \bsk
    
    a) Faça o gráfico da função $x(t)$.
    
    b) Faça o gráfico da função $y(t)$.
    
    \msk
    
    c) Dê definições por casos das funções $x(t)$ e $y(t)$ em formatos
    parecidos com o da $F_3(t)$ do exercício 1, em que cada número tinha
    um significado geométrico e olhométrico claro.
    
    \msk
    
    d) Calcule $\Intt{0}{3}{x(t)}$ e $\Intt{0}{3}{y(t)}$ só olhando pros
    gráficos delas e contando quadrados e triângulos.
    
    \bsk
    
    Agora reveja as definições de somas de Riemann, partições, e dos
    métodos [L] e [R] nestes links aqui...
    
    \msk
    
    \Ca{2dT178} (def-integral, p.14) Partição preferida
    
    \Ca{2fT67} (somas-de-riemann, p.8) métodos a e b
    
    \Ca{2fT91} (TFC1-e-TFC2, p.3) A definição de partição
    
    \Ca{2eT34} (somas-3, p.13) Métodos L e R
    
    \Ca{2fT125} (P2, p.4) Métodos L e R

}\anothercol{

  ...e represente graficamente cada uma
  
  destas somas de retângulos:
  
  \msk
  
  e) $[L]_{\{0,1,2,3\}}$
  
  f) $[L]_{\{0,1,1.5, 2,3\}}$
  
  g) $[R]_{\{0,1,2,3\}}$
  
  h) $[R]_{\{0,1,1.5, 2,3\}}$
  
  i) $[L]_{\{0, 0.25, 0.5, \ldots, 3\}}$
  
  j) $[R]_{\{0, 0.25, 0.5, \ldots, 3\}}$

}}

\newpage

% «exercicio-4»  (to ".exercicio-4")
% (c4m231introp 7 "exercicio-4")
% (c4m231introa   "exercicio-4")

{\bf Exercício 4.}

\scalebox{0.9}{\def\colwidth{10.5cm}\firstcol{

Leia isto aqui:

\Ca{Stew10p15} (p.672: Arc Length)

\msk

a) Calcule o comprimento de arco da curva $F(t)$ entre $t=0$ e $t=3$
no olhômetro.

\msk

b) Faça um desenho parecido com o da figura 4 dessa página para a
curva $F(t)$. Considere que $t_0=0$, $t_1=1$, $t_2=2$ e $t_3=3$.

\msk

c) Escreva a sua idéia do item (a) como uma soma de três raizes
quadradas -- como se você tivesse pego o somatório da última linha
dessa página e expandido ele.

\msk

d) Agora reescreva o que você fez no item (c) usando o `$\sum$'.





}\anothercol{
}}




\GenericWarning{Success:}{Success!!!}  % Used by `M-x cv'

\end{document}


% Local Variables:
% coding: utf-8-unix
% ee-tla: "c4mi"
% ee-tla: "c4m231intro"
% End:
